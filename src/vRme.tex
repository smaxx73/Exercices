\titre{Séries de Bertrand}
\theme{séries}
\auteur{}
\organisation{AMSCC}



\texte{ Soient $\alpha \in \R$ et $\beta \in \R$ deux paramètres réels. On s'intéresse à la série $\displaystyle \sum_{n \geq 2} u_n$ avec 
$$u_n = \frac{1}{n^\alpha \left(\ln(n)\right)^\beta}$$
qui s'appelle une série de Bertrand. Sa convergence dépend des valeurs prises par $\alpha$ et $\beta$. }
\begin{enumerate}
	\item \texte{ Supposons que $\alpha >1$ ($\beta$ est quelconque).  On pose $\gamma = \frac{1+\alpha}{2}$. }
	\begin{enumerate}
		\item \question{ Vérifier que $\gamma < \alpha$. Que peut-on dire de la série $\sum \frac{1}{n^\gamma}$ ? }
\reponse{On observe que $\alpha >1$ d'où $2\alpha = \alpha + \alpha > 1+\alpha = 2\gamma$ d'où $\alpha > \gamma$. De plus, $\alpha > 1$ d'où $1+\alpha > 2$ d'où $\frac{1+\alpha}{2} = \gamma > 1$. }
		\item \question{ Vérifier que $\lim\limits_{n \to +\infty} n^{\gamma} u_n = 0$. }
\reponse{Si $\beta \geq 0$ alors on peut majorer $n^{\gamma} u_n = \frac{1}{n^{\alpha-\gamma} \left(\ln(n)\right)^\beta} \leq  \frac{1}{n^{\alpha-\gamma}} \xrightarrow[n \to +\infty]{} 0$ d'où le résultat. Enfin, si $\beta <0$ alors par croissances comparées, $\frac{ \left(\ln(n)\right)^{-\beta}}{n^{\alpha-\gamma}}  \xrightarrow[n \to +\infty]{} 0$. }
		\item \question{ Conclure sur la convergence de   $\displaystyle \sum_{n \geq 1} u_n$. }
\reponse{Donc par définition, $u_n = o\left(\frac{1}{n^{\gamma}}\right)$ or $\sum \frac{1}{n^\gamma}$ est une série convergente donc par comparaison de séries à termes positifs, $\sum u_n$ est également une série convergente. }
	\end{enumerate}
\item \question{  Supposons que $\alpha <1$ : calculer $\lim\limits_{n \to +\infty} n^{} u_n$ et  conclure sur la convergence de   $\displaystyle \sum_{n \geq 1} u_n$. }
\item \texte{ Supposons que $\alpha = 1$ et $\beta \leq 0$.  }
\begin{enumerate}
	\item \question{ Montrer qu'il existe un rang $N$ à partir duquel pour tout $n \geq N$ alors $\frac{1}{ \left(\ln(n)\right)^\beta} \geq 1$.  }
	\item   \question{ Conclure sur la convergence de   $\displaystyle \sum_{n \geq 1} u_n$. }
\end{enumerate}
\item **  \texte{ Supposons que $\alpha = 1$ et $\beta > 0$. Pour tout $x >1$, On pose $f(x) = \frac{1}{x \ln(x)^\beta}$ de sorte que $u_n = f(n)$ pour tout $n \geq 2$.  }
\begin{enumerate}
	\item \question{ Vérifier que $f$ est positive et décroissante sur $]1+\infty[$. }
	\item \question{ On considère l'intégrale $$I = \int_2^{+\infty} f(t)dt$$
	En effectuant le changement de variable $u = \ln(t)$, montrer que $I$ est une intégrale convergente si et seulement si $\beta >1$.  }
	\item  \question{ Conclure sur la convergence de   $\displaystyle \sum_{n \geq 1} u_n$. }
\end{enumerate}
\end{enumerate}
