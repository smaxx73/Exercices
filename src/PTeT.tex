\uuid{PTeT}
\titre{Comparaison de deux systèmes de détection d'intrusion}

\niveau{L3} 				
\module{Statistiques} 	
\chapitre{Estimation}   			
\sousChapitre{Intervalles de confiance}			

\theme{Intervalle de confiance, différence de proportions, théorème central limite}				
\auteur{Maxime Nguyen}
\datecreate{2026-01-07}
\organisation{AMSCC}			
\difficulte{3}			

\contenu{
	
	\texte{ 
		Une entreprise de cybersécurité souhaite comparer l'efficacité de deux systèmes de détection d'intrusion (IDS) : le système historique \textbf{IDS-A} et un nouveau système basé sur l'intelligence artificielle \textbf{IDS-B}. 
		
		Pour évaluer ces systèmes, l'équipe de sécurité a réalisé des tests en conditions réelles sur deux environnements identiques pendant une période d'un mois. Les deux systèmes ont été confrontés à des scénarios d'attaques variés (injections SQL, XSS, scans de ports, déni de service, etc.).
		
		On note les variables aléatoires $X_A$ donnant le nombre d'attaques détectées sur IDS-A et $X_B$ donnant le nombre d'attaques détectées sur IDS-B. 
		
		\textbf{Résultats des tests :}
		\begin{itemize}
			\item \textbf{IDS-A} : Sur $n_A = 250$ attaques, le système a détecté $X_A = 195$ attaques
			\item \textbf{IDS-B} : Sur $n_B = 280$ attaques, le système a détecté $X_B = 238$ attaques
		\end{itemize}
		
		On note $p_A$ le taux de détection réel du système IDS-A et $p_B$ celui du système IDS-B. L'objectif est de construire un intervalle de confiance à 95\% pour la différence $\Delta = p_B - p_A$ afin de quantifier le gain de performance apporté par le nouveau système.
		
		On supposera que les échantillons sont indépendants et que les conditions d'application du théorème central limite sont vérifiées.
	}
	
	\begin{enumerate}
		\item   \question{Choisir des estimateurs  sans biais $\hat{p}_A$ et $\hat{p}_B$ des taux de détection. En déduire une estimation sans biais de la différence $\Delta = p_B - p_A$.}
		\indication{Les estimateurs naturels des proportions sont les fréquences empiriques.}
		\reponse{
			Les estimateurs ponctuels des taux de détection sont :
			$$\hat{p}_A = \frac{X_A}{n_A} = \frac{195}{250} = 0{,}78 = 78\%$$
			$$\hat{p}_B = \frac{X_B}{n_B} = \frac{238}{280} = 0{,}85 = 85\%$$
			
			L'estimateur ponctuel de la différence est :
			$$\hat{\Delta} = \hat{p}_B - \hat{p}_A = 0{,}85 - 0{,}78 = 0{,}07 = 7\%$$
			
			Le nouveau système IDS-B semble avoir un taux de détection supérieur de 7 points de pourcentage par rapport au système historique IDS-A.
		}
		\item \question{ Donner la loi de chaque variable aléatoire $X_A$ et $X_B$. }
		\reponse{ $X_A \sim \mathcal{B}(n_A, p_A)$ et $X_B \sim \mathcal{B}(n_B, p_B)$. }
		\item   \question{Rappeler la variance de l'estimateur $\hat{p}_A$ (respectivement $\hat{p}_B$) en fonction de $p_A$ et $n_A$ (respectivement $p_B$ et $n_B$).}
		\reponse{
			Pour une proportion estimée par la fréquence empirique, la variance est :
			$$\text{Var}(\hat{p}_A) = \frac{p_A(1-p_A)}{n_A}$$
			$$\text{Var}(\hat{p}_B) = \frac{p_B(1-p_B)}{n_B}$$
			
			Ces formules proviennent du fait que $X_A \sim \mathcal{B}(n_A, p_A)$ et $X_B \sim \mathcal{B}(n_B, p_B)$.
		}
		
		\item   \question{En utilisant l'indépendance des deux échantillons, exprimer la variance de l'estimateur $\hat{\Delta} = \hat{p}_B - \hat{p}_A$ en fonction de $p_A$, $p_B$, $n_A$ et $n_B$.}
		\indication{Pour deux variables aléatoires indépendantes $X$ et $Y$, on a $\text{Var}(X - Y) = \text{Var}(X) + \text{Var}(Y)$.}
		\reponse{
			Comme les échantillons sont indépendants, $\hat{p}_A$ et $\hat{p}_B$ sont indépendants. On a donc :
			\begin{align*}
				\text{Var}(\hat{\Delta}) &= \text{Var}(\hat{p}_B - \hat{p}_A) \\
				&= \text{Var}(\hat{p}_B) + \text{Var}(-\hat{p}_A) \\
				&= \text{Var}(\hat{p}_B) + \text{Var}(\hat{p}_A) \\
				&= \frac{p_B(1-p_B)}{n_B} + \frac{p_A(1-p_A)}{n_A}
			\end{align*}
			
			L'écart-type de $\hat{\Delta}$ est donc :
			$$\sigma_{\hat{\Delta}} = \sqrt{\frac{p_B(1-p_B)}{n_B} + \frac{p_A(1-p_A)}{n_A}}$$
		}
		
		\item   \question{En pratique, les valeurs $p_A$ et $p_B$ sont inconnues. Proposer une estimation $\hat{\sigma}_{\hat{\Delta}}$ de l'écart-type de $\hat{\Delta}$.}
		\indication{Remplacer les paramètres inconnus par leurs estimations.}
		\reponse{
			On estime l'écart-type de $\hat{\Delta}$ en remplaçant $p_A$ et $p_B$ par leurs estimations $\hat{p}_A$ et $\hat{p}_B$ :
			$$\hat{\sigma}_{\hat{\Delta}} = \sqrt{\frac{\hat{p}_B(1-\hat{p}_B)}{n_B} + \frac{\hat{p}_A(1-\hat{p}_A)}{n_A}}$$
			
			Application numérique :
			\begin{align*}
				\hat{\sigma}_{\hat{\Delta}} &= \sqrt{\frac{0{,}85 \times 0{,}15}{280} + \frac{0{,}78 \times 0{,}22}{250}} \\
				&= \sqrt{\frac{0{,}1275}{280} + \frac{0{,}1716}{250}} \\
				&= \sqrt{0{,}000455 + 0{,}000686} \\
				&= \sqrt{0{,}001141} \\
				&\approx 0{,}0338
			\end{align*}
			
			L'écart-type estimé de $\hat{\Delta}$ est donc d'environ $0{,}034$ ou $3{,}4\%$.
		}
		
		\item   \question{D'après le théorème central limite, quelle est la loi approximative de la variable aléatoire $\frac{\hat{\Delta} - \Delta}{\sigma_{\hat{\Delta}}}$ pour des tailles d'échantillons suffisamment grandes ?}
		\indication{Le TCL s'applique car $\hat{\Delta}$ est une somme (avec coefficients) de variables aléatoires.}
		\reponse{
			D'après le théorème central limite, pour des tailles d'échantillons suffisamment grandes (conditions vérifiées ici avec $n_A = 250$ et $n_B = 280$), on a :
			$$\frac{\hat{\Delta} - \Delta}{\sigma_{\hat{\Delta}}} \xrightarrow{\mathcal{L}} \mathcal{N}(0, 1)$$
			
			Autrement dit, la variable aléatoire $\hat{\Delta}$ suit approximativement une loi normale :
			$$\hat{\Delta} \sim \mathcal{N}\left(\Delta, \sigma_{\hat{\Delta}}^2\right)$$
			
			En pratique, comme $\sigma_{\hat{\Delta}}$ est inconnu, on le remplace par son estimation $\hat{\sigma}_{\hat{\Delta}}$, et on utilise :
			$$\frac{\hat{\Delta} - \Delta}{\hat{\sigma}_{\hat{\Delta}}} \overset{\text{approx.}}{\sim} \mathcal{N}(0, 1)$$
		}
		
		\item   \question{En utilisant le résultat précédent, construire un intervalle de confiance à 95\% pour la différence $\Delta = p_B - p_A$. Interpréter ce résultat du point de vue de l'équipe de cybersécurité.}
		\indication{Pour une loi normale centrée réduite, $\mathbb{P}(-1{,}96 \leq Z \leq 1{,}96) = 0{,}95$.}
		\reponse{
			Pour un niveau de confiance de 95\%, on utilise le quantile $z_{0{,}975} = 1{,}96$ de la loi normale centrée réduite.
			
			On a :
			$$\mathbb{P}\left(-1{,}96 \leq \frac{\hat{\Delta} - \Delta}{\hat{\sigma}_{\hat{\Delta}}} \leq 1{,}96\right) \approx 0{,}95$$
			
			En isolant $\Delta$, on obtient :
			$$\mathbb{P}\left(\hat{\Delta} - 1{,}96 \times \hat{\sigma}_{\hat{\Delta}} \leq \Delta \leq \hat{\Delta} + 1{,}96 \times \hat{\sigma}_{\hat{\Delta}}\right) \approx 0{,}95$$
			
			L'intervalle de confiance à 95\% pour $\Delta$ est donc :
			$$IC_{95\%}(\Delta) = \left[\hat{\Delta} - 1{,}96 \times \hat{\sigma}_{\hat{\Delta}} ; \hat{\Delta} + 1{,}96 \times \hat{\sigma}_{\hat{\Delta}}\right]$$
			
			Application numérique :
			\begin{align*}
				IC_{95\%}(\Delta) &= [0{,}07 - 1{,}96 \times 0{,}0338 ; 0{,}07 + 1{,}96 \times 0{,}0338] \\
				&= [0{,}07 - 0{,}0662 ; 0{,}07 + 0{,}0662] \\
				&= [0{,}0038 ; 0{,}1362]
			\end{align*}
			
			Soit environ $[0{,}38\% ; 13{,}62\%]$.
			
			\textbf{Interprétation pour l'équipe de cybersécurité :}
			
			Avec un niveau de confiance de 95\%, l'amélioration du taux de détection apportée par le système IDS-B par rapport au système IDS-A se situe entre 0,38 et 13,62 points de pourcentage.
			
			\textbf{Conclusions importantes :}
			\begin{itemize}
				\item L'intervalle de confiance est entièrement positif (ne contient pas 0), ce qui prouve statistiquement que le système IDS-B est significativement plus performant que le système IDS-A.
				\item L'amélioration minimale attendue est d'au moins 0,38\%, ce qui justifie le déploiement du nouveau système.
				\item L'amélioration pourrait aller jusqu'à 13,62\%, ce qui représenterait un gain majeur en termes de sécurité.
				\item L'estimation ponctuelle de 7\% d'amélioration est au milieu de cet intervalle, ce qui renforce la confiance dans cette valeur.
			\end{itemize}
			
			L'équipe peut donc recommander le déploiement du système IDS-B avec une forte assurance statistique de son efficacité supérieure.
		}
		
	\end{enumerate}
	
}