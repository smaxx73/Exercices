\uuid{5cbY}
\chapitre{Statistique}
\niveau{L2}
\module{Probabilité et statistique}
\sousChapitre{Tests d'hypothèses, intervalle de confiance}
\titre{Intervalles de confiance (Corrigé)}
\theme{statistiques, intervalle de confiance}
\auteur{Maxime NGUYEN}
\datecreate{2022-09-07}
\organisation{AMSCC}
\difficulte{}
\contenu{
	
	\texte{Pour étudier le degré de pollution des eaux d'une rivière par les déchets d'une usine, on mesure la teneur en milligrammes d'un certain polluant par litre d'eau.}
	
	\begin{enumerate}
		\item \question{On effectue 17 mesures pour lesquelles les valeurs observées des teneurs en polluant ont une moyenne de 5.2~mg/L et un écart-type de 1.2~mg/L. 
			\begin{enumerate}
				\item Donner une estimation sans biais de la moyenne et de l'écart type de la teneur en polluant dans cette rivière.
				\item Donner un intervalle de confiance au seuil de 5\% pour la teneur moyenne en polluant de cette rivière.
				\item A partir de ces résultats, peut-on considérer que l'usine respecte la législation en vigueur selon laquelle cette teneur moyenne en polluant ne doit pas dépasser 7~mg/L ?
			\end{enumerate}
		}
		\reponse{
			\begin{enumerate}
				\item La moyenne observée sur l'échantillon, $\overline{x}=5.2$~mg/L, est une estimation ponctuelle et sans biais de la moyenne de la teneur en polluant.
				L'écart-type observé de 1.2~mg/L est une estimation biaisée. Pour obtenir une estimation non biaisée, on calcule d'abord la variance non biaisée (ou corrigée) $s^2$ :
				$s^2 = \frac{n}{n-1} \times (\text{écart-type obs.})^2 = \frac{17}{16} \times (1.2)^2 = 1.53$.
				L'estimation non biaisée de l'écart-type est donc $s = \sqrt{1.53} \approx 1.237$~mg/L.
				
				\item L'écart-type de la population est inconnu et la taille de l'échantillon est faible ($n=17<30$), on utilise donc une loi de Student à $n-1=16$ degrés de liberté. On suppose que la distribution de la teneur en polluant est normale.
				Le quantile $t_{16; 0.025}$ est d'environ $2.12$.
				L'intervalle de confiance est donné par $\overline{x} \pm t_{n-1; \alpha/2} \frac{s}{\sqrt{n}}$.
				$I.C._{95\%} = \left[ 5.2 - 2.12 \times \frac{1.237}{\sqrt{17}} \,;\, 5.2 + 2.12 \times \frac{1.237}{\sqrt{17}} \right] \approx [5.2 - 0.636 \,;\, 5.2 + 0.636]$.
				L'intervalle de confiance est donc environ \textbf{[4.56~;~5.84]~mg/L}.
				
				\item La borne supérieure de l'intervalle de confiance, 5.84~mg/L, est inférieure à la limite légale de 7~mg/L. Au niveau de confiance de 95\%, on peut donc considérer que l'usine respecte la législation.
			\end{enumerate}
		}
		\item \question{On suppose ici l'écart type de la population connu et égal à 1.2~mg/L. Combien de mesures devrait-on faire pour estimer la teneur moyenne en polluant avec une précision de 0.1~mg/L au niveau de confiance 95\% ?}
		\indication{Attention, l'écart-type de la population $\sigma$ est ici supposé connu. Quelle loi statistique doit-on utiliser dans ce cas, quelle que soit la taille de l'échantillon ? De plus, la "précision" correspond à la marge d'erreur, c'est-à-dire au demi-intervalle.}
		\reponse{
			L'écart-type de la population $\sigma$ étant connu ($\sigma=1.2$), on utilise la loi normale centrée réduite. 
			La précision souhaitée $d=0.1$~mg/L correspond à la marge d'erreur (ou demi-longueur de l'intervalle).
			La formule de la marge d'erreur est $d = z_{\alpha/2} \frac{\sigma}{\sqrt{n}}$.
			
			On cherche $n$ tel que $d \leq 0.1$. Pour un niveau de confiance de 95\%, $z_{\alpha/2} = z_{0.025} \approx 1.96$.
			
			On doit donc résoudre l'inéquation :
			$1.96 \times \frac{1.2}{\sqrt{n}} \leq 0.1$
			
			Soit $\sqrt{n} \geq \frac{1.96 \times 1.2}{0.1}$
			
			$\sqrt{n} \geq 23.52$
			
			$n \geq (23.52)^2$
			
			$n \geq 553.19$
			
			Comme $n$ doit être un entier, on doit effectuer au minimum \fbox{$n=554$} mesures.
			
			\href{https://stcyrterrenetdefensegouvf-my.sharepoint.com/:x:/g/personal/maxime_nguyen_st-cyr_terre-net_defense_gouv_fr/Ec9ve_iQd-pNn_Tf86GSa8EBJUrDEkRpISpMW4xkp23PeQ?e=oOly0h}{Détail des calculs sur tableur}
		}
	\end{enumerate}
}