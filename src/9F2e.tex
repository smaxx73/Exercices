\uuid{9F2e}
\titre{Loi jointe et indépendance d'un couple de variables aléatoires}
\niveau{L2}
\module{Probabilité et statistique}
\chapitre{Probabilité discrète}
\sousChapitre{Variable aléatoire discrète}
\theme{Loi jointe, loi marginale, espérance, variance, indépendance}
\auteur{}
\datecreate{2025-09-24}
\organisation{}
\difficulte{4}
\contenu{
	\texte{
		Soit \((X, Y)\) un couple de variables aléatoires dont la loi jointe est donnée par le tableau suivant :
		
		\begin{center}
			\begin{tabular}{|c|c|c|c|}
				\hline
				\(X \backslash Y\) & -1 & 0 & 1 \\
				\hline
				-1 & 0 & \(a\) & 0 \\
				\hline
				0 & \(a\) & 0 & \(a\) \\
				\hline
				1 & 0 & \(a\) & 0 \\
				\hline
			\end{tabular}
		\end{center}
	}
	\begin{enumerate}
		\item \question{Déterminer la valeur de \(a\).}
		\item \question{Quelles sont les lois marginales du couple \((X, Y)\) ?}
		\item \question{Calculer l'espérance et la variance de la variable aléatoire \(X\).}
		\item \question{Les variables aléatoires \(X\) et \(Y\) sont-elles indépendantes ?}
	\end{enumerate}
	\texte{
		On pose \(U = X + Y\) et \(V = X - Y\).
	}
	\begin{enumerate}
		\setcounter{enumi}{4}
		\item \question{Donner la loi jointe du couple \((U, V)\).}
		\item \question{En déduire les lois marginales du couple \((U, V)\).}
		\item \question{Les variables \(U\) et \(V\) sont-elles indépendantes ?}
	\end{enumerate}
}


