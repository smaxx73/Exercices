\uuid{DlAQ}
\exo7id{5354}
\auteur{rouget}
\organisation{exo7}
\datecreate{2010-07-04}
\isIndication{false}
\isCorrection{true}
\chapitre{Groupe, anneau, corps}
\sousChapitre{Groupe de permutation}

\contenu{
\texte{
\label{exo:suprou2ter}
Démontrer que $S_n$ est engendré par $\tau_{1,2}$, $\tau_{1,3}$,...,$\tau_{1,n}$.
}
\reponse{
$(S_n,\circ)$ est engendré par les transpositions. Il suffit donc de montrer que pour $2\leq i<j\leq n$, la transposition $\tau_{i,j}$ est produit des $\tau_{1,k}$, $2\leq k\leq n$.

Mais $\tau_{1,i}\circ\tau_{1,j}\circ\tau_{1,i}=(i 1 j)(j i 1)(i 1 j)=(1 i j)=\tau_{i,j}$ ce qu'il fallait démontrer.
}
}
