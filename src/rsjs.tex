\uuid{rsjs}
\chapitre{Optimisation}
\sousChapitre{Autre}
\titre{Optimisation quadratique, moindres carrés}
\theme{optimisation}
\auteur{}
\datecreate{2024-10-15}
\organisation{AMSCC}
\difficulte{4}
\contenu{


\texte{
  Soit $N \in \mathbb{N}^*$, et $\{(t_i, x_i)\}_{1 \leq i \leq N}$ un nuage de points. On cherche à déterminer les coefficients $a$, $b$ et $c$ de la parabole $P$ d'équation $y = at^2 + bt + c$ qui minimise la somme des carrés des distances des points $(t_i, x_i)$ à cette parabole. 
}

\begin{enumerate}
  \item \question{Écrire ce problème comme un problème de minimisation quadratique, c’est-à-dire un problème de la forme
  \[
  \inf_{X \in \mathbb{R}^n} J(X) \quad \text{avec} \quad J(X) = \frac{1}{2} \langle AX, X \rangle - \langle b, X \rangle,
  \]
  avec $A \in \mathcal{S}_n(\mathbb{R})$, $b \in \mathbb{R}^n$. On devra donc expliciter $n$, $A$, et $b$. On utilisera la notation $S_k = \sum_{i=1}^{N} t_i^k$.}
    \indication{Exprimer la somme des carrés comme le carré de la norme euclidienne d'un vecteur résidu $R = k - MX$. Développer $\|R\|^2$ en utilisant les propriétés du produit scalaire et de la transposée matricielle pour faire apparaître une forme quadratique en $X$.}
  \reponse{Le problème est celui de la régression parabolique pour un nuage de points $\{(t_i, x_i)\}_{1 \leq i \leq N}$, où l'on cherche la parabole $P$ d'équation $y = at^2 + bt + c$ qui minimise la somme des carrés des distances des points $(t_i, x_i)$ à cette parabole. Le problème peut s'écrire :
  \[
  \inf_{X \in \mathbb{R}^3} J(X) \quad \text{avec} \quad X = \begin{pmatrix} a \\ b \\ c \end{pmatrix} \quad \text{et} \quad J(X) = \sum_{i=1}^{N} (x_i - at_i^2 - bt_i - c)^2.
  \]
  En écrivant $J(X) = \|MX - k\|^2$ avec $M = \begin{pmatrix} t_1^2 & t_1 & 1 \\ \vdots & \vdots & \vdots \\ t_N^2 & t_N & 1 \end{pmatrix}$ et $k = \begin{pmatrix} x_1 \\ \vdots \\ x_N \end{pmatrix}$, on obtient
  \[
  J(X) = \frac{1}{2} \langle AX, X \rangle - \langle b, X \rangle
  \]
  avec $A = M^T M$ et $b = M^T k$. La matrice $A$ est donc donnée par :
  \[
  A = \begin{pmatrix} S_4 & S_3 & S_2 \\ S_3 & S_2 & S_1 \\ S_2 & S_1 & N \end{pmatrix}.
  \]}
  
  \item \question{Discuter de l’existence des solutions d’un tel problème.}
      \indication{On peut voir le problème géométriquement comme la recherche du point dans l'image de $M$ le plus proche du vecteur $k$. Alternativement, on peut montrer que la fonctionnelle $J$ est continue et coercive sur $\mathbb{R}^3$.}
  \reponse{Ce problème est équivalent à celui de minimiser la distance euclidienne de $k$ au sous-espace vectoriel $\text{Im}(M)$, qui est de dimension finie. Il s’agit donc d’un problème de projection orthogonale, qui admet toujours une solution.}
  
  \item \question{On suppose que la matrice $A$ est définie positive. Démontrer que ce problème possède une unique solution.}
  
  \reponse{Si $A$ est définie positive, alors la fonction $J(X)$ est strictement convexe. Par conséquent, $J(X)$ possède un unique minimum sur $\mathbb{R}^n$, donc le problème admet une unique solution.}
  
  \item    \question{À quelle condition sur les points $\{t_i\}_{1 \leq i \leq N}$ la matrice $A$ est-elle définie positive ? En déduire une condition nécessaire et suffisante sur le nuage de points pour que le problème admette une unique solution.}
  
\reponse{
	La matrice $A = M^T M$ est symétrique. Elle est définie positive si et seulement si pour tout $X \in \mathbb{R}^3 \setminus \{0\}$, $\langle AX, X \rangle > 0$. Or,
	\[ \langle AX, X \rangle = \langle M^T M X, X \rangle = \langle MX, MX \rangle = \|MX\|_2^2. \]
	Ainsi, $A$ est définie positive si et seulement si $\ker(M) = \{0\}$.
	
	Le noyau de $M$ est l'ensemble des vecteurs $X = (a, b, c)^T$ tels que $MX=0$, ce qui équivaut au système $at_i^2 + bt_i + c = 0$ pour tout $i \in \{1, \dots, N\}$.
	Cette condition signifie que le polynôme du second degré $P(t) = at^2 + bt + c$ admet pour racines toutes les valeurs distinctes présentes dans l'ensemble $\{t_1, \dots, t_N\}$.
	
	Un polynôme de degré au plus 2 et non nul ne peut avoir plus de deux racines distinctes. Par conséquent, si l'ensemble $\{t_i\}$ contient au moins trois valeurs distinctes, le polynôme $P(t)$ doit être le polynôme nul, ce qui implique $a=b=c=0$, soit $X=0$. Dans ce cas, $\ker(M)=\{0\}$.
	
	Inversement, si l'ensemble $\{t_i\}$ contient seulement une ou deux valeurs distinctes, il est possible de construire un polynôme $P(t)$ non nul de degré au plus 2 qui s'annule en ces valeurs, fournissant un vecteur $X \neq 0$ dans le noyau de $M$.
	
	\textbf{Conclusion :} La matrice $A$ est définie positive si et seulement si l'ensemble des abscisses $\{t_1, \dots, t_N\}$ contient au moins trois valeurs distinctes.
	Comme l'unicité du minimum de la fonctionnelle $J(X)$ est assurée par sa stricte convexité (qui équivaut au fait que $A$ soit définie positive), c'est aussi la condition nécessaire et suffisante pour que le problème admette une solution unique.
}
\end{enumerate}
}
