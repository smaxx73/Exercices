	\uuid{fsT2}
	\titre{Contrôle qualité : lois dérivées et approximations}
	\niveau{L2} 
	\module{Probabilités et Statistiques} 
	\chapitre{Variables aléatoires}
	\sousChapitre{Loi normale, loi du chi-deux, approximations}
	\theme{Théorème central limite, loi du $\chi^2$, approximation d'une loi binomiale}
	\auteur{AMSCC}
	\datecreate{2025-12-02}
	\organisation{AMSCC}
	\difficulte{3}
	\contenu{
		\texte{ 
			Dans un atelier, on produit des boulons dont le diamètre $D$ (en mm) suit une loi normale $\mathcal{N}(10, \sigma=0.2)$. 
		}
		
		\subsection*{Partie A - Échantillonnage}
		
		\texte{
			On prélève un échantillon aléatoire de $n = 20$ boulons. On note $D_i$ le diamètre du boulon $i$ ($1 \leq i \leq 20$).
		}
		
		\begin{enumerate}
			\item   \question{Déterminer la loi suivie par la moyenne empirique :
				$$\bar{D} = \frac{1}{20}\sum_{i=1}^{20} D_i$$
				Préciser son espérance et sa variance. Justifier.}
			\indication{Utiliser la stabilité de la loi normale par somme et combinaison linéaire.}
			\reponse{
				Les variables aléatoires $D_i$ sont indépendantes et suivent toutes la loi $\mathcal{N}(10, 0.2)$.
				
				\textbf{Espérance :} Par linéarité de l'espérance :
				$$\mathbb{E}(\bar{D}) = \mathbb{E}\left(\frac{1}{20}\sum_{i=1}^{20} D_i\right) = \frac{1}{20}\sum_{i=1}^{20} \mathbb{E}(D_i) = \frac{1}{20} \times 20 \times 10 = 10 \text{ mm}$$
				
				\textbf{Variance :} Par indépendance des $D_i$ :
				$$\text{Var}(\bar{D}) = \text{Var}\left(\frac{1}{20}\sum_{i=1}^{20} D_i\right) = \frac{1}{20^2}\sum_{i=1}^{20} \text{Var}(D_i)$$
				Sachant que l'écart-type est $\sigma=0.2$, la variance est $\text{Var}(D_i) = 0.2^2 = 0.04$.
				$$\text{Var}(\bar{D}) = \frac{1}{400} \times 20 \times 0.04 = \frac{0.8}{400} = 0.002$$
				
				\textbf{Loi de $\bar{D}$ :}
				La moyenne empirique d'un échantillon i.i.d. gaussien suit une loi normale d'espérance $\mu$ et d'écart-type $\frac{\sigma}{\sqrt{n}}$.
				
				Ici, l'écart-type de $\bar{D}$ est $\sqrt{0.002} \approx 0.0447$ (ou exactement $\frac{0.2}{\sqrt{20}}$).
				$$\boxed{\bar{D} \sim \mathcal{N}\left(10, \frac{0.2}{\sqrt{20}}\right)}$$
			}
			
			\item   \question{On définit :
				$$Q = \frac{1}{0.04}\sum_{i=1}^{20} (D_i - 10)^2$$
				Quelle loi suit $Q$ ? Déterminer la valeur $q$ telle que $P(Q > q) = 0.05$.}
			
			\indication{Remarquer que $0.04 = \sigma^2$ et utiliser la définition de la loi du $\chi^2$.}
			\reponse{
				On sait que $D_i \sim \mathcal{N}(10, 0.2)$.
				Posons $Z_i = \dfrac{D_i - 10}{0.2}$. Alors $Z_i \sim \mathcal{N}(0,1)$.
				
				On remarque que :
				$$(D_i - 10)^2 = 0.2^2 \times Z_i^2 = 0.04 \times Z_i^2 \implies \frac{(D_i - 10)^2}{0.04} = Z_i^2$$
				
				Donc :
				$$Q = \sum_{i=1}^{20} \frac{(D_i - 10)^2}{0.04} = \sum_{i=1}^{20} Z_i^2$$
				
				Par définition, la somme des carrés de $n$ variables normales centrées réduites indépendantes suit une loi du chi-deux à $n$ degrés de liberté. Ici $n=20$.
				$$\boxed{Q \sim \chi^2(20)}$$
				
				\textbf{Calcul de $q$ :}
				On cherche le quantile d'ordre $1 - 0.05 = 0.95$ de la loi $\chi^2(20)$.
				D'après la table de la loi du $\chi^2$ (ligne $\nu=20$, colonne $0.95$) :
				$$\boxed{q \approx 31.41}$$
			}
		\end{enumerate}
		
		\subsection*{Partie B - Contrôle qualité}
		
		\texte{
			Un boulon est dit \textbf{conforme} si son diamètre est compris entre 9.6 mm et 10.4 mm.
		}
		
		\begin{enumerate}
			\item   \question{Calculer $P(9.6 \leq D \leq 10.4)$ pour un boulon pris au hasard.
				
				Méthode : Standardiser avec $Z = \dfrac{D - 10}{0.2} \sim \mathcal{N}(0,1)$.}
			\reponse{
				$$P(9.6 \leq D \leq 10.4) = P\left(\frac{9.6-10}{0.2} \leq Z \leq \frac{10.4-10}{0.2}\right) = P(-2 \leq Z \leq 2)$$
				
				Par symétrie de la loi $\mathcal{N}(0,1)$ :
				$$P(-2 \leq Z \leq 2) = 2\Phi(2) - 1$$
				Avec $\Phi(2) \approx 0.9772$ (lu dans la table) :
				$$P(9.6 \leq D \leq 10.4) = 2(0.9772) - 1 = 1.9544 - 1 = \boxed{0.9544}$$
			}
			
			\item   \question{Le critère de conformité impose qu'au moins 95\% des boulons soient conformes. Ce critère est-il respecté ?}
			\reponse{
				La probabilité qu'un boulon soit conforme est de $95.44\%$.
				Comme $95.44\% > 95\%$, le critère est \textbf{respecté}.
			}
			
			\item   \question{On s'intéresse maintenant à la moyenne empirique $\bar{D}$ et à la variance empirique $S^2$ d'un échantillon de $n = 20$ boulons. On rappelle que :
				$$S^2 = \frac{1}{19}\sum_{i=1}^{20}(D_i - \bar{D})^2$$
				
				On définit la variable :
				$$T = \frac{\bar{D} - 10}{\sqrt{\frac{S^2}{20}}}$$
				
				Quelle loi suit $T$ ? Justifier.}
			\indication{Utiliser le théorème de Fisher et la structure de la loi de Student.}
			\reponse{
				Nous cherchons à identifier la structure de la variable $T$.
				
				1. D'après la question A.1, $\bar{D} \sim \mathcal{N}(10, \frac{0.2}{\sqrt{20}})$. La variable centrée réduite associée est :
				$$Z = \frac{\bar{D} - 10}{0.2/\sqrt{20}} \sim \mathcal{N}(0, 1)$$
				
				2. D'après le théorème de Fisher, la variable aléatoire liée à la variance empirique vérifie :
				$$V = \frac{(n-1)S^2}{\sigma^2} = \frac{19 S^2}{0.2^2} \sim \chi^2(19)$$
				De plus, $\bar{D}$ et $S^2$ sont indépendantes.
				
				3. Réécrivons $T$ en faisant apparaître $Z$ et $V$ :
				$$T = \frac{\bar{D} - 10}{\sqrt{S^2/20}} = \frac{\bar{D} - 10}{\frac{0.2}{\sqrt{20}}} \times \frac{\frac{0.2}{\sqrt{20}}}{\sqrt{\frac{S^2}{20}}}$$
				$$T = Z \times \frac{0.2}{\sqrt{S^2}} = Z \times \sqrt{\frac{0.04}{S^2}}$$
				Or, $V = \frac{19 S^2}{0.04} \implies \frac{0.04}{S^2} = \frac{19}{V}$. Donc :
				$$T = Z \times \sqrt{\frac{19}{V}} = \frac{Z}{\sqrt{V/19}}$$
				
				C'est exactement la définition d'une loi de Student à 19 degrés de liberté (ratio d'une normale standard par la racine d'un $\chi^2$ divisé par ses degrés de liberté).
				$$\boxed{T \sim \mathcal{S}t(19)}$$
			}
			
			\item   \question{Déterminer un intervalle de confiance à 95\% pour le diamètre moyen $\mu$ si on observe $\bar{d} = 10.05$ mm et $s^2 = 0.038$ sur un échantillon de 20 boulons.}
			\reponse{
				Puisque la variance réelle est supposée inconnue (on utilise $S^2$), on utilise la loi de Student.
				$$IC_{1-\alpha}(\mu) = \left[ \bar{d} - t_{1-\alpha/2}^{(n-1)} \frac{s}{\sqrt{n}} \ ; \ \bar{d} + t_{1-\alpha/2}^{(n-1)} \frac{s}{\sqrt{n}} \right]$$
				
				\textbf{Données :}
				\begin{itemize}
					\item $\bar{d} = 10.05$
					\item $n=20$
					\item $s = \sqrt{0.038} \approx 0.1949$
					\item Quantile de Student pour $1-\alpha = 0.95$ et $df=19$ : $t_{0.975}^{(19)} \approx 2.093$.
				\end{itemize}
				
				\textbf{Calcul de la marge d'erreur :}
				$$ME = 2.093 \times \frac{\sqrt{0.038}}{\sqrt{20}} = 2.093 \times \sqrt{\frac{0.038}{20}} = 2.093 \times \sqrt{0.0019}$$
				$$ME \approx 2.093 \times 0.04359 \approx 0.0912$$
				
				\textbf{Intervalle :}
				$$[10.05 - 0.0912 \ ; \ 10.05 + 0.0912] = [9.9588 \ ; \ 10.1412]$$
				
				$$\boxed{IC_{95\%} \approx [9.96 \ ; \ 10.14] \text{ mm}}$$
			}
		\end{enumerate}
	}
