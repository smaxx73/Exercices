\uuid{dmDL}
\chapitre{Continuité, limite et étude de fonctions réelles}
\sousChapitre{Fonctions équivalentes, fonctions négligeables}
\titre{Equivalents : vrai ou faux}
\theme{limite}
\auteur{}
\datecreate{2023-05-11}
\organisation{AMSCC}
\contenu{

\colonnes{\solution}{2}{1}
	\begin{enumerate}
		\item \question{$x+1 \underset{x\longrightarrow+\infty}{\sim} x$}
		\reponse{
			Vrai. En effet, par définition, $f(x)\sim g(x)$ lorsque
			\[
			\frac{f(x)}{g(x)}\longrightarrow 1.
			\]
			Ici,
			\[
			\frac{x+1}{x} = 1+\frac{1}{x}\longrightarrow 1 \quad \text{lorsque } x\longrightarrow+\infty.
			\]
		}
		
		\item \question{$x^2-x \underset{x\longrightarrow+\infty}{\sim} x$}
		\reponse{
			Faux. On a :
			\[
			\frac{x^2-x}{x} = x-1,
			\]
			et puisque $x-1\longrightarrow+\infty$ lorsque $x\longrightarrow+\infty$, le rapport ne tend pas vers 1.
		}
		
		\item \question{$\ln(x) \underset{x\longrightarrow+\infty}{\sim}\ln(10^6x)$}
		\reponse{
			Vrai. En utilisant la propriété logarithmique,
			\[
			\ln(10^6x)=\ln(10^6)+\ln(x),
			\]
			on obtient :
			\[
			\frac{\ln(10^6x)}{\ln(x)} = 1+\frac{\ln(10^6)}{\ln(x)}\longrightarrow 1 \quad \text{lorsque } x\longrightarrow+\infty.
			\]
		}
		
		\item \question{$x^2+2x+1 \underset{x\longrightarrow+\infty}{\sim} x^2+2x$}
		\reponse{
			Vrai. En divisant numérateur et dénominateur par $x^2$, on trouve :
			\[
			\frac{x^2+2x+1}{x^2+2x}=\frac{1+\frac{2}{x}+\frac{1}{x^2}}{1+\frac{2}{x}}\longrightarrow 1 \quad \text{lorsque } x\longrightarrow+\infty.
			\]
		}
		
		\item \question{$\sqrt{x+1} \underset{x\longrightarrow 0}{\sim} 1$}
		\reponse{
			Vrai. Par développement limité de $\sqrt{1+x}$ au voisinage de $0$, on a :
			\[
			\sqrt{1+x}=1+\frac{x}{2}-\frac{x^2}{8}+\cdots,
			\]
			ce qui implique que
			\[
			\frac{\sqrt{x+1}}{1}\longrightarrow 1 \quad \text{lorsque } x\longrightarrow 0.
			\]
		}
		
		\item \question{$ x \underset{x\longrightarrow 0}{\sim} 2x$}
		\reponse{
			Faux. En effet,
			\[
			\frac{x}{2x}=\frac{1}{2}\neq 1.
			\]
		}
		
		\item \question{$e^x \underset{x\longrightarrow+\infty}{\sim} e^{x+10^6}$}
		\reponse{
			Faux. On a :
			\[
			\frac{e^x}{e^{x+10^6}} = e^{-10^6}\neq 1,
			\]
			car $e^{-10^6}$ est une constante strictement inférieure à 1.
		}
		
		\item \question{$e^x \underset{x\longrightarrow 0}{\sim} e^{2x}$}
		\reponse{
			Vrai. En effet,
			\[
			\frac{e^x}{e^{2x}}=e^{-x}\longrightarrow 1 \quad \text{lorsque } x\longrightarrow 0.
			\]
		}
		
		\item \question{$\frac{6x^3+2x}{2x+1} \underset{x\longrightarrow+\infty}{\sim} 3x^2$}
		\reponse{
			Vrai. Pour $x\longrightarrow+\infty$, les termes dominants sont $6x^3$ au numérateur et $2x$ au dénominateur. Ainsi,
			\[
			\frac{6x^3+2x}{2x+1}\sim\frac{6x^3}{2x}=3x^2.
			\]
		}
		
		\item \question{$\frac{6x^3+2x}{2x+1} \underset{x\longrightarrow 0}{\sim} 2x$}
		\reponse{
			Vrai. Lorsque $x\longrightarrow 0$, on a :
			\[
			6x^3+2x\sim 2x \quad \text{et} \quad 2x+1\sim 1,
			\]
			donc :
			\[
			\frac{6x^3+2x}{2x+1}\sim 2x.
			\]
		}
		
	\end{enumerate}
\fincolonnes{\solution}{2}{1}
    \href{https://moodle.st-cyr.terre.defense.gouv.fr/moodle/mod/quiz/view.php?id=31595}{A faire sur Moodle}
}
