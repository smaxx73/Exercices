\uuid{3DyT}
\chapitre{Probabilité discrète}
\niveau{L2}
\module{Probabilité et statistique}
\sousChapitre{Variable aléatoire discrète}
\titre{Logistique d’un dépôt}
\theme{variables aléatoires, inégalité de probabilité}
\auteur{Maxime Nguyen}
\datecreate{2025-10-07}
\organisation{AMSCC}

\difficulte{2}

\contenu{
	
	\texte{ 
		Un dépôt de ravitaillement reçoit chaque jour un certain nombre de camions de livraison. 
		Ce nombre est modélisé par une variable aléatoire d'espérance $100$ et d'écart-type $8$. 
		On souhaite estimer le risque qu’un jour le dépôt reçoive trop peu de camions pour maintenir ses stocks. 
	}
	
	\question{ 
		Donner un majorant de la probabilité qu’un jour, il arrive au plus $60$ camions. 
	}
	
	\reponse{ 
		Soit $X$ le nombre de camions arrivant au dépôt dans une journée. 
		On sait que $\E(X)=100$ et $\sigma(X)=8$. \\[0.5em]
		Par l’inégalité de Bienaymé–Tchebychev, on a :
		\begin{align*}
			\prob(X\leq 60)
			&= \prob(X-100\leq -40) \\
			&\leq \prob(|X-100|\geq 40) \\
			&\leq \frac{\sigma^2(X)}{40^2}
			= \frac{8^2}{40^2}
			= \frac{1}{25}
			= 0.04.
		\end{align*}
		Il y a donc au plus $4\%$ de chances qu’un jour le dépôt reçoive moins de $60$ camions, 
		ce qui permet de quantifier un risque logistique faible mais non négligeable. 
	}
}
