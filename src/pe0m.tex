\uuid{pe0m}
\exo7id{1977}
\auteur{ridde}
\organisation{exo7}
\datecreate{1999-11-01}
\isIndication{false}
\isCorrection{false}
\chapitre{Géométrie affine dans le plan et dans l'espace}
\sousChapitre{Géométrie affine dans le plan et dans l'espace}

\contenu{
\texte{
Soit $E$ un espace affine et f une application affine de $E$ dans $E$.
}
\begin{enumerate}
    \item \question{Montrer que $f$ est une translation ssi $\vec {f} = id$.}
    \item \question{Montrer que si $\vec {f} = \lambda id$ où $\lambda \neq 1$ alors $f$ est
une homoth\'etie (on montrera que $f$ admet un point fixe).}
    \item \question{On note $\mathcal{T}$ l'ensemble des translations. Montrer que $\mathcal{T}$
est un sous-groupe du groupe affine.}
    \item \question{On note $\mathcal{H}$ l'ensemble des homoth\'eties bijectives.
Montrer que $\mathcal{T} \cup \mathcal{H}$ est un sous-groupe du groupe affine.}
\end{enumerate}
}
