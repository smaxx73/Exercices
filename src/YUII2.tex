\uuid{YUII2}
\titre{ Test d'une variance }
\theme{statistiques}
\auteur{}
\organisation{AMSCC}
\contenu{

\texte{ 	Un laboratoire vient d'acquérir une nouvelle balance et on souhaite comparer la régularité du travail de cette dernière pour de très petites pesées à la norme habituelle du descriptif pour laquelle la variance $\sigma_0^2$ est égale à $4 g^2$. On prélève alors un échantillon de 30 masses dont les valeurs (en grammes) sont données ci-après :
2.53 ; 1.51 ; 1.52 ; 1.44 ; 4.32 ; 2.36 ; 2.41 ; 2.06 ; 1.57 ; 1.68 ; 3.09 ; 0.54 ; 2.32 ; 0.19 ; 2.66 ; 2.20 ; 1.04 ; 1.02 ; 0.74 ; 1.01 ; 0.35 ; 2.42 ; 2.66 ; 1.11 ; 0.56 ; 1.75 ; 1.51 ; 3.80 ; 2.22 ; 2.88

On suppose que les masses sont distribuées selon une loi normale. 
 }

\question{ Au risque de $5\%$, peut-on affirmer que la variance de l'échantillon est conforme à la norme du descriptif ? }

%\texte{ Indication : ce test n'est pas détaillé dans le cours. Pour le mettre en place, on pourra définir une variable de décision en s'appuyant sur le fait que si $X_1,...,X_n$ est un $n$-échantillon dont la loi mère est une loi normale $\mathcal{N}(\mu,\sigma^2)$ et $\overline{X}$ la moyenne empirique de l'échantillon, alors la variable aléatoire 
%$$S =  \sum_{i=1}^{n} \left(\frac{X_i-\overline{X}}%{\sigma} \right)^2$$
%suit une loi $\chi^2(n-1)$. }
}