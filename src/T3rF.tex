\uuid{T3rF}
\exo7id{4621}
\auteur{quercia}
\organisation{exo7}
\datecreate{2010-03-14}
\isIndication{false}
\isCorrection{true}
\chapitre{Série entière}
\sousChapitre{Autre}

\contenu{
\texte{

}
\begin{enumerate}
    \item \question{Développer en série entière $f$ : $z \mapsto z(1-z)^{-2}$. Montrer que $f$
    est injective sur $\mathring D(0,1)$.}
\reponse{$\frac z{(1-z)^2} = \sum_{n=1}^\infty nz^n$ ($R=1$).

    Pour $z,t\in \mathring D(0,1)$ on a
    $\frac{z}{(1-z)^2} - \frac{t}{(1-t)^2} = \frac{(z-t)(1-zt)}{(1-z)^2(1-t)^2}$,
    quantité nulle si et seulement si $z=t$, d'où l'injectivité de $z \mapsto\frac z{(1-z)^2}$.}
    \item \question{Soit $f(z)=z+\sum_{n=2}^{+\infty} a_nz^n$ la somme d'une série entière de rayon de 
    convergence au moins $1$ à coefficients réels.
    On suppose $f$ injective sur $\mathring D(0,1)$
    et on veut prouver~: $\forall\ n\ge 1, |a_n|\le n$.
  \begin{enumerate}}
\reponse{\begin{enumerate}}
    \item \question{Montrer pour $|z|<1$ que $f(z)\in\R\Leftrightarrow z\in\R$ et en déduire~:
    $\Im(z)\ge 0 \Rightarrow  \Im(f(z))\ge 0$.}
\reponse{$f(z)\in\R \Leftrightarrow f(z) = \overline{f(z)} = f(\overline z) \Leftrightarrow z=\overline z \Leftrightarrow z\in\R$.

Par injectivité, on en déduit que $\Im(f(z))$ garde un signe constant sur chaque demi-disque
limité par $]-1,1[$, et comme $f(z) = z + o_{z\to0}(z)$, ce signe est celui de~$\Im z$.}
    \item \question{Pour $0<r<1$ calculer $ \int_{t=0}^{\pi }\Im(f(re^{it}))\sin nt\,d t$.
    En déduire  $|a_n|r^n\le n|a_1|r$ et conclure.}
\reponse{$ \int_{t=0}^{\pi }\Im(f(re^{it}))\sin nt\,d t = \frac{\pi a_nr^n}2$.

On a $|\sin(nt)|\le n\sin(t)$ pour $0\le t\le\pi$ par récurrence,
donc $\frac{\pi |a_n|r^n}2\le n \int_{t=0}^{\pi }\Im(f(re^{it}))\sin t\,d t = \frac{n\pi a_1r}2$.
On en déduit $|a_n|r^n\le n|a_1|r$ et on conclut $|a_n|\le n$ en faisant tendre $r$ vers~$1$.}
\end{enumerate}
}
