\uuid{ASRD}
\exo7id{987}
\auteur{cousquer}
\organisation{exo7}
\datecreate{2003-10-01}
\isIndication{false}
\isCorrection{true}
\chapitre{Espace vectoriel}
\sousChapitre{Base}

\contenu{
\texte{
Dans $\mathbb{R}^3$, les vecteurs suivants forment-ils une base~? Sinon
d\'ecrire le sous-espace qu'ils engendrent.
}
\begin{enumerate}
    \item \question{$v_1 =(1,1,1), v_2=(3,0,-1),v_3=(-1,1,-1).$}
\reponse{C'est une base.}
    \item \question{$v_1 =(1,2,3), v_2=(3,0,-1),v_3=(1,8,13).$}
\reponse{Ce n'est pas une base : $v_3=4v_1-v_2$. Donc l'espace $\mathrm{Vect} (v_1,v_2,v_3)=\mathrm{Vect}(v_1,v_2)$.}
    \item \question{$v_1 =(1,2,-3), v_2=(1,0,-1),v_3=(1,10,-11).$}
\reponse{Ce n'est pas une base : $v_3=5v_1-4v_2$. Donc l'espace $\mathrm{Vect} (v_1,v_2,v_3)=\mathrm{Vect}(v_1,v_2)$.}
\end{enumerate}
}
