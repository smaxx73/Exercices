\uuid{b8e4f9a3-2d51-4c3b-9e7f-2a6c8b4d9e53}
\titre{Analyse de données pharmaceutiques avec Excel}
\niveau{L2} 
\module{Probabilités et Statistiques} 
\chapitre{Échantillonnage et estimation}
\sousChapitre{Intervalles de confiance, utilisation d'un tableur}
\theme{Statistiques descriptives, loi normale, intervalle de confiance de Student}
\auteur{AMSCC}
\datecreate{2025-12-02}
\organisation{AMSCC}
\difficulte{2}
\contenu{
    \texte{ 
    Un laboratoire pharmaceutique teste le temps d'efficacité (en heures) d'un nouveau médicament. Les durées mesurées sur \textbf{25 patients} sont fournies ci-dessous :
    
    \begin{center}
    \begin{tabular}{ccccc}
    4.2 & 5.1 & 4.8 & 5.5 & 4.6 \\
    5.3 & 4.9 & 5.7 & 4.5 & 5.0 \\
    4.7 & 5.2 & 4.4 & 5.6 & 4.3 \\
    5.4 & 4.8 & 5.1 & 4.9 & 5.3 \\
    4.6 & 5.5 & 4.7 & 5.0 & 4.8
    \end{tabular}
    \end{center}
    }
    
    \subsection*{Partie A - Statistiques descriptives}
    
    \begin{enumerate}
        \item   \question{À l'aide du tableur, calculer :
                \begin{itemize}
                    \item La moyenne $\bar{x}$ ;
                    \item La variance empirique corrigée $s^2$ ;
                    \item L'écart-type $s$ ;
                \end{itemize}
                Rappel : $s^2 = \dfrac{1}{n-1}\displaystyle\sum_{i=1}^{n}(x_i - \bar{x})^2$}
                \indication{Saisir les données dans une colonne Excel (par exemple A1:A25) et utiliser les fonctions statistiques.}
                \reponse{
                \textbf{Saisie dans Excel :}
                \begin{itemize}
                    \item Cellule B1 : \texttt{=MOYENNE(A1:A25)}
                    \item Cellule B2 : \texttt{=VAR.S(A1:A25)}
                    \item Cellule B3 : \texttt{=ECARTYPE.STANDARD(A1:A25)}
                \end{itemize}
                
                \textbf{Résultats numériques :}
                \begin{itemize}
                    \item Moyenne : $\bar{x} = 4.932$ heures
                    \item Variance : $s^2 = 0.1836$ heures²
                    \item Écart-type : $s = 0.4285$ heures
                \end{itemize}
                
                \textbf{Interprétation :} Le temps moyen d'efficacité observé est d'environ 4.93 heures, avec une dispersion relativement faible (écart-type de 0.43 heures, soit environ 26 minutes).
                }
                
        \item   \question{Le laboratoire affirme que le temps moyen d'efficacité est de 5 heures.
                \begin{itemize}
                    \item Calculer l'écart absolu : $|\bar{x} - 5|$
                    \item Calculer l'écart relatif : $\dfrac{|\bar{x} - 5|}{5} \times 100\%$
                    \item Commenter : l'écart observé vous semble-t-il significatif ?
                \end{itemize}}
                \indication{Pour l'écart relatif, utiliser \texttt{=ABS(B1-5)/5*100} dans Excel.}
                \reponse{
                \textbf{Calculs :}
                \begin{itemize}
                    \item Écart absolu : $|\bar{x} - 5| = |4.932 - 5| = 0.068$ heures $\approx 4$ minutes
                    \item Écart relatif : $\dfrac{0.068}{5} \times 100\% = 1.36\%$
                \end{itemize}
                
                \textbf{Commentaire :} L'écart relatif est très faible (environ 1.4\%). À ce stade, sans test statistique formel, cet écart pourrait être dû aux fluctuations d'échantillonnage. L'intervalle de confiance (partie C) permettra de conclure de façon rigoureuse.
                }
    \end{enumerate}
    
    \subsection*{Partie B - Modélisation par une loi normale}
    
    \texte{
    On suppose que le temps d'efficacité $X$ suit une loi normale $\mathcal{N}(\mu, \sigma^2)$ où $\mu$ et $\sigma$ sont estimés par $\bar{x}$ et $s$.
    }
    
    \begin{enumerate}
        \item   \question{Avec le tableur, calculer avec une précision $10^{-6}$ :
                \begin{itemize}
                    \item $P(X \leq 4.5)$ ;
                    \item $P(4.8 \leq X \leq 5.2)$.
                \end{itemize}
                Interprétation : Quel pourcentage de patients ont une efficacité entre 4.8 et 5.2 heures ?}
                \indication{Pour $P(a \leq X \leq b)$, calculer $P(X \leq b) - P(X \leq a)$.}
                \reponse{
                \textbf{Formules Excel :}
                \begin{itemize}
                    \item Cellule B4 : \texttt{=LOI.NORMALE(4.5; B1; B3; VRAI)}
                    \item Cellule B5 : \texttt{=LOI.NORMALE(5.2; B1; B3; VRAI) - LOI.NORMALE(4.8; B1; B3; VRAI)}
                \end{itemize}
                
                \textbf{Résultats :}
                \begin{itemize}
                    \item $P(X \leq 4.5) \approx 0.1562$, soit environ 15.6\%
                    \item $P(4.8 \leq X \leq 5.2) \approx 0.4660$, soit environ 46.6\%
                \end{itemize}
                
                \textbf{Interprétation :} Environ $47\%$ des patients ont un temps d'efficacité entre 4.8 et 5.2 heures. La distribution est centrée autour de 4.93 heures.
                }
                
        \item   \question{Déterminer le temps $t_{90}$ tel que $90\%$ des patients ont une efficacité inférieure à $t_{90}$ heures.}
                \indication{Il s'agit du quantile d'ordre 0.9 de la loi normale estimée.}
                \reponse{
                \textbf{Formule Excel :}
                
                Cellule B6 : \texttt{=LOI.NORMALE.INVERSE(0.9; B1; B3)}
                
                \textbf{Résultat :}
                $$t_{90} \approx 5.481 \text{ heures}$$
                
                \textbf{Interprétation :} $90\%$ des patients ont un temps d'efficacité inférieur à 5.48 heures (soit environ 5h29min). Seulement $10\%$ des patients dépassent cette durée.
                }
    \end{enumerate}
    
    \subsection*{Partie C - Intervalle de confiance}
    
    \begin{enumerate}
        \item   \question{Déterminer une estimation du temps moyen $\mu$ à l'aide d'un intervalle de confiance à $95\%$.}
                \indication{On utilise la loi de Student car $\sigma$ est inconnu et estimé par $s$.}
                \reponse{
                \textbf{Formules Excel :}
                \begin{itemize}
                    \item Cellule B7 : \texttt{=LOI.STUDENT.INVERSE.N(0.975; 24)} $\rightarrow t_{0.025}^{(24)} \approx 2.064$
                    \item Cellule B8 : \texttt{=B7 * B3 / RACINE(25)} $\rightarrow ME \approx 0.177$
                    \item Borne inférieure B9 : \texttt{=B1 - B8} $\rightarrow 4.755$
                    \item Borne supérieure B10 : \texttt{=B1 + B8} $\rightarrow 5.109$
                \end{itemize}
                
                \textbf{Intervalle de confiance à 95\% :}
                $$IC_{95\%} = [4.755 \; ; \; 5.109] \text{ heures}$$
                
                \textbf{Interprétation :} Avec 95\% de confiance, le temps moyen d'efficacité du médicament dans la population est compris entre 4.76 et 5.11 heures.
                }
                
        \item   \question{Comment interpréter ce résultat quant à l'affirmation du laboratoire ?}
                \indication{Vérifier si $5 \in [4.755; 5.109]$.}
                \reponse{
                \textbf{Oui}, la valeur 5 heures appartient à l'intervalle $[4.755; 5.109]$.
                
                \textbf{Conclusion :} Au niveau de confiance de 95\%, les données observées sont \textbf{compatibles} avec l'affirmation du laboratoire selon laquelle le temps moyen d'efficacité est de 5 heures. On ne peut pas rejeter cette affirmation sur la base de cet échantillon.
                
                Remarque : bien que la moyenne observée soit légèrement inférieure à 5h, cet écart n'est pas statistiquement significatif au seuil de 5\%.
                }
                
        \item   \question{Si on voulait diviser par 2 la largeur de l'intervalle de confiance, quelle devrait être la nouvelle taille d'échantillon $n'$ ?
                
                Indication : La largeur est proportionnelle à $\dfrac{1}{\sqrt{n}}$, donc si $\dfrac{\text{largeur}_{n'}}{\text{largeur}_n} = \dfrac{1}{2}$, alors...}
                \indication{Écrire $\frac{1}{\sqrt{n'}} = \frac{1}{2} \times \frac{1}{\sqrt{n}}$ et résoudre.}
                \reponse{
                La largeur de l'intervalle est $L = 2 \times ME = 2 \times t_{0.025} \times \dfrac{s}{\sqrt{n}}$
                
                Elle est proportionnelle à $\dfrac{1}{\sqrt{n}}$. Pour diviser la largeur par 2 :
                $$\frac{L_{n'}}{L_n} = \frac{1}{2} \Rightarrow \frac{\frac{1}{\sqrt{n'}}}{\frac{1}{\sqrt{n}}} = \frac{1}{2} \Rightarrow \sqrt{\frac{n}{n'}} = \frac{1}{2}$$
                
                D'où :
                $$\frac{n}{n'} = \frac{1}{4} \Rightarrow n' = 4n = 4 \times 25 = \boxed{100 \text{ patients}}$$
                
                \textbf{Conclusion :} Pour diviser par 2 la largeur de l'intervalle de confiance, il faudrait multiplier par 4 la taille de l'échantillon, soit mesurer 100 patients au lieu de 25.
                }
    \end{enumerate}
}
