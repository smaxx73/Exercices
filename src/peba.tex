\uuid{peba}
\titre{}
\theme{AM}
\auteur{Q. Liard}
\organisation{AMSCC}
\contenu{
\texte{ On s'intéresse à 15 relevés donnant la teneur en sodium dans deux stations d'eau différentes. Les résultats sont consignés dans le tableau \ref{tab:polluant}.
\begin{table}[H]
    \centering
    \renewcommand{\arraystretch}{1.3}
    \begin{tabular}{|c|c c c c c c c c c c c c c c c|}
        \hline
        \textbf{Relevés} & \textbf{1} & \textbf{2} & \textbf{3} & \textbf{4} & \textbf{5} & \textbf{6} & \textbf{7} & \textbf{8} & \textbf{9} & \textbf{10} & \textbf{11} & \textbf{12} & \textbf{13} & \textbf{14} & \textbf{15} \\
        \hline
        \textbf{Station 1} & 30 & 45 & 98 & 85 & 170 & 210 & 50 & 78 & 123 & 190 & 205 & 215 & 95 & 110 & 180 \\
        \textbf{Station 2} & 140 & 180 & 35 & 110 & 95 & 130 & 145 & 175 & 60 & 115 & 125 & 200 & 185 & 90 & 160 \\
        \hline
    \end{tabular}
    \caption{Relevés du sodium en µg/L}
    \label{tab:polluant}
\end{table}
}
\begin{enumerate}
    \item Quelles sont les variables statistiques décrites dans la Table \ref{tab:polluant} ? Quelle est leur nature ?
    \item Si on appelle \( Z \) la variable statistique donnant la teneur en sodium sur les deux stations d'eau, quelle est l'étendue de cette variable ?
    \item Calculer la variance inter-groupe et intra-groupe de la teneur en sodium.
    \item Calculer la variance totale de \( Z \) de deux façons différentes.
    \item On rappelle que le rapport de corrélation est donné par :
  $$
        \eta^2 = \frac{\text{variance inter-groupes}}{\text{variance totale}}.
    $$
    Calculer \( \eta^2 \), puis commenter le résultat.
\end{enumerate}
}