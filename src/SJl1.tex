\uuid{SJl1}
\titre{Limite d'une suite et intégrale de Riemann}
\niveau{L2}
\module{Analyse}
\chapitre{Intégration}
\sousChapitre{Intégrale de Riemann}
\theme{suite, logarithme, intégration par parties}
\auteur{Erwan L'Haridon}
\datecreate{2026-02-10}
\organisation{AMSCC}
\difficulte{3}

\contenu{
    \texte{Pour \( n \in \mathbb{N}^* \), on pose: \( u_n = \prod_{k=1}^{n} \left(1 + \frac{k^2}{n^2}\right)^{\frac{1}{n}} \) et \( v_n = \ln u_n \).}

    \begin{enumerate}
        \item \question{Montrer que \( v_n = \frac{1}{n} \cdot \sum_{k=1}^{n} \ln \left(1 + \frac{k^2}{n^2}\right) \).}
        \item \question{Exprimer la limite de \( v_n \) (quand \( n \to +\infty \)) sous la forme d'une intégrale définie et calculer cette intégrale. On pourra utiliser une intégration par parties.}
        \item \question{En déduire \( \lim_{n \to +\infty} u_n \).}
    \end{enumerate}
}
