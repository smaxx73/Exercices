\uuid{wrPE}
\titre{Détection d'anomalies dans un système de sécurité réseau}

\niveau{L3} 				
\module{Statistiques} 	
\chapitre{Tests statistiques}   			
\sousChapitre{Test de conformité d'une moyenne}			

\theme{Test d'hypothèses, intervalle de confiance, risque de deuxième espèce}				
\auteur{Maxime Nguyen}
\datecreate{2026-01-07}
\organisation{AMSCC}			
\difficulte{3}			

\contenu{
	
	\texte{ 
		Une entreprise de cybersécurité surveille les temps de réponse (en millisecondes) d'un serveur web critique. En fonctionnement normal, le temps de réponse moyen est de $\mu_0 = 85$ ms avec un écart-type $\sigma_0 = 15$ ms.
		
		Le système de détection d'intrusions doit identifier rapidement toute anomalie significative, spécifiquement une **augmentation** du temps de réponse pouvant indiquer une attaque par déni de service (DDoS). Pour cela, l'équipe de sécurité collecte un échantillon de 200 requêtes consécutives.
		
		Les données collectées sont disponibles dans le fichier \texttt{temps\_reponse.csv} (une valeur par ligne).
		
		On suppose que les temps de réponse suivent une loi normale et on considère un seuil de risque $\alpha = 5\%$ pour tous les tests.
	}
	
	\begin{enumerate}
		\item   \question{Après avoir importé les données dans un tableur, calculer la moyenne empirique $\overline{x}$ et l'écart-type empirique $s$ de l'échantillon.}
		\indication{Utiliser les fonctions MOYENNE() et ECARTYPE.STANDARD() du tableur.}
		\reponse{
			On trouve : $\overline{x} = 87{,}90$ ms et $s = 13{,}78$ ms.
			
			La moyenne empirique est supérieure à la valeur théorique $\mu_0 = 85$ ms.
		}
		
		\item   \question{Déterminer un intervalle de confiance à 95\% pour la moyenne $\mu$ des temps de réponse. Ce résultat suggère-t-il une anomalie ?}
		\indication{Utiliser la loi de Student. Le quantile bilatéral $t_{0,975}$ (pour 199 ddl) vaut environ $1,972$.}
		\reponse{
			L'intervalle de confiance bilatéral à 95\% est :
			$$IC_{95\%}(\mu) = \left[\overline{x} - 1{,}972 \frac{s}{\sqrt{n}} ; \overline{x} + 1{,}972 \frac{s}{\sqrt{n}}\right]$$
			
			Application numérique ($SE = 13{,}78/\sqrt{200} \approx 0{,}974$) :
			$$IC_{95\%}(\mu) = \left[87{,}90 - 1{,}92 ; 87{,}90 + 1{,}92\right] = [85{,}98 ; 89{,}82] \text{ ms}$$
			
			Interprétation : La valeur nominale $\mu_0 = 85$ ms n'appartient pas à cet intervalle (elle est inférieure à la borne basse), ce qui suggère une anomalie.
		}
		
		\item   \question{On soupçonne une attaque DDoS. Formuler les hypothèses d'un test \textbf{unilatéral} pour confirmer une augmentation anormale du temps de réponse. Effectuer ce test au seuil de 5\%.}
		\indication{On teste $H_0 : \mu = 85$ contre $H_1 : \mu > 85$. La région critique est de la forme $T > t_{0,95}$. Pour 199 ddl, $t_{0,95} \approx 1{,}653$.}
		\reponse{
			\textbf{Hypothèses :}
			\begin{itemize}
				\item $H_0$ : $\mu = 85$ ms (fonctionnement normal)
				\item $H_1$ : $\mu > 85$ ms (augmentation du temps de réponse / attaque)
			\end{itemize}
			
			\textbf{Statistique de test :}
			$$t_{obs} = \frac{\overline{x} - \mu_0}{s/\sqrt{n}} = \frac{87{,}90 - 85}{0{,}974} \approx 2{,}97$$
			
			\textbf{Région de rejet :} 
			Puisque le test est unilatéral à droite, on place tout le risque $\alpha=5\%$ dans la queue supérieure. La valeur critique est le quantile $t_{0,95}$ de la loi Student(199).
			$$t_{crit} \approx 1{,}653$$
			On rejette $H_0$ si $t_{obs} > 1{,}653$.
			
			\textbf{Décision :} $2{,}97 > 1{,}653$, donc on rejette $H_0$. L'augmentation du temps de réponse est statistiquement significative.
		}
		
		\item   \question{En réalité, une attaque augmente le temps moyen à $\mu_1 = 90$ ms (avec le même écart-type $\sigma_0 = 15$ ms). Calculer la probabilité $\beta$ de ne pas détecter cette attaque avec ce test unilatéral.}
		\indication{On accepte $H_0$ si $T \leq 1{,}653$. Traduire cette condition sur $\overline{X}$ en utilisant $\sigma_0$, puis calculer la probabilité sous la loi $\mathcal{N}(90, \sigma_0^2/n)$.}
		\reponse{
			\textbf{Borne d'acceptation :}
			On conserve $H_0$ si $t_{obs} \leq 1{,}653$, ce qui correspond à :
			$$\frac{\overline{X} - 85}{15/\sqrt{200}} \leq 1{,}653 \iff \overline{X} \leq 85 + 1{,}653 \times 1{,}061 \iff \overline{X} \leq 86{,}75$$
			
			\textbf{Calcul du risque $\beta$ :}
			C'est la probabilité d'accepter $H_0$ alors que $\mu = 90$.
			Sous $H_1$, $\overline{X} \sim \mathcal{N}(90 ; 1{,}061^2)$.
			$$\beta = \mathbb{P}(\overline{X} \leq 86{,}75) = \mathbb{P}\left(Z \leq \frac{86{,}75 - 90}{1{,}061}\right)$$
			$$\beta = \mathbb{P}(Z \leq -3{,}06) = \Phi(-3{,}06) \approx 0{,}0011$$
			
			\textbf{Commentaire :} Le risque de manquer cette attaque est extrêmement faible ($0{,}11\%$). Le test unilatéral est plus puissant que le test bilatéral (où $\beta$ était d'environ $0{,}31\%$) car il concentre la zone de rejet du côté de l'anomalie attendue.
		}
		
	\end{enumerate}
	
}