\titre{}

\texte{ L'équipe médicale d'une entreprise fait ses
 statistiques sur le taux de cholestérol de ses employés; les
observations sur 100 employés tirés au sort sont les suivantes.

\begin{center}
	\begin{tabular}{c|c}
	taux de cholestérol en cg:(centre classe) & effectif d'employés: \\
	\hline $120$ & $9$ \\ 
	$160$ & $22$ \\ 
	$200$ & $25$ \\ 
	$240$ & $21$ \\ 
	$280$ & $16$ \\ 
	$320$ & $7$
\end{tabular}
\end{center} }

\begin{enumerate}
	\item \question{ Calculer la moyenne $m_{e}$ et la variance $\sigma_{e}^2$ sur l'échantillon. }
	\item \question{ Estimer sans biais la moyenne et l'écart-type pour le taux de cholestérol dans toute l'entreprise. }
	\item \question{ Déterminer un intervalle de confiance  permettant d'estimer la moyenne du taux de cholesterol de tous les employés de cette entreprise avec une confiance de $90\%$.  }
\end{enumerate}




\reponse{
\begin{enumerate}
	\item On obtient, sur l'échantillon, la moyenne $m_{e}=214$, l'écart-type $\sigma _{e}=55.77$.
	\item La moyenne sur l'entreprise est estimée par $m_{e}$.
	L'écart-type est estimé par: $\widehat{\sigma _{e}}=\sqrt{\frac{100}{99}}55.77\simeq 56.05$.
	\item On en déduit, au seuil 95\%, un intervalle de confiance pour la
	moyenne :
	$[m_{e} - y_{\alpha }\frac{\widehat{\sigma _{e}}}{\sqrt{n}};
	m_{e} + y_{\alpha }\frac{\widehat{\sigma _{e}}}{\sqrt{n}}]=[203.01;224.99]$.
	Ainsi le taux moyen de cholestérol est, à un seuil de confiance $95$\%, 
	située entre $203$ et $225$ cg.
\end{enumerate}
}