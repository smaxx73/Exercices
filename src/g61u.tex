\uuid{g61u}
\exo7id{3165}
\auteur{quercia}
\organisation{exo7}
\datecreate{2010-03-08}
\isIndication{false}
\isCorrection{false}
\chapitre{Polynôme, fraction rationnelle}
\sousChapitre{Autre}

\contenu{
\texte{
Soit $n\in\N^*$. Pour $k\in{[[0,n]]}$ on pose $P_k = X^k(1-X)^{n-k}$.
D{\'e}montrer que ${\cal B} = (P_0,\dots,P_n)$ est une base de~$\R_n[X]$.
Calculer les composantes dans $\cal B$ de
$\frac{d^n}{d x^n} \bigl(X^n(1-X)^n\bigr)$. En d{\'e}duire la valeur de
$\sum_{k=0}^n (C_n^k)^2$.
}
}
