\uuid{si1j}
\exo7id{3979}
\auteur{quercia}
\organisation{exo7}
\datecreate{2010-03-11}
\isIndication{false}
\isCorrection{true}
\chapitre{Dérivabilité des fonctions réelles}
\sousChapitre{Autre}

\contenu{
\texte{
On note~$E$ l'ensemble des fonctions~$f$ de classe~$\mathcal{C}^1$ bijectives
de~$]0,+\infty[$ sur~$]0,+\infty[$ telles que $f' = f^{-1}$.
}
\begin{enumerate}
    \item \question{Trouver un élément de~$E$ du type $x \mapsto c x^m$, où $c$ et $m$ sont réels.}
    \item \question{Quelle est la limite en~$0$ de~$f$~?}
    \item \question{Montrer que~$f$ est un $\mathcal{C}^\infty$ difféomorphisme de~$]0,+\infty[$ sur~$]0,+\infty[$.}
    \item \question{Montrer que~$f$ admet un unique point fixe.}
    \item \question{Soit~$g$ un deuxième élément de~$E$. Montrer que $g$ admet le même
    point fixe que~$f$.}
\reponse{
$m=\frac{1+\sqrt5}2$, $c=m^{-1/m}$.
$f$ et $f'$ ont des limites nulles en $0^+$ et infinies en~$+\infty$
    donc $f(x) = o_{x\to0^+}(x)$ et $x = o_{x\to+\infty}(f(x))$,
    ce qui implique que $f(x)-x$ s'annule sur~$]0,+\infty[$.
    S'il y a deux points fixes, $a<b$, alors par le thm. des accroissements
    finis l'équation $f'(x) = 1$ admet une solution dans~$]0,a[$ et une dans~$]a,b[$,
    en contradiction avec la bijectivité de~$f' = f^{-1}$.
On note~$a$ le point fixe de~$f$, $b$ celui de~$g$ et on suppose
    $a\ne b$, par exemple $a<b$. On a $g(x) < x$ pour $x\in{]0,b[}$ donc $g(a) < a=f(a)$.
    Par conséquent $g(x) < x \le f(x)$ si $x\in{[a,b[}$~; soit $]c,b[$ le plus grand
    intervalle sur lequel $g(x) < f(x)$.
    On a $0\le c < a$, $g(c^+) = f(c^+) \le c$ et $f$, $g$ sont strictement
    croissantes, donc $g^{-1}(x) > f^{-1}(x)$ pour $x\in{]c,b[}$. Ainsi
    $g-f$ est strictement croissante sur~$]c,b[$ a une limite nulle en~$c^+$
    et est négative en~$b$, c'est absurde.
    
    Remarque~: le point fixe est le nombre d'or~$m$. De plus, si $f$ et $g$
    sont deux éléments de~$E$ distincts alors $f-g$ n'est de signe constant
    sur aucun voisinage de~$m^-$ (même démonstration).
}
\end{enumerate}
}
