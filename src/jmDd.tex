\titre{ Durée de vie }
\theme{probabilités}
\auteur{}
\organisation{AMSCC}
\contenu{

\texte{ Une machine est composée de trois alternateurs indépendants. La durée de vie de $T$ de chaque alternateur suit une loi exponentielle de paramètre $\lambda$. La machine fonctionne si et seulement si au moins deux des alternateurs fonctionnent.  On appelle $X$ la variable aléatoire mesurant le temps de fonctionnement de la machine. }

\begin{enumerate}

\item \question{ Déterminer la loi de $X$ et calculer son espérance. }

\reponse{  }

\item \question{ Soient les réels $t>0$, $h>0$. Sachant que la machine a déjà fonctionné pendant un temps $t$, quelle est la probabilité qu'elle fonctionne encore pendant un temps $h$ ? Déterminer la limite de cette probabilité, à $h$ fixé, lorsque $t \to +\infty$. }

\reponse{  }

\end{enumerate}}
