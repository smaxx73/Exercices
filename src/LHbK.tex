\titre{\'Etude d'une série numérique}
\theme{Série}
\auteur{}
\organisation{AMSCC}

\contenu{ Soit $\alpha$ un nombre r\'eel et $(u_n)$ la suite d\'efinie pour tout entier naturel $n\geq 1$ par:
$$u_n=\left[\left(1+\frac{1}{n^{\alpha}}\right)\times\left(1+\frac{2}{n^{\alpha}}\right)\times \cdots \times \left(1+\frac{n}{n^{\alpha}}\right) \right]-1.$$

Pour tout entier naturel $n \geq 1$, on pose : 
$$v_n =  \frac{n(n+1)}{2n^{\alpha}} \quad \text{et} \quad w_n = \left(1+\frac{n}{n^{\alpha}}\right)^{n}-1 \,. $$
\begin{enumerate}
	\item \question{ Déterminer un équivalent simple de $v_n$ quand $n$ tend vers l'infini. }
	\item \question{ En déduire que l'ensemble des valeurs de $\alpha$ pour lesquelles la série $\displaystyle\sum_{n\geq 1}v_n$ diverge.  }
	\item \question{ Donner un développement limité de $\ln\left(1+\frac{1}{n^{\alpha-1}}\right)$ à l'ordre $1$ quand $n \to +\infty$. En déduire que : $$w_n = \frac{1}{n^{\alpha-2}} + \underset{n\to +\infty}o\left(\frac{1}{n^{\alpha-2}}\right) \, .$$ }
	\item \question{Montrer l'inégalité suivante, pour tout $n \geq 1:$
$$0\le u_n \leq \left(1+\frac{n}{n^{\alpha}}\right)^{n}-1.$$
En déduire que la série $\displaystyle\sum_{n\geq 1}u_n$ converge si $\alpha>3$.
}
\item \question{Montrer que pour tout entier naturel $n \geq 1,$ la suite $(u_n)$ v\'erifie l'inégalité suivante:
	$$u_n \geq \frac{n(n+1)}{2n^{\alpha}} \,.$$
}
\item \question{ Déduire des questions précédentes l'ensemble des valeurs de $\alpha$ pour lesquelles  la série $\displaystyle\sum_{n\geq 1}u_n$ converge. }
\end{enumerate}

}
