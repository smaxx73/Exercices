\uuid{1j3q}
\titre{Fiabilité des systèmes en parallèle et loi exponentielle}
\niveau{L2}
\module{Probabilité et statistique}
\chapitre{Probabilité continue}
\sousChapitre{Loi exponentielle}
\theme{Loi exponentielle, durée de vie, probabilité conditionnelle, minimum de variables aléatoires}
\auteur{Erwan Hillion}
\datecreate{2025-11-12}
\organisation{AMSCC}
\difficulte{3}

\contenu{
	\texte{Un poste de commandement utilise deux systèmes de communication indépendants et identiques, chacun ayant une durée de vie modélisée par une variable aléatoire exponentielle de paramètre \(\lambda = 0.05\). Cela signifie que pour \(i=1\) ou \(2\), la durée de vie \(X_i\) (en heures) du système \(i\) suit une loi exponentielle \(\mathcal{E}(0.05)\).}
	
	\texte{On rappelle que la loi exponentielle \(\mathcal{E}(\lambda)\) a pour densité :
		$$f: x \mapsto \begin{cases}
			\lambda e^{-\lambda x} & \textrm{si } x > 0, \\
			0 & \textrm{sinon.}
		\end{cases}$$
		
		Pour améliorer la fiabilité, le poste de commandement utilise les deux systèmes en parallèle : dès qu'un système tombe en panne, une alerte est déclenchée et une équipe technique intervient. On s'intéresse au temps \(T\) avant la première défaillance, défini par :
		\[ T = \min(X_1, X_2). \]}
	
	\begin{enumerate}
		\item \question{Montrer que, pour \(i = 1\) ou \(2\) et \(x > 0\), on a \(\mathbb{P}(X_i > x) = \exp(-\lambda x)\).}
		
		\reponse{
			La probabilité \(\mathbb{P}(X_i > x)\) est le complément de la fonction de répartition \(F_{X_i}(x) = \mathbb{P}(X_i \le x)\).
			Commençons par calculer la fonction de répartition pour \(x > 0\) :
			\[ F_{X_i}(x) = \int_{-\infty}^{x} f(t) \,dt = \int_{0}^{x} \lambda e^{-\lambda t} \,dt \]
			Car la densité est nulle pour \(t \le 0\).
			\[ F_{X_i}(x) = \left[ -e^{-\lambda t} \right]_{0}^{x} = \left( -e^{-\lambda x} \right) - \left( -e^{-\lambda \cdot 0} \right) = -e^{-\lambda x} - (-1) = 1 - e^{-\lambda x} \]
			La probabilité recherchée, aussi appelée fonction de survie, est donc :
			\[ \mathbb{P}(X_i > x) = 1 - F_{X_i}(x) = 1 - (1 - e^{-\lambda x}) = e^{-\lambda x} \]
			Ce qui correspond à la formule \(\exp(-\lambda x)\).
		}
		
		\item \question{Montrer que, pour \(x > 0\), on a : \(\mathbb{P}(T > x) = \mathbb{P}(X_1 > x \cap X_2 > x)\) et en déduire que \(\mathbb{P}(T > x) = \exp(-2 \lambda x)\).}
		
		\reponse{
			La variable aléatoire \(T\) est définie comme le minimum de \(X_1\) et \(X_2\), c'est-à-dire \(T = \min(X_1, X_2)\).
			
			\begin{itemize}
				\item \textbf{Démonstration de l'égalité :}
				L'événement \(\{T > x\}\) signifie que le minimum des deux durées de vie est strictement supérieur à \(x\).
				Pour que le minimum de deux nombres soit supérieur à \(x\), il est nécessaire et suffisant que \textit{chacun} de ces nombres soit supérieur à \(x\).
				Ainsi, l'événement \(\{T > x\}\) est équivalent à l'événement \(\{X_1 > x \text{ et } X_2 > x\}\).
				En termes de probabilités, cela se traduit par l'égalité :
				\[ \mathbb{P}(T > x) = \mathbb{P}(X_1 > x \cap X_2 > x) \]
				
				\item \textbf{Déduction de la probabilité :}
				L'énoncé précise que les deux systèmes de communication sont \textbf{indépendants}. Par conséquent, les variables aléatoires \(X_1\) et \(X_2\) sont indépendantes.
				La probabilité de l'intersection de deux événements indépendants est le produit de leurs probabilités :
				\[ \mathbb{P}(X_1 > x \cap X_2 > x) = \mathbb{P}(X_1 > x) \times \mathbb{P}(X_2 > x) \]
				En utilisant le résultat de la question 1, on a :
				\[ \mathbb{P}(T > x) = e^{-\lambda x} \times e^{-\lambda x} = e^{-\lambda x - \lambda x} = e^{-2 \lambda x} \]
				Donc, \(\mathbb{P}(T > x) = \exp(-2 \lambda x)\).
			\end{itemize}
		}
		
		\item \question{En déduire que la variable aléatoire \(T\) suit une loi exponentielle dont on précisera le paramètre.}
		
		\reponse{
			Une variable aléatoire \(Y\) suit une loi exponentielle de paramètre \(\mu\) si sa fonction de survie est de la forme \(\mathbb{P}(Y > x) = e^{-\mu x}\) pour \(x > 0\).
			
			D'après la question précédente, nous avons montré que la fonction de survie de la variable aléatoire \(T\) est :
			\[ \mathbb{P}(T > x) = e^{-2 \lambda x} \]
			Cette expression correspond exactement à la forme de la fonction de survie d'une loi exponentielle, en posant le paramètre \(\mu = 2\lambda\).
			
			Par conséquent, la variable aléatoire \(T\) suit une loi exponentielle de paramètre \(2\lambda\).
			
			L'énoncé donne \(\lambda = 0.05\), le paramètre de la loi de \(T\) est donc :
			\[ 2 \lambda = 2 \times 0.05 = 0.1 \]
			Conclusion : \(T\) suit la loi exponentielle \(\mathcal{E}(0.1)\).
		}
	\end{enumerate}
}