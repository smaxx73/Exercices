\uuid{td27}
\titre{Défaillances et loi de Poisson}
\theme{}
\auteur{}
\datecreate{2025-03-20}
\organisation{}

\contenu{

\texte{
Un statisticien observe chaque jour, pendant $n$ jours, le nombre d’ampoules défaillantes à la sortie d’une chaîne de fabrication. Il veut estimer la probabilité de n’avoir aucune ampoule défaillante. Il note $X_i$ le nombre d’ampoules défaillantes à la sortie de la chaîne le $i$-ème jour.
}

\begin{enumerate}
    \item \question{Dans un premier temps, il compte le nombre $N_n$ de jours où aucune défaillance n’a été observée. Il propose d’estimer $P(X_1 = 0)$ par $p_{b1} = \frac{N_n}{n}$. \\
    (a) Montrer, en supposant juste les $X_i$ i.i.d., que $p_{b1}$ est un estimateur sans biais de $P(X_1 = 0)$. Calculer son risque quadratique et donner sa loi limite. \\
    (b) Donner des intervalles de confiance pour $P(X_1 = 0)$ de niveau $(1-\alpha)$.}
    \indication{}
    \reponse{}
    \item \question{Le statisticien suppose de plus que $X_1 \sim P(\lambda)$. Il propose d’estimer $P(X_1 = 0) = \exp(-\lambda)$ par $p_{b2} = e^{-\bar{X}_n}$.\\
     (a) Préliminaire : soient $Y_1 \sim P(\lambda_1)$ et $Y_2 \sim P(\lambda_2)$, deux v.a. indépendantes. Rappeler la loi de $Y_1 + Y_2$. \\
     (b) Expliquer sa démarche. Montrer que $p_{b2}$ est biaisé. Calculer sa variance et son biais. Déterminer des équivalents asymptotiques des quantités précédentes. \\
      (c) Lequel de $p_{b1}$ ou $p_{b2}$ choisiriez-vous pour estimer $e^{-\lambda}$ ?}
    \indication{}
    \reponse{}
\end{enumerate}

}