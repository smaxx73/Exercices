\uuid{a1ba}
\titre{Test non paramétrique}
\theme{Statistique}
\auteur{Erwan L'Haridon}
\organisation{AMSCC}
\contenu{




\texte{

Un professeur particulièrement perfide utilise un générateur de nombres aléatoires pour désigner qui de ses élèves ira au tableau. Il modélise sa classe de 10 élèves par les nombres entiers et il prétend que son générateur suit une loi uniforme sur l’ensemble.

Un élève prend le temps de tester 200 fois ce générateur de nombres et il obtient les occurrences suivantes :

\begin{center}
\begin{tabular}{|c|c|c|c|c|c|c|c|c|c|c|}
    \hline
    Numéro & 1 & 2 & 3 & 4 & 5 & 6 & 7 & 8 & 9 & 10 \\
    \hline
    Nombre d’apparition & 19 & 23 & 20 & 11 & 15 & 16 & 24 & 23 & 24 & 25 \\
    \hline
\end{tabular}
\end{center}

Le professeur prétend que la distribution de ses tirages est uniforme. L’élève affirme au contraire, que certains élèves sont privilégiés de manière significative par ce processus. Avec un seuil de confiance de 95 \%, qui a raison ?








}
}