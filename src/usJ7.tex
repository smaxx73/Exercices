\uuid{usJ7}
\exo7id{4750}
\auteur{quercia}
\organisation{exo7}
\datecreate{2010-03-16}
\isIndication{false}
\isCorrection{false}
\chapitre{Topologie}
\sousChapitre{Topologie des espaces vectoriels normés}

\contenu{
\texte{
Soit $(A_n)$ une suite de matrices de $\mathcal{M}_p(\R)$ v{\'e}rifiant les propri{\'e}t{\'e}s
suivantes :
$$\begin{cases} 1:\ A_n\xrightarrow[n\to\infty]{} A \in \mathcal{M}_p(\R) \cr
          2:\ \text{pour tout $n$, $A_n$ est inversible} \cr
          3:\ A_n^{-1}\xrightarrow[n\to\infty]{} B \in \mathcal{M}_p(\R). \cr \end{cases}$$
}
\begin{enumerate}
    \item \question{Montrer que $A$ est inversible et $A^{-1} = B$.}
    \item \question{Peut-on retirer la propri{\'e}t{\'e} 3 ?}
\end{enumerate}
}
