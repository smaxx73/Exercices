\uuid{5yqq}
\niveau{L2}
\module{Probabilité et statistique}
\chapitre{Probabilité continue}
\sousChapitre{Espérance et Fonction de répartition}

\titre{Durée de vie d'un émetteur récepteur}
\theme{Loi Normale}
\auteur{L'Haridon E.}
\datecreate{2025-10-07}
\organisation{AMSCC}
\difficulte{3}
\contenu{

\texte{
Une unité de communication militaire est composée de trois émetteurs-récepteurs indépendants.
La durée de vie $T_i$ de chaque émetteur-récepteur $i$ suit une loi exponentielle de paramètre $\lambda$. Le système de communication fonctionne si et seulement si au moins deux des émetteurs-récepteurs sont opérationnels. On appelle X la variable aléatoire mesurant le temps de fonctionnement de la machine.
}

\begin{enumerate}
	\item   \question{Montrer l'égalité entre événements:
		\[
		\left[X\geq t\right] = \left[T_1\geq t,T_2\geq t,T_3\geq t\right] \cup \left[T_1\leq t,T_2\geq t,T_3\geq t\right] \cup 
		\]
		\[
		\left[T_1\geq t,T_2\leq t,T_3\geq t\right] \cup \left[T_1\geq t,T_2\geq t,T_3\leq t\right]
		\]
	}
	\reponse{L’événement $\left[X\geq t\right]$ signifie que le système de communication fonctionne encore au temps t, donc d'après le modèle que deux ou trois des émetteurs-récepteurs sont encore opérationnels au temps t. Le premier événement de l'union traduit le fait que les trois émetteurs-récepteurs sont encore opérationnels, les trois autres événements signifient que les émetteurs-récepteurs 2 et 3, resp. 1 et 3, resp. 1 et 2 sont encore opérationnels.}
	
	\item   \question{En déduire une expression de la fonction de répartition $F_X$ faisant intervenir les fonctions de répartition $F_{T_i}$.}
	\reponse{On remarque que les quatre événements de l’union décrite à la question précédente sont deux à deux disjoints. En utilisant l'indépendance des variables \( T_i \), on obtient :
		
		\begin{align*}
			P[X \geq t] &= P[T_1 \geq t, T_2 \geq t, T_3 \geq t] + P[T_1 \leq t, T_2 \geq t, T_3 \geq t] \\
			&\quad + P[T_1 \geq t, T_2 \leq t, T_3 \geq t] + P[T_1 \geq t, T_2 \geq t, T_3 \leq t] \\
			&= P(T_1 \geq t)P(T_2 \geq t)P(T_3 \geq t) + P(T_1 \leq t)P(T_2 \geq t)P(T_3 \geq t) + \dots \\
			&= (1 - F_T(t))^3 + 3 F_T(t) (1 - F_T(t))^2.
		\end{align*}
		
		Comme \( X \) suit une loi à densité, sa fonction de répartition est :
		\[
		F_X(t) = 1 - P(X \geq t) = 1 - \left[ (1 - F_T(t))^3 + 3F_T(t)(1 - F_T(t))^2 \right].
		\]
	}
	
	\item   \question{Montrer que $X$ suit la loi de densité $f_X$ définie pour tout réel strictement positif par:
		\[
		f_X(t)=6\lambda e^{-2\lambda t}-6\lambda e^{-3\lambda t}
		\]}
	\reponse{Soit \( t \geq 0 \). La fonction de répartition d'une loi exponentielle est $F_T(t) = 1-e^{-\lambda t}$. En substituant dans l'expression de $F_X(t)$ :
		\begin{align*}
			F_X(t) &= 1 - \left[ (e^{-\lambda t})^3 + 3(1-e^{-\lambda t})(e^{-\lambda t})^2 \right] \\
			&= 1 - \left[ e^{-3\lambda t} + 3(e^{-2\lambda t} - e^{-3\lambda t}) \right] \\
			&= 1 - \left( 3e^{-2\lambda t} - 2e^{-3\lambda t} \right) \\
			&= 1 - 3e^{-2\lambda t} + 2e^{-3\lambda t}.
		\end{align*}
		Si \( t < 0 \), alors \( F_X(t) = 0 \). La fonction de répartition \( F_X \) est continue et dérivable sauf potentiellement en 0. La densité \( f_X \) est la dérivée de \( F_X \) :
		\[
		f_X(t) = F_X'(t) = 
		\begin{cases}
			6\lambda e^{-2\lambda t} - 6\lambda e^{-3\lambda t} & \text{si } t > 0, \\
			0 & \text{si } t < 0.
		\end{cases}
		\]}
	
	\item   \question{Calculer l'espérance de $X$.}
	\reponse{L’espérance \( \mathbb{E}(X) \) se calcule par le théorème de transfert :
		\[
		\mathbb{E}(X) = \int_{0}^{+\infty} t f_X(t)\, dt 
		= \int_{0}^{+\infty} t \left( 6\lambda e^{-2\lambda t} - 6\lambda e^{-3\lambda t} \right) dt.
		\]
		On sépare l'intégrale en deux parties :
		\[
		\mathbb{E}(X) = 6\lambda \int_{0}^{+\infty} t e^{-2\lambda t}\, dt - 6\lambda \int_{0}^{+\infty} t e^{-3\lambda t}\, dt.
		\]
		En reconnaissant l'espérance d'une loi exponentielle (ou par intégration par parties), on sait que $\int_0^\infty t \alpha e^{-\alpha t} dt = 1/\alpha$.
		\[
		\int_{0}^{+\infty} t e^{-2\lambda t}\, dt = \frac{1}{(2\lambda)^2} = \frac{1}{4\lambda^2} \quad \text{et} \quad
		\int_{0}^{+\infty} t e^{-3\lambda t}\, dt = \frac{1}{(3\lambda)^2} = \frac{1}{9\lambda^2}.
		\]
		Ainsi :
		\[
		\mathbb{E}(X) = 6\lambda \left( \frac{1}{4\lambda^2} - \frac{1}{9\lambda^2} \right)
		= 6\lambda \left( \frac{9 - 4}{36\lambda^2} \right)
		= 6\lambda \cdot \frac{5}{36\lambda^2}
		= \frac{5}{6\lambda}.
		\]
	}
	
	\item   \question{Soient les réels $t>0,h>0$. Calculer la probabilité conditionnelle $P\left(\left[X\geq t+h \mid X\geq t \right]\right)$ puis calculer la limite de la probabilité, à $h$ fixé, lorsque $t \to +\infty$.}
	\reponse{
		Par définition, sachant que l'événement $\{X \geq t+h\}$ est inclus dans $\{X \geq t\}$ :
		\[
		P(X \geq t + h \mid X \geq t)
		= \frac{P\left( [X \geq t + h] \cap [X \geq t] \right)}{P(X \geq t)}
		= \frac{P(X \geq t + h)}{P(X \geq t)}.
		\]
		En utilisant $P(X \geq t) = 1 - F_X(t) = 3 e^{-2\lambda t} - 2 e^{-3\lambda t}$, on obtient :
		\[
		P(X \geq t + h \mid X \geq t)
		= \frac{3 e^{-2\lambda (t + h)} - 2 e^{-3\lambda (t + h)}}{3 e^{-2\lambda t} - 2 e^{-3\lambda t}}
		= \frac{e^{-2\lambda t} \left( 3 e^{-2\lambda h} - 2 e^{-\lambda t} e^{-3\lambda h} \right)}{e^{-2\lambda t} \left( 3 - 2 e^{-\lambda t} \right)}
		= \frac{3 e^{-2\lambda h} - 2 e^{-\lambda(t + 3h)}}{3 - 2 e^{-\lambda t}}.
		\]
		Lorsque $t \to +\infty$, le terme $e^{-\lambda t}$ tend vers 0. Par conséquent :
		\[
		\lim_{t \to +\infty} P(X \geq t + h \mid X \geq t) = \frac{3e^{-2\lambda h} - 0}{3 - 0} = e^{-2\lambda h}.
		\]}
\end{enumerate}

}


