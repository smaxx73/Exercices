\uuid{n5y6}
\chapitre{Probabilité continue}
\sousChapitre{Espérance et Fonction de répartition}

\titre{Durée de vie d'un émetteur récepteur}
\theme{Loi Normale}
\auteur{L'Haridon E.}
\datecreate{2025-10-07}
\organisation{AMSCC}
\difficulte{3}
\contenu{

\texte{
Une unité de communication militaire est composée de trois émetteurs-récepteurs indépendants.
La durée de vie $T_i$ de chaque émetteur-récepteur $i$ suit une loi exponentielle de paramètre $\lambda$. Le système de communication fonctionne si et seulement si au moins deux des émetteurs-récepteurs sont opérationnels. On appelle X la variable aléatoire mesurant le temps de fonctionnement de la machine.



}
\begin{enumerate}
\item   \question{Montrer l'égalité entre événements:
$$\big[X\geq t\big]=\big[T_1\geq t,T_2\geq t,T_3\geq t\big] \cup \big[T_1\leq t,T_2\geq t,T_3\geq t\big] \cup $$ $$\big[T_1\geq t,T_2\leq t,T_3\geq t\big] \cup \big[T_1\geq t,T_2\geq t,T_3\leq t\big] $$
}
\reponse{L’événement $X\geq t$ signifie que le système de communication fonctionne encore au temps t, donc d'après le modèle que deux ou trois des émetteurs-récepteurs sont encore opérationnels
au temps t. Le premier événement de l'union traduit le fait que les trois émetteurs-récepteurs
sont encore opérationnels, les trois autres événements signifient que les émetteurs-récepteurs 2
et 3, resp. 1 et 3, resp. 1 et 2 sont encore opérationnels.}
\item   \question{En déduire une expression de la fonction de répartition $F_X$ faisant intervenir les fonctions de répartition $F_{T_i}}$. \reponse{On remarque que les quatre événements de l’union décrite à la question précédente sont deux à deux disjoints, donc :

$$
P[X \geq t] = P[T_1 \geq t,\, T_2 \geq t,\, T_3 \geq t] 
+ P[T_1 \leq t,\, T_2 \geq t,\, T_3 \geq t] $$

$$
+ P[T_1 \geq t,\, T_2 \leq t,\, T_3 \geq t]
+ P[T_1 \leq t,\, T_2 \geq t,\, T_3 \leq t].
$$

De plus, les variables \( T_i \) sont indépendantes, donc :

\[
P[T_1 \geq t,\, T_2 \geq t,\, T_3 \geq t] 
= P(T_1 \geq t)\, P(T_2 \geq t)\, P(T_3 \geq t)
= (1 - F_T(t))^3.
\]

De même :

\[
P[T_1 \leq t,\, T_2 \geq t,\, T_3 \geq t] 
= P(T_1 \leq t)\, P(T_2 \geq t)\, P(T_3 \geq t)
= F_T(t)\,(1 - F_T(t))^2.
\]

Les probabilités des deux autres événements sont identiques.

On en déduit donc :

\[
P[X \geq t] = (1 - F_T(t))^3 + 3\,F_T(t)\,(1 - F_T(t))^2.
\]

et ainsi (comme \( X \) suit une loi à densité) :

\[
F_X(t) = 1 - P(X \geq t)= 1 - \big[(1 - F_T(t))^3 + 3F_T(t)(1 - F_T(t))^2 \big].
\]
}
\item   \question{Montrer que $X$ suit la loi de densité $f_X$ définie pour tout réel strictement positif par:
$$f_X(t)=6\lambda e^{-2\lambda t}-6\lambda e^{-3\lambda t}$$}
\reponse{Soit \( t \geq 0 \). On rappelle que :
\[
F_T(t) = \int_{0}^{t} \lambda e^{-\lambda u}\, du =1-e^{-\lambda t}.
\]

D’après la question précédente, on a alors :
$$
F_X(t) = 1 - \big[\big(e^{-\lambda t}\big)^3 - 3(1-e^{-\lambda t})\big(e^{-\lambda t}\big)^2 \big]$$
En simplifiant :
\[
F_X(t) = 1 - 3 e^{-2\lambda t} + 2 e^{-3\lambda t}.
\]

Si \( t < 0 \), on a \( F_T(t) = 0 \), et on en déduit immédiatement que \( F_X(t) = 0 \).

\medskip

La fonction de répartition \( F_X \) est partout dérivable sauf en \( t = 0 \).  
On en déduit que \( X \) admet une densité \( f_X \) donnée par :
\[
f_X(t) = F_X'(t).
\]
Ainsi :
\[
f_X(t) =
\begin{cases}
0, & \text{si } t < 0, \\
6\lambda e^{-2\lambda t} - 6\lambda e^{-3\lambda t}, & \text{si } t > 0.
\end{cases}
\]}
\item   \question{Calculer l'espérance de $X$.}
\reponse{L’espérance \( \mathbb{E}(X) \) se calcule par le théorème de transfert :
\[
\mathbb{E}(X) = \int_{-\infty}^{+\infty} t f_X(t)\, dt 
= \int_{0}^{+\infty} t \left( 6\lambda e^{-2\lambda t} - 6\lambda e^{-3\lambda t} \right) dt
= 6\lambda \int_{0}^{+\infty} t e^{-2\lambda t}\, dt - 6\lambda \int_{0}^{+\infty} t e^{-3\lambda t}\, dt.
\]

\medskip

On calcule la première intégrale par intégration par parties :

\[
\int_{0}^{+\infty} t e^{-2\lambda t}\, dt 
= \left[ -\frac{t e^{-2\lambda t}}{2\lambda} \right]_0^{+\infty} + \int_{0}^{+\infty} \frac{e^{-2\lambda t}}{2\lambda}\, dt
= 0 + \frac{1}{(2\lambda)^2}
= \frac{1}{4\lambda^2}.
\]

De même :
\[
\int_{0}^{+\infty} t e^{-3\lambda t}\, dt = \frac{1}{(3\lambda)^2} = \frac{1}{9\lambda^2}.
\]

Ainsi :
\[
\mathbb{E}(X) = 6\lambda \left( \frac{1}{4\lambda^2} - \frac{1}{9\lambda^2} \right)
= 6\lambda \cdot \frac{5}{36\lambda^2}
= \frac{5}{6\lambda}.
\]
}
\item   \question{Soient les réels $t>0,h>0$. Calculer la probabilité conditionnelle $P(\big[X\geq t+h \vert X\geq t \big]$ puis calculer la limite de la probabilité, à $h$ fixé, lorsque $t \to +\infty$.}
\reponse{
Par définition d’une probabilité conditionnelle, on a :
\[
P(X \geq t + h \mid X \geq t)
= \frac{P([X \geq t + h] \cap [X \geq t])}{P(X \geq t)}.
\]

Or, on a l’inclusion \( \{ X \geq t + h \} \subset \{ X \geq t \} \),
donc l’intersection de ces deux événements est \( \{ X \geq t + h \} \).
Ainsi :
\[
P(X \geq t + h \mid X \geq t)
= \frac{P(X \geq t + h)}{P(X \geq t)}.
\]

D’après les résultats précédents :
\[
P(X \geq t) = (1 - F_T(t))^3 + 3F_T(t)(1 - F_T(t))^2
= 3 e^{-2\lambda t} - 2 e^{-3\lambda t},
\]
et de même :
\[
P(X \geq t + h) = 3 e^{-2\lambda (t + h)} - 2 e^{-3\lambda (t + h)}.
\]

Ainsi :
\[
P(X \geq t + h \mid X \geq t)
= \frac{3 e^{-2\lambda (t + h)} - 2 e^{-3\lambda (t + h)}}{3 e^{-2\lambda t} - 2 e^{-3\lambda t}}.
\]

En multipliant le numérateur et le dénominateur par \( e^{2\lambda t} \), on obtient :
\[
P(X \geq t + h \mid X \geq t)
= \frac{3 e^{-2\lambda h} - 2 e^{-\lambda (t + 3h)}}{3 - 2 e^{-\lambda t}}.
\]

Comme \( \lambda > 0 \), on a :
\[
\lim_{t \to +\infty} e^{-\lambda t} = 0,
\]
donc :
\[
\lim_{t \to +\infty} P(X \geq t + h \mid X \geq t)
= e^{-2\lambda h}.
\]}

\end{enumerate}

}


