\uuid{td23}
\titre{Loi normale}
\theme{}
\auteur{}
\datecreate{2025-03-20}
\organisation{}

\contenu{

\texte{
On observe $X_1, \ldots, X_n$ indépendantes et de même loi. On suppose qu’il existe $\theta > 0$ tel que cette loi admette la densité $f_\theta(x) = \frac{1}{\sqrt{2\pi\theta}} \exp\left(-\frac{x^2}{2\theta}\right)$.
}

\begin{enumerate}
    \item \question{Proposer un estimateur $\theta_n$ de $\theta$ et étudier sa loi.}
    \indication{}
    \reponse{}
    \item \question{Soit $\alpha \in ]0, 1[$. Construire un intervalle de confiance $I_{1-\alpha} = [A_1, B_1]$ de niveau $(1-\alpha)$ pour $\theta$, tel que $P(\theta < A_1) = P(\theta > B_1) = \alpha/2$. Proposer deux autres intervalles de confiance de niveau $(1-\alpha)$, notés $I_{1-\alpha}^{(2)} = [A_2, B_2]$ et $I_{1-\alpha}^{(3)} = [A_3, B_3]$, tels que $P(\theta < A_2) = P(\theta > B_3) = \alpha$. Quel intervalle vous semble préférable ?}
    \indication{}
    \reponse{}
    \item \question{Donner la loi asymptotique de $\theta_n$ et en déduire un intervalle de confiance asymptotique $J_{1-\alpha}$.}
    \indication{}
    \reponse{}
    \item \question{On se place dans la situation où $n = 10$, $\theta_n(\omega) = 2$ et $\alpha = 5\%$. Comparer l’intervalle de confiance non asymptotique $I_{1-\alpha}^{(1)}$ à l’intervalle de confiance asymptotique $J_{1-\alpha}$.}
    \indication{}
    \reponse{}
    \item \question{On souhaite à présent tester $H_0 : \theta \leq 3$ contre $H_1 : \theta > 3$. (a) À partir de la définition du niveau, construire un test $T_\alpha(X)$ de niveau non-asymptotique $\alpha$ pour les hypothèses $H_0$ et $H_1$. Déterminer la taille de ce test et sa fonction puissance. (b) Peut-on retrouver ce résultat grâce au lien avec les intervalles de confiance de la question 2 ? (c) Pour un seuil $\alpha = 5\%$, lorsque $n = 10$ et qu’on observe $\theta_n(\omega) = 4$, rejette-t-on l’hypothèse nulle ? Calculer la p-valeur du test et la comparer à $\alpha$.}
    \indication{}
    \reponse{}
\end{enumerate}

}