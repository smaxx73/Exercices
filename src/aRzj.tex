\titre{Recherche d'un estimateur}
\theme{probabilité}
\auteur{}
\organisation{AMSCC}


\texte{ Soient $\theta$ un réel strictement positif et $X_1, X_2, \ldots, X_n$ un échantillon dont la loi mère a pour densité la fonction $f$ définie sur $\R$ par : $$f \colon x \mapsto \frac{1}{\theta^2} x e^{-\frac{x}{\theta}} {\textbf{1}}_{]0;+\infty[}(x)$$. }

\begin{enumerate}
	\item \question{ Déterminer un estimateur de $\theta$ issu de la méthode du maximum de vraisemblance. } \reponse{ $\hat{\theta}=\frac{1}{2 n} \sum_{i=1}^n X_i$ }
	\item \question{ Déterminer le biais de cet estimateur.  \\(Indication : on admet que pour tout $n \in \N$, $\int_0^{+\infty} x^n e^{-x} \, \mathrm{d}x = n!$.)}
\end{enumerate}