\uuid{1diV}
\exo7id{4001}
\auteur{quercia}
\organisation{exo7}
\datecreate{2010-03-11}
\isIndication{false}
\isCorrection{false}
\chapitre{Dérivabilité des fonctions réelles}
\sousChapitre{Fonctions convexes}

\contenu{
\texte{
Soit $(f_n)$ une suite de fonctions convexes sur $[a,b]$ convergeant
simplement vers une fonction $f$ supposée continue.

Soit $\varepsilon > 0$.
}
\begin{enumerate}
    \item \question{Montrer qu'il existe $p \in \N^*$ tel que :
      $\forall\ x,y \in {[a,b]},\
      |x-y| \le \frac {b-a}p  \Rightarrow  |f(x) - f(y)| \le \varepsilon$.

On choisit un tel $p$, et on fixe une subdivision $(a_k)$ de $[a,b]$ telle
que $a_k = a + k\frac {b-a}p$.}
    \item \question{Soit $t \in {[0,1]}$. Encadrer $f_n\bigl(ta_k + (1-t)a_{k+1}\bigr)$ par deux
      fonctions affines de t en utilisant la convexité de~$f_n$.}
    \item \question{Montrer que la suite $(f_n)$ converge uniformément vers $f$.}
\end{enumerate}
}
