\titre{Problème de clés}
\theme{probabilités}
\auteur{}
\organisation{AMSCC}	

\contenu{

Un gardien de nuit doit ouvrir une des portes à contrôler durant sa tournée dans le noir. Il possède 10 clés d'allure semblable, mais une seule peut ouvrir la porte en question. Le gardien dispose de deux méthodes.
\begin{itemize}
	\item Méthode A : Il pose les 10 clés devant lui et les essaye l'une après l'autre dans l'ordre dans lequel elles se présentent ;
	\item Méthode B : Il essaye une clé après avoir agité le trousseau, en recommençant cette opération jusqu'à ce qu'il trouve la bonne clé.
\end{itemize}

On appelle $X_A$ (respectivement $X_B$) la variable aléatoire, définie sur l'espace probabilisé $(\Omega, \mathcal{T}, \prob)$, qui désigne le nombre de clés essayées (y compris celle qui donne satisfaction) par la méthode A (respectivement B).

Rappel : soit $a \in ]0 ; 1[$, alors $\displaystyle \sum_{k=0}^{n} a^k = \frac{1 - a^{n+1}}{1 - a}$ et $\displaystyle \sum_{k=0}^{+\infty} a^k = \frac{1}{1 - a}$.

\begin{enumerate}
	\item \question{ Montrer que pour tout $k \in \{1, \dots, 10\}$, $\prob(X_A = k) = p$ où $p$ est une valeur constante à déterminer.  }
	\item \question{ Quelle est la loi suivie par $X_B$ ? }
	\item \question{ Quelle est la probabilité d'essayer plus de 8 clés : par la méthode A ? Par la méthode B ? On notera $H$ l'événement : « essayer plus de 8 clés ». }
	\item \question{ On admet que le gardien utilise la méthode A deux fois sur trois. Quelle est la probabilité conditionnelle que le gardien utilise la méthode B sachant que les 8 premiers essais ont échoué ? }
\end{enumerate}

}