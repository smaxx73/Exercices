\uuid{JqDI}
\exo7id{6046}
\auteur{queffelec}
\organisation{exo7}
\datecreate{2011-10-16}
\isIndication{false}
\isCorrection{false}
\chapitre{Espace topologique, espace métrique}
\sousChapitre{Espace topologique, espace métrique}

\contenu{
\texte{

}
\begin{enumerate}
    \item \question{Soit $X$ un espace topologique, et $D$ un sous-ensemble (partout) dense dans
$X$. Montrer qu'il est aussi équivalent de dire 

(i) Le complémentaire de $D$ est d'intérieur vide.

(ii) Si $F$ est un fermé contenant $D$, alors $F=X$.

(iii) $D$ rencontre tout ouvert non vide de $X$.

Montrer qu'un ensemble $A\subset X$ rencontre toute partie dense dans $X$ si et
seulement si il est d'intérieur non vide.}
    \item \question{Soit $E$ et $G$ deux ouverts denses dans $X$; montrer que $E\cap G$ est encore
dense dans $X$. En déduire que toute intersection dénombrable d'ouverts denses
est une intersection décroissante d'ouverts denses.}
\end{enumerate}
}
