\uuid{bd38}
\titre{Factorisation de polynôme et décomposition en éléments simples}
\niveau{L1}
\module{Algèbre}
\chapitre{Polynômes}
\sousChapitre{Factorisation et fractions rationnelles}
\theme{Racines de polynômes, décomposition en éléments simples}
\auteur{}
\datecreate{2026-01-13}
\organisation{}
\difficulte{3}
\contenu{
	\texte{
		On pose le polynôme $$Q(X)=X^3-(\sqrt{2}+5)X^2+(1+5\sqrt{2})X-5.$$}
	\begin{enumerate}
		\item \question{Montrer que $5$ est racine simple de $Q$.}
		\indication{Calculer $Q(5)$ et vérifier que $Q'(5) \neq 0$.}
		\reponse{
			$$Q(5) = 125 - (\sqrt{2}+5) \times 25 + (1+5\sqrt{2}) \times 5 - 5$$
			$$= 125 - 25\sqrt{2} - 125 + 5 + 25\sqrt{2} - 5 = 0$$
			
			Donc $5$ est racine de $Q$.
			
			Calculons $Q'(X) = 3X^2 - 2(\sqrt{2}+5)X + (1+5\sqrt{2})$.
			
			$$Q'(5) = 3 \times 25 - 2(\sqrt{2}+5) \times 5 + (1+5\sqrt{2})$$
			$$= 75 - 10\sqrt{2} - 50 + 1 + 5\sqrt{2}$$
			$$= 26 - 5\sqrt{2} \neq 0$$
			
			Donc $5$ est racine simple de $Q$.
		}
		
		\item \question{Déterminer $S$ le polynôme de degré $2$ vérifiant l'égalité suivante:
			$$Q(X)=(X-5)S(X).$$
			Le polynôme $S$ est-il irréductible dans $\mathbb{R}[X]$?}
		\indication{Effectuer la division euclidienne de $Q$ par $(X-5)$ ou utiliser la méthode d'identification.}
		\reponse{
			On cherche $S(X) = X^2 + pX + q$ tel que $Q(X) = (X-5)S(X)$.
			
			Par identification des coefficients ou division euclidienne :
			$$(X-5)(X^2+pX+q) = X^3 + pX^2 + qX - 5X^2 - 5pX - 5q$$
			$$= X^3 + (p-5)X^2 + (q-5p)X - 5q$$
			
			En identifiant avec $Q(X) = X^3 - (\sqrt{2}+5)X^2 + (1+5\sqrt{2})X - 5$ :
			\begin{itemize}
				\item Coefficient de $X^2$ : $p - 5 = -(\sqrt{2}+5)$ donc $p = -\sqrt{2}$
				\item Terme constant : $-5q = -5$ donc $q = 1$
				\item Vérification avec le coefficient de $X$ : $q - 5p = 1 - 5(-\sqrt{2}) = 1 + 5\sqrt{2}$ ✓
			\end{itemize}
			
			Donc $S(X) = X^2 - \sqrt{2}X + 1$.
			
			Pour savoir si $S$ est irréductible dans $\mathbb{R}[X]$, calculons son discriminant :
			$$\Delta = 2 - 4 = -2 < 0$$
			
			Puisque $\Delta < 0$, le polynôme $S$ n'a pas de racine réelle, donc $S$ est irréductible dans $\mathbb{R}[X]$.
		}
		
		\item \question{Déterminer les racines complexes de $S$ et donner sa décomposition en facteurs irréductibles dans $\mathbb{C}[X]$.}
		\indication{Utiliser la formule du discriminant pour trouver les racines complexes.}
		\reponse{
			Le discriminant est $\Delta = -2$, donc les racines sont :
			$$z_1 = \frac{\sqrt{2} + i\sqrt{2}}{2} = \frac{\sqrt{2}}{2}(1+i) \quad \text{et} \quad z_2 = \overline{z_1} = \frac{\sqrt{2}}{2}(1-i)$$
			
			La décomposition en facteurs irréductibles dans $\mathbb{C}[X]$ est :
			$$S(X) = \left(X - \frac{\sqrt{2}}{2}(1+i)\right)\left(X - \frac{\sqrt{2}}{2}(1-i)\right)$$
			
			La décomposition complète de $Q$ dans $\mathbb{C}[X]$ est :
			$$Q(X) = (X-5)\left(X - \frac{\sqrt{2}}{2}(1+i)\right)\left(X - \frac{\sqrt{2}}{2}(1-i)\right)$$
		}
		
		\item \question{Dans cette question on se propose de déterminer la décomposition en éléments simples sur $\mathbb{R}(X)$ de $\frac{1}{Q(X)},$ c'est-à-dire l'existence de 3 coefficients $a,b,c$ réels tels que
			$$\frac{1}{Q(X)}=\frac{a}{X-5}+\frac{bX+c}{S(X)}.$$
		}
		On supposera que cette décomposition existe pour la suite.
		\begin{itemize}
			\item \question{Calculer $\lim_{X \to +\infty}\frac{X}{Q(X)}$ de deux façons différentes puis en déduire que $a+b=0$.}
			\indication{Première méthode : comportement asymptotique. Seconde méthode : utiliser la décomposition proposée.}
			\reponse{
				\textbf{Première méthode :} 
				$$\lim_{X \to +\infty} \frac{X}{Q(X)} = \lim_{X \to +\infty} \frac{X}{X^3(1 + o(1))} = \lim_{X \to +\infty} \frac{1}{X^2} = 0$$
				
				\textbf{Seconde méthode :} En multipliant la décomposition par $X$ :
				$$\frac{X}{Q(X)} = \frac{aX}{X-5} + \frac{(bX+c)X}{S(X)}$$
				
				Quand $X \to +\infty$ :
				$$\frac{aX}{X-5} \sim \frac{aX}{X} = a \quad \text{et} \quad \frac{(bX+c)X}{S(X)} \sim \frac{bX^2}{X^2} = b$$
				
				Donc $\lim_{X \to +\infty} \frac{X}{Q(X)} = a + b$.
				
				Par unicité de la limite : $a + b = 0$.
			}
			
			\item \question{Calculer la limite quand $X$ tend vers $5$ de $\frac{X-5}{Q(X)}$ de deux façons différentes. En déduire $a$ puis $b$.}
			\indication{Utiliser le fait que $5$ est racine simple.}
			\reponse{
				\textbf{Première méthode :} Puisque $Q(X) = (X-5)S(X)$ :
				$$\lim_{X \to 5} \frac{X-5}{Q(X)} = \lim_{X \to 5} \frac{X-5}{(X-5)S(X)} = \lim_{X \to 5} \frac{1}{S(X)} = \frac{1}{S(5)}$$
				
				Calculons $S(5) = 25 - 5\sqrt{2} + 1 = 26 - 5\sqrt{2}$.
				
				Donc $\lim_{X \to 5} \frac{X-5}{Q(X)} = \frac{1}{26-5\sqrt{2}}$.
				
				\textbf{Seconde méthode :} En multipliant la décomposition par $(X-5)$ :
				$$\frac{X-5}{Q(X)} = a + \frac{(bX+c)(X-5)}{S(X)}$$
				
				Quand $X \to 5$, le second terme tend vers $0$ car $(X-5) \to 0$ et $S(5) \neq 0$.
				
				Donc $\lim_{X \to 5} \frac{X-5}{Q(X)} = a$.
				
				Par unicité : $a = \frac{1}{26-5\sqrt{2}}$.
				
				On peut rationaliser : $a = \frac{26+5\sqrt{2}}{(26-5\sqrt{2})(26+5\sqrt{2})} = \frac{26+5\sqrt{2}}{676-50} = \frac{26+5\sqrt{2}}{626}$.
				
				Comme $a + b = 0$, on a $b = -a = -\frac{26+5\sqrt{2}}{626}$.
			}
			
			\item \question{En posant $X=0$ déterminer $c$ puis donner la décomposition en éléments simples de $\frac{1}{Q(X)}$ sur $\mathbb{R}(X)$.}
			\indication{Évaluer l'égalité en $X=0$ et utiliser les valeurs de $a$ et $b$ déjà trouvées.}
			\reponse{
				En posant $X = 0$ dans la décomposition :
				$$\frac{1}{Q(0)} = \frac{a}{-5} + \frac{c}{S(0)}$$
				
				On a $Q(0) = -5$ et $S(0) = 1$, donc :
				$$\frac{1}{-5} = \frac{a}{-5} + c$$
				
				D'où $-\frac{1}{5} = -\frac{a}{5} + c$, soit $c = \frac{a-1}{5}$.
				
				Avec $a = \frac{26+5\sqrt{2}}{626}$ :
				$$c = \frac{1}{5}\left(\frac{26+5\sqrt{2}}{626} - 1\right) = \frac{1}{5} \cdot \frac{26+5\sqrt{2}-626}{626} = \frac{-600+5\sqrt{2}}{3130}$$
				
				On peut simplifier en divisant numérateur et dénominateur par 5 : $c = \frac{-120+\sqrt{2}}{626}$.
				
				La décomposition en éléments simples est :
				$$\frac{1}{Q(X)} = \frac{26+5\sqrt{2}}{626(X-5)} + \frac{-\frac{26+5\sqrt{2}}{626}X + \frac{-120+\sqrt{2}}{626}}{X^2-\sqrt{2}X+1}$$
				
				Soit, en factorisant par $\frac{1}{626}$ et en arrangeant les signes :
				$$\frac{1}{Q(X)} = \frac{1}{626}\left[\frac{26+5\sqrt{2}}{X-5} - \frac{(26+5\sqrt{2})X + 120-\sqrt{2}}{X^2-\sqrt{2}X+1}\right]$$
			}
		\end{itemize}
	\end{enumerate}
}