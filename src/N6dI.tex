\uuid{N6dI}
\exo7id{893}
\auteur{legall}
\organisation{exo7}
\datecreate{1998-09-01}
\isIndication{true}
\isCorrection{true}
\chapitre{Espace vectoriel}
\sousChapitre{Définition, sous-espace}

\contenu{
\texte{
Soit $E$ un espace vectoriel.
}
\begin{enumerate}
    \item \question{Soient $F$ et $G$ deux sous-espaces de $E$. Montrer que
  $$F\cup G \hbox{ est un sous-espace vectoriel de } E
 \quad \Longleftrightarrow \quad F\subset G \hbox{ ou } G \subset F.$$}
    \item \question{Soit $H$ un troisi\`eme sous-espace vectoriel de $E$. Prouver
  que
  $$G \subset F \Longrightarrow F\cap(G+H) = G + (F\cap H) .$$}
\reponse{
Sens $\Leftarrow$. Si $F\subset G$ alors $F\cup G=G$ donc $F\cup
  G$ est un sous-espace vectoriel.  De m\^eme si $G\subset F$.
  
  Sens $\Rightarrow$. On suppose que $F\cup G$ est un sous-espace
  vectoriel.  Par l'absurde supposons que $F$ n'est pas inclus dans
  $G$ et que $G$ n'est pas inclus dans $F$. Alors il existe $x\in
  F\setminus G$ et $y\in G\setminus F$. Mais alors $x\in F\cup G$,
  $y\in F\cup G$ donc $x+y\in F\cup G$ (car $F\cup G$ est un
  sous-espace vectoriel). Comme $x+y\in F\cup G$ alors
$x+y\in F$ ou $x+y\in G$.
\begin{itemize}
Si $x+y\in F$ alors, comme $x\in F$, $(x+y)+(-x) \in F$ donc $y\in F$, ce qui est absurde.
Si $x+y\in G$ alors, comme $y\in G$, $(x+y)+(-y) \in G$ donc $x\in G$, ce qui est absurde.
\end{itemize}  
Dans les deux cas nous obtenons une contradiction. Donc $F$ est inclus
dans $G$ ou $G$ est inclus dans $F$.
Supposons $G\subset F$.
\begin{itemize}
Inclusion $\supset$. Soit $x\in G+(F\cap H)$. Alors il existe $a\in G$, $b\in F\cap H$ tels que $x=a+b$. Comme $G\subset F$ alors $a\in F$, de plus $b\in F$ donc $x=a+b\in F$.
D'autre part $a\in G$, $b\in H$, donc $x=a+b\in G+H$. Donc $x\in F\cap(G+H)$.
Inclusion $\subset$. Soit $x\in F\cap(G+H)$. $x\in G+H$ alors il existe $a\in G$,
$b\in H$ tel que $x=a+b$. Maintenant $b=x-a$ avec $x\in F$ et $a\in G\subset F$, donc $b\in F$,
donc $b\in F\cap H$. Donc $x=a+b\in G+(F\cap H)$.
\end{itemize}
}
\indication{\begin{enumerate}
\item Pour le sens $\Rightarrow$ : raisonner par l'absurde et prendre
  un vecteur de $F\setminus G$ et un de $G\setminus F$. Regarder la
  somme de ces deux vecteurs.
\item Raisonner par double inclusion, revenir aux vecteurs.
\end{enumerate}}
\end{enumerate}
}
