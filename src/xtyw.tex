\uuid{xtyw}
\titre{Maximum de vraisemblance pour une loi uniforme}
\niveau{L2} 
\module{Probabilités et Statistiques} 
\chapitre{Échantillonnage et estimation}
\sousChapitre{Maximum de vraisemblance}
\theme{Estimation d'un paramètre, loi uniforme, estimateur biaisé}
\auteur{AMSCC}
\datecreate{2025-12-02}
\organisation{AMSCC}
\difficulte{3}
\contenu{
    \texte{ 
    Une entreprise de transport maritime mesure la profondeur de corrosion (en mm) sur des coques de navires. On modélise cette profondeur par une variable aléatoire $X$ suivant une \textbf{loi uniforme} sur l'intervalle $[0, \theta]$ où $\theta > 0$ est inconnu.
    
    \textbf{Rappels :} Pour $X \sim \mathcal{U}[0, \theta]$, la densité est donnée par :
    $$f_\theta(x) = \begin{cases} \dfrac{1}{\theta} & \text{si } 0 \leq x \leq \theta \\ 0 & \text{sinon} \end{cases}$$
    
    On dispose d'un échantillon $(X_1, \ldots, X_n)$ de $n$ mesures indépendantes. On note $(x_1, \ldots, x_n)$ les valeurs observées.
    }
    
    \begin{enumerate}
	    \item   \question{Écrire la fonction de vraisemblance $L(x_1, \ldots, x_n, \theta)$ pour cet échantillon.}
	            \indication{La vraisemblance est le produit des densités. Attention : elle est nulle si $\theta < \max(x_1, \ldots, x_n)$.}
                \reponse{
                Par définition, la fonction de vraisemblance est :
                $$L(x_1, \ldots, x_n, \theta) = \prod_{i=1}^{n} f_\theta(x_i)$$
                
                Pour que cette fonction soit non nulle, il faut que tous les $x_i$ appartiennent à $[0, \theta]$, c'est-à-dire :
                $$\forall i \in \{1, \ldots, n\}, \quad 0 \leq x_i \leq \theta$$
                
                Cette condition est équivalente à :
                $$\theta \geq \max(x_1, \ldots, x_n)$$
                
                Par conséquent :
                $$L(x_1, \ldots, x_n, \theta) = \begin{cases} 
                \dfrac{1}{\theta^n} & \text{si } \theta \geq \max(x_1, \ldots, x_n) \\ 
                0 & \text{sinon}
                \end{cases}$$
                }
                
        \item   \question{Montrer graphiquement ou par un raisonnement que $L(\theta)$ est maximale quand $\theta$ est minimal, sous la contrainte $\theta \geq \max(x_1, \ldots, x_n)$.}
                \indication{Étudier la fonction $\theta \mapsto \frac{1}{\theta^n}$ sur l'intervalle $[\max(x_1, \ldots, x_n), +\infty[$.}
                \reponse{
                Pour $\theta \geq \max(x_1, \ldots, x_n)$, on a :
                $$L(\theta) = \frac{1}{\theta^n}$$
                
                Cette fonction est strictement décroissante car sa dérivée est :
                $$\frac{dL}{d\theta} = -\frac{n}{\theta^{n+1}} < 0$$
                
                Donc $L(\theta)$ est maximale lorsque $\theta$ prend sa valeur minimale possible, c'est-à-dire :
                $$\theta = \max(x_1, \ldots, x_n)$$
                
                \textbf{Graphiquement :} La fonction $L(\theta)$ vaut 0 pour $\theta < \max(x_i)$, puis saute à une valeur positive en $\theta = \max(x_i)$, et décroît ensuite vers 0 quand $\theta \to +\infty$.
                }
                
        \item   \question{En déduire que l'estimateur du maximum de vraisemblance est :
                $$\hat{\Theta}_n = \max(X_1, \ldots, X_n)$$}
                \indication{Remplacer les observations $x_i$ par les variables aléatoires $X_i$.}
                \reponse{
                D'après la question précédente, la vraisemblance est maximale pour :
                $$\hat{\theta} = \max(x_1, \ldots, x_n)$$
                
                En remplaçant les observations par les variables aléatoires, on obtient l'estimateur du maximum de vraisemblance :
                $$\boxed{\hat{\Theta}_n = \max(X_1, \ldots, X_n)}$$
                
                Cet estimateur est aussi noté $\hat{\Theta}_n = X_{(n)}$ (statistique d'ordre maximale).
                }
                
        \item   \question{On admet que $\mathbb{E}(\hat{\Theta}_n) = \dfrac{n}{n+1}\theta$. Cet estimateur est-il sans biais ? Proposer un estimateur sans biais $\widetilde{\Theta}_n$ de $\theta$.}
                \indication{Un estimateur est sans biais si son espérance est égale au paramètre à estimer. Pour corriger le biais, multiplier par un coefficient approprié.}
                \reponse{
                Le biais de l'estimateur $\hat{\Theta}_n$ est :
                $$b(\hat{\Theta}_n) = \mathbb{E}(\hat{\Theta}_n) - \theta = \frac{n}{n+1}\theta - \theta = -\frac{1}{n+1}\theta \neq 0$$
                
                \textbf{Donc $\hat{\Theta}_n$ est biaisé.} Plus précisément, il sous-estime systématiquement $\theta$.
                
                Pour construire un estimateur sans biais, on pose :
                $$\widetilde{\Theta}_n = \frac{n+1}{n} \hat{\Theta}_n = \frac{n+1}{n} \max(X_1, \ldots, X_n)$$
                
                Vérifions :
                $$\mathbb{E}(\widetilde{\Theta}_n) = \frac{n+1}{n} \mathbb{E}(\hat{\Theta}_n) = \frac{n+1}{n} \times \frac{n}{n+1}\theta = \theta$$
                
                \textbf{Donc $\boxed{\widetilde{\Theta}_n = \dfrac{n+1}{n} \max(X_1, \ldots, X_n)}$ est un estimateur sans biais de $\theta$.}
                }
                
        \item   \question{Application numérique : On observe 5 mesures de corrosion : \textbf{2.3, 3.1, 2.8, 3.5, 2.6 mm}. Calculer les estimations $\hat{\theta}$ et $\widetilde{\theta}$.}
                \indication{Identifier le maximum des observations, puis appliquer les formules.}
                \reponse{
                On a $n = 5$ mesures : $x_1 = 2.3$, $x_2 = 3.1$, $x_3 = 2.8$, $x_4 = 3.5$, $x_5 = 2.6$.
                
                Le maximum des observations est :
                $$\max(x_1, \ldots, x_5) = 3.5 \text{ mm}$$
                
                \textbf{Estimation par maximum de vraisemblance (biaisée) :}
                $$\hat{\theta} = 3.5 \text{ mm}$$
                
                \textbf{Estimation sans biais (corrigée) :}
                $$\widetilde{\theta} = \frac{n+1}{n} \times \hat{\theta} = \frac{6}{5} \times 3.5 = 1.2 \times 3.5 = \boxed{4.2 \text{ mm}}$$
                
                \textbf{Interprétation :} L'estimateur du maximum de vraisemblance sous-estime la profondeur maximale de corrosion. L'estimateur corrigé $\widetilde{\theta} = 4.2$ mm est plus prudent et sans biais.
                }
    \end{enumerate}
}
