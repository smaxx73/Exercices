\uuid{tfXD}
\exo7id{6945}
\auteur{ruette}
\organisation{exo7}
\datecreate{2013-01-24}
\isIndication{false}
\isCorrection{true}
\chapitre{Loi, indépendance, loi conditionnelle}
\sousChapitre{Loi, indépendance, loi conditionnelle}

\contenu{
\texte{
Soit $X$ une variable aléatoire de loi binomiale $\mathcal{B}(n,p)$.
}
\begin{enumerate}
    \item \question{Calculer la fonction génératrice de $X$.}
\reponse{$\displaystyle G_X(z)=\sum_{k=0}^nC^k_np^k(1-p)^{n-k}z^k
=\sum_{k=0}^nC^k_n(pz)^k(1-p)^{n-k}$ donc $G_X(z)=(pz+1-p)^n$.}
    \item \question{Soit $Y$ une variable aléatoire de loi $\mathcal{B}(m,p)$, indépendante de $X$.
Quelle est la fonction génératrice de $X+Y$ ? En déduire que
$\mathcal{B}(n,p)*\mathcal{B}(m,p)=\mathcal{B}(n+m,p)$.}
\reponse{$G_Y(z)=(pz+1-p)^m$ par a). Par indépendance,
$G_{X+Y}(z)=G_X(z)G_Y(z)=(pz+1-p)^{n+m}$. C'est la fonction génératrice
de la loi binomiale $\mathcal{B}(n+m,p)$. Or la loi de $X+Y$ est  
$\mathcal{B}(n,p)*\mathcal{B}(m,p)$, donc $\mathcal{B}(n,p)*\mathcal{B}(m,p)=
\mathcal{B}(n+m,p)$.}
\end{enumerate}
}
