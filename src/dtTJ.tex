\uuid{dtTJ}
\exo7id{1634}
\auteur{barraud}
\organisation{exo7}
\datecreate{2003-09-01}
\isIndication{false}
\isCorrection{false}
\chapitre{Réduction d'endomorphisme, polynôme annulateur}
\sousChapitre{Diagonalisation}

\contenu{
\texte{
On considère un endomorphisme $f$ d'un $\C$ espace vectoriel $E$ de
dimension finie $n$, tel que $f^2$ est diagonalisable. Le but de cet
exercice est de démontrer que :
$$
  f\text{ est diagonalisable }\Leftrightarrow \mathrm{Ker} f=\mathrm{Ker} f^2
$$
}
\begin{enumerate}
    \item \question{On suppose que $f$ est diagonalisable. On note
  $\alpha_{1},...,\alpha_{r}$ les valeurs propres (distinctes) de
  $A$, et $E_{1},...,E_{r}$ les espaces propres associés.
  \begin{enumerate}}
    \item \question{Montrer que si $\mathrm{Ker} f=\{0\}$ alors $\mathrm{Ker} f^{2}=\{0\}$.}
    \item \question{On suppose maintenant que $\mathrm{Ker} f\neq\{0\}$. On note
    $\alpha_{\alpha_{1}},...,\alpha_{\alpha_{r}}$ les autres valeurs
    propres de $f$, et $E_{0},...,E_{r}$ ses espaces propres. En
    utilisant que $E=E_{0}\oplus E_{1}\oplus ...\oplus E_{r}$, montrer
    que si $f^{2}(x)=0$ alors $f(x)=0$. En déduire que $\mathrm{Ker} f=\mathrm{Ker} f^2$.}
\end{enumerate}
}
