\uuid{b6cm}
\chapitre{Probabilité continue}
\sousChapitre{Densité de probabilité}
\titre{ Autour d'une loi exponentielle }
\theme{variables aléatoires à densité, loi exponentielle}
\auteur{Maxime Nguyen}
\datecreate{2023-10-16}
\organisation{AMSCC}

\contenu{
\texte{   Soient $X$ et $Y$ deux variables aléatoires indépendantes suivant chacune une loi exponentielle $\mathcal{E}(3)$. On rappelle qu'une densité de probabilité de la loi exponentielle $\mathcal{E}(\lambda)$ est donnée par : $$f(x) = \begin{cases} 
    \lambda e^{-\lambda x} & \text{ si } x \geq 0 \\
    0 & \text{ sinon }
\end{cases}.$$
  
  On note $Z = \min(X,Y)$ la variable aléatoire donnant le minimum de $X$ et $Y$.  }

\begin{enumerate}
    \item \question{ Déterminer $\prob(X \geq t)$ pour tout $t \in \R$. }
    \reponse{
        Soit $t \in \R$. Si $t \geq 0$, on a : 
        \begin{align*}
            \prob(X \geq t) &= \int_t^{+\infty} f(x) \d x \\
            &= \int_t^{+\infty} 3 e^{-3x} \d x \\
            &= \left[ -e^{-3x} \right]_t^{+\infty} \\
            &= e^{-3t}.
        \end{align*}
        Si $t < 0$, on a $\prob(X \geq t) = 1$.
    }
    \item \question{ Déterminer $\prob(Z \geq t)$ pour tout $t \in \R$.}
    \reponse{
        Soit $t \in \R$. Si $t \geq 0$, on a : 
        \begin{align*}
            \prob(Z \geq t) &= \prob(X \geq t \text{ et } Y \geq t) \\
            &= \prob(X \geq t) \prob(Y \geq t) \text{ par indépendance de $X$ et $Y$} \\
            &= e^{-3t} \times e^{-3t} \\
            &= e^{-6t}.
        \end{align*}
        Si $t < 0$, on a $\prob(Z \geq t) = 1$.

        On voit ainsi que $Z = \min(X,Y)$ suit une loi exponentielle $\mathcal{E}(6)$.
    }
    %\item \question{ Déterminer la loi de $Z$. }
\end{enumerate}
}