\uuid{Rg7q}
\titre{Calcul d'intégrales par reconnaissance de dérivées}
\niveau{L2}
\module{Analyse}
\chapitre{Intégration}
\sousChapitre{Intégrale de Riemann}
\theme{trigonométrie, exponentielle, logarithme}
\auteur{Erwan L'Haridon}
\datecreate{2026-02-10}
\organisation{AMSCC}
\difficulte{3}

\contenu{
    \texte{Calculer les intégrales suivantes. On pourra reconnaître des dérivées de fonctions composées :}
    \begin{enumerate}
        \item \question{\( I_1 = \int_{0}^{\frac{\pi}{3}} (1 - \cos(3x)) dx \)}
        \item \question{\( I_2 = \int_{0}^{\sqrt{\pi}} 2x \sin(x^2) dx \)}
        \item \question{\( I_3 = \int_{1}^{2} \frac{1}{x} (\ln x)^{\frac{1}{2}} dx \)}
    \end{enumerate}
}
