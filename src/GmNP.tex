\uuid{GmNP}
\chapitre{Probabilité continue}
\sousChapitre{Convergence en loi}
\titre{Convergence en loi}
\theme{fonction caractéristique, convergence en loi}
\auteur{}
\datecreate{2022-12-14}
\organisation{AMSCC}
\contenu{

\texte{ Soit $\lambda >0$. Pour tout $n \geq \lambda$, on définit une suite $\left(X_k^n\right)_{k \in \N}$une suite de variables aléatoires indépendantes suivant une loi de Bernoulli de paramètre $p_n = \frac{\lambda}{n}$. On considère alors la variable aléatoire :
$$N_n = \frac{1}{n} \inf\{ k \in N \, , \, X_k^n = 1\}$$
}
\begin{enumerate}
	\item \question{ Soit un entier $n \geq \lambda$. Justifier que la variable $nN_n$ suit une loi géométrique dont on précisera le paramètre. }
	\item \question{ Déterminer la fonction caractéristique de la variable aléatoire $N_n$. }
	\item \question{ En déduire que la suite de variables aléatoires $(N_n)$ converge en loi vers une loi usuelle que l'on précisera. }
\end{enumerate}
}
