\uuid{qAuO}
\chapitre{Statistique}
\niveau{L2}
\module{Probabilité et statistique}
\sousChapitre{Tests d'hypothèses, intervalle de confiance}
\titre{Comparaison d'homogénéité des évaluations}
\theme{tests d'hypothèses, test de Fisher, comparaison de variances}
\auteur{Maxime Nguyen}
\datecreate{2022-10-19}

\organisation{AMSCC}
\difficulte{3}
\contenu{
	\texte{ 
		Deux départements Mesures Physiques d'IUT utilisent le même référentiel de compétences 
		pour évaluer leurs étudiants en statistiques. On souhaite comparer l'homogénéité 
		de leur système de notation.
		
		On suppose que les notes dans chaque IUT suivent une loi normale. 
		Pour effectuer cette comparaison, on prélève :
		\begin{itemize}
			\item un échantillon de $n_1 = 10$ notes d'étudiants de l'IUT 1, 
			qui donne un écart-type empirique $s_1 = 5$ points
			\item un échantillon de $n_2 = 25$ notes d'étudiants de l'IUT 2, 
			qui donne un écart-type empirique $s_2 = 3$ points
		\end{itemize}
		
		\textbf{Contexte :} Une différence significative d'écart-type pourrait indiquer :
		\begin{itemize}
			\item des critères d'évaluation plus ou moins stricts
			\item une hétérogénéité différente dans les groupes d'étudiants
			\item des méthodes pédagogiques ayant des impacts différents
		\end{itemize}
	}
	
	\begin{enumerate}
		\item \question{ 
			Avant de réaliser le test, calculer le coefficient de variation pour chaque IUT 
			si on vous donne les moyennes suivantes : $\bar{x}_1 = 12.5$ et $\bar{x}_2 = 13.2$.
			
			Rappel : le coefficient de variation est $CV = \frac{s}{\bar{x}} \times 100\%$. 
			Que peut-on en conclure intuitivement ?
		}
		
		\reponse{
			\textbf{Calcul des coefficients de variation :}
			
			Pour l'IUT 1 :
			$$CV_1 = \frac{s_1}{\bar{x}_1} \times 100 = \frac{5}{12.5} \times 100 = 40\%$$
			
			Pour l'IUT 2 :
			$$CV_2 = \frac{s_2}{\bar{x}_2} \times 100 = \frac{3}{13.2} \times 100 \approx 22.7\%$$
			
			\textbf{Interprétation intuitive :}
			
			\begin{itemize}
				\item L'IUT 1 présente une dispersion relative de 40\% autour de sa moyenne, 
				contre 22.7\% pour l'IUT 2
				\item L'IUT 2 semble avoir une notation plus homogène (écarts plus faibles 
				par rapport à la moyenne)
				\item L'IUT 1 présente une plus grande variabilité des notes, 
				ce qui pourrait indiquer soit une plus grande hétérogénéité des étudiants, 
				soit des critères de notation plus différenciants
			\end{itemize}
			
			Toutefois, cette observation descriptive ne suffit pas : il faut un test statistique 
			pour déterminer si cette différence est significative ou due au hasard de l'échantillonnage.
		}
		
		\indication{
			Avec un tableur : \texttt{=(5/12.5)*100} et \texttt{=(3/13.2)*100}
		}
		
		\item \question{ 
			On souhaite tester si les deux IUT ont la même dispersion de notes. 
			\begin{itemize}
				\item Formuler les hypothèses $H_0$ et $H_1$ de ce test
				\item Quelle statistique de test doit-on utiliser ? Préciser sa loi sous $H_0$
				\item Ce test nécessite-t-il des conditions particulières ?
			\end{itemize}
		}
		
		\reponse{
			\textbf{Hypothèses du test :}
			
			On teste l'égalité des variances (ou des écarts-types, ce qui est équivalent) :
			$$H_0 : \sigma_1^2 = \sigma_2^2 \quad \text{ou de manière équivalente} \quad H_0 : \frac{\sigma_1^2}{\sigma_2^2} = 1$$
			$$H_1 : \sigma_1^2 \neq \sigma_2^2 \quad \text{ou} \quad H_1 : \frac{\sigma_1^2}{\sigma_2^2} \neq 1$$
			
			Il s'agit d'un test bilatéral car on ne privilégie aucune direction a priori.
			
			\textbf{Statistique de test (test de Fisher) :}
			
			On utilise le rapport des variances empiriques :
			$$F = \frac{s_1^2}{s_2^2}$$
			
			Par convention, on place la plus grande variance au numérateur pour avoir $F \geq 1$.
			
			Sous $H_0$ (si $\sigma_1^2 = \sigma_2^2$), cette statistique suit une loi de Fisher :
			$$F \sim \mathcal{F}(n_1-1, n_2-1) = \mathcal{F}(9, 24)$$
			
			\textbf{Conditions d'application :}
			
			\begin{enumerate}
				\item Les deux échantillons sont indépendants ✓
				\item Les notes dans chaque IUT suivent une loi normale (hypothèse de l'énoncé) ✓
				\item Les échantillons sont aléatoires simples
			\end{enumerate}
			
			\textbf{Attention :} Le test de Fisher est sensible à la non-normalité. 
			Si les distributions ne sont pas normales, ce test peut donner des résultats incorrects.
		}
		
		\item \question{ 
			Au seuil de signification $\alpha = 5\%$, peut-on considérer que les enseignants 
			des deux IUT ont la même manière de noter les étudiants ?
		}
		
		\reponse{
			\textbf{Calcul de la statistique de test :}
			
			On calcule le rapport des variances (la plus grande au numérateur) :
			$$F_{\text{obs}} = \frac{s_1^2}{s_2^2} = \frac{5^2}{3^2} = \frac{25}{9} \approx 2.778$$
			
			\textbf{Détermination de la région critique :}
			
			Pour un test bilatéral au seuil $\alpha = 5\%$, on rejette $H_0$ si :
			$$F_{\text{obs}} > F_{1-\alpha/2; n_1-1, n_2-1} = F_{0.975; 9, 24}$$
			ou si
			$$F_{\text{obs}} < F_{\alpha/2; n_1-1, n_2-1} = F_{0.025; 9, 24}$$
			
			En pratique, comme on a placé la plus grande variance au numérateur ($F > 1$), 
			on compare uniquement avec la valeur critique supérieure.
			
			D'après les tables de Fisher (ou avec un tableur) :
			$$F_{0.975; 9, 24} \approx 2.90$$
			
			Pour la borne inférieure (si nécessaire) :
			$$F_{0.025; 9, 24} = \frac{1}{F_{0.975; 24, 9}} \approx \frac{1}{3.61} \approx 0.277$$
			
			\textbf{Décision :}
			
			$F_{\text{obs}} = 2.778 < 2.90$ : on se situe dans la région d'acceptation.
			
			\textbf{On ne rejette pas $H_0$ au seuil $\alpha = 5\%$.}
			
			\textbf{Calcul de la $p$-valeur :}
			
			Pour un test bilatéral, avec $F = 2.778$ :
			$$p\text{-valeur} = 2 \times \mathbb{P}(F_{9,24} > 2.778) 
			= 2 \times (1 - \mathbb{P}(F_{9,24} \leq 2.778))$$
			
			Avec un tableur : $p$-valeur $\approx 0.053$ (légèrement supérieure à 5\%).
			
			\textbf{Conclusion :}
			
			Au niveau de confiance de 95\%, on ne peut pas conclure à une différence 
			significative entre les écarts-types des deux IUT. 
			
			Les données ne permettent pas d'affirmer que les enseignants ont des manières 
			de noter différentes en termes d'homogénéité.
			
			\textbf{Remarque importante :} La $p$-valeur étant très proche de 5\% (environ 5.3\%), 
			le résultat est « limite ». Avec un échantillon plus grand, on pourrait peut-être 
			détecter une différence significative. C'est ce qu'on appelle un résultat « marginalement significatif ».
		}
		
		\indication{
			Avec un tableur :
			\begin{itemize}
				\item Statistique $F$ : \texttt{=25/9} donne $2.778$
				\item Valeur critique : \texttt{=LOI.F.INVERSE(0.025; 9; 24)} donne $\approx 2.90$
				\item $p$-valeur : \texttt{=2*(1-LOI.F(2.778; 9; 24; VRAI))} donne $\approx 0.053$
				\item Ou directement : \texttt{=LOI.F.BILATERALE(2.778; 9; 24)}
			\end{itemize}
		}
		
		\item \question{
			Construire un intervalle de confiance à 95\% pour le rapport $\frac{\sigma_1^2}{\sigma_2^2}$. 
			Que peut-on en déduire concernant la comparaison des variances ?
		}
		
		\reponse{
			\textbf{Construction de l'intervalle de confiance :}
			
			Un intervalle de confiance à $1-\alpha = 95\%$ pour le rapport des variances est donné par :
			$$IC_{95\%}\left(\frac{\sigma_1^2}{\sigma_2^2}\right) 
			= \left[\frac{s_1^2/s_2^2}{F_{1-\alpha/2; n_1-1, n_2-1}} \,;\, 
			\frac{s_1^2/s_2^2}{F_{\alpha/2; n_1-1, n_2-1}}\right]$$
			
			$$= \left[\frac{F_{\text{obs}}}{F_{0.975; 9, 24}} \,;\, 
			\frac{F_{\text{obs}}}{F_{0.025; 9, 24}}\right]$$
			
			Avec les valeurs numériques :
			\begin{itemize}
				\item $F_{\text{obs}} = 2.778$
				\item $F_{0.975; 9, 24} = 2.90$
				\item $F_{0.025; 9, 24} = 0.277$
			\end{itemize}
			
			$$IC_{95\%}\left(\frac{\sigma_1^2}{\sigma_2^2}\right) 
			= \left[\frac{2.778}{2.90} \,;\, \frac{2.778}{0.277}\right]$$
			
			$$= [0.958 \,;\, 10.03]$$
			
			\textbf{Interprétation :}
			
			Avec 95\% de confiance, le rapport des variances $\frac{\sigma_1^2}{\sigma_2^2}$ 
			se situe entre 0.958 et 10.03.
			
			\begin{itemize}
				\item Cet intervalle contient la valeur 1 (test de $H_0 : \frac{\sigma_1^2}{\sigma_2^2} = 1$)
				\item Cela confirme qu'on ne peut pas rejeter l'hypothèse d'égalité des variances
				\item L'intervalle est très large, ce qui reflète l'incertitude due aux petits échantillons
				\item La variance de l'IUT 1 pourrait être jusqu'à 10 fois plus grande que celle de l'IUT 2, 
				ou légèrement plus petite (facteur 0.96)
			\end{itemize}
			
			\textbf{Conclusion pratique :}
			
			La grande largeur de l'intervalle suggère qu'il faudrait des échantillons plus grands 
			pour conclure avec précision sur la comparaison des variances.
		}
		
		\indication{
			Avec un tableur :
			\begin{itemize}
				\item Borne inf : \texttt{=2.778/LOI.F.INVERSE(0.025; 9; 24)}
				\item Borne sup : \texttt{=2.778/LOI.F.INVERSE(0.975; 9; 24)}
			\end{itemize}
		}
		
		\item \question{
			\textbf{[Question de réflexion]}
			
			\begin{itemize}
				\item Que se passerait-il si on augmentait la taille des échantillons 
				(par exemple $n_1 = 30$ et $n_2 = 50$) tout en conservant les mêmes écarts-types ?
				\item Quelle serait la taille d'échantillon minimale nécessaire pour l'IUT 1 
				(en gardant $n_2 = 25$ et les mêmes écarts-types) pour pouvoir rejeter $H_0$ 
				au seuil de 5\% ?
				\item Dans le contexte pédagogique, quelles pourraient être les causes 
				d'une différence d'écart-type si elle était avérée ?
			\end{itemize}
		}
		
		\reponse{
			\textbf{1. Impact de l'augmentation des échantillons :}
			
			Si $n_1 = 30$ et $n_2 = 50$ avec les mêmes écarts-types :
			\begin{itemize}
				\item La statistique $F$ reste la même : $F = 2.778$
				\item Mais la distribution de référence change : $\mathcal{F}(29, 49)$
				\item La valeur critique diminue : $F_{0.975; 29, 49} \approx 1.87$ 
				(au lieu de 2.90)
				\item Avec $F_{\text{obs}} = 2.778 > 1.87$, on rejetterait $H_0$
				\item La $p$-valeur diminuerait significativement (environ 0.003)
			\end{itemize}
			
			\textbf{Conclusion :} Des échantillons plus grands permettent de détecter 
			des différences plus faibles avec plus de confiance.
			
			\textbf{2. Taille minimale pour $n_1$ :}
			
			On cherche $n_1$ tel que $F_{\text{obs}} = 2.778 > F_{0.975; n_1-1, 24}$ 
			au seuil $\alpha = 5\%$.
			
			Par tâtonnements (ou résolution numérique) :
			\begin{itemize}
				\item Pour $n_1 = 11$ : $F_{0.975; 10, 24} \approx 2.74$ $\Rightarrow$ rejet
				\item Pour $n_1 = 10$ : $F_{0.975; 9, 24} \approx 2.90$ $\Rightarrow$ non-rejet
			\end{itemize}
			
			Il faudrait au moins $n_1 = 11$ observations pour l'IUT 1.
			
			\textbf{3. Causes possibles d'une différence d'écart-type :}
			
			Si les écarts-types étaient significativement différents, cela pourrait s'expliquer par :
			
			\textit{Côté étudiants :}
			\begin{itemize}
				\item Hétérogénéité différente des niveaux (recrutement, prérequis)
				\item Motivation et implication variables
				\item Taille des groupes différente (petits groupes vs grands amphithéâtres)
			\end{itemize}
			
			\textit{Côté enseignement :}
			\begin{itemize}
				\item Critères d'évaluation plus ou moins stricts/différenciants
				\item Barèmes différents (notation sur 20 vs notation plus « resserrée »)
				\item Type d'épreuves (QCM vs exercices ouverts)
				\item Pédagogie différenciée vs enseignement uniforme
			\end{itemize}
			
			\textit{Côté organisationnel :}
			\begin{itemize}
				\item Nombre d'évaluations différent (moyennes de plusieurs notes vs note unique)
				\item Contrôle continu vs examen terminal
				\item Politique de rattrapage/seconde chance
			\end{itemize}
			
			Une dispersion plus forte n'est ni bonne ni mauvaise en soi : 
			elle doit être cohérente avec les objectifs pédagogiques.
		}
		
		\indication{
			Pour la question 1, calculer avec \texttt{=LOI.F.INVERSE(0.025; 29; 49)} 
			pour voir la nouvelle valeur critique.
		}
	\end{enumerate}
}