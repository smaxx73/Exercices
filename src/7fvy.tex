\uuid{7fvy}
\exo7id{6952}
\auteur{ruette}
\organisation{exo7}
\datecreate{2013-01-24}
\isIndication{false}
\isCorrection{true}
\chapitre{Variance, covariance, fonction génératrice}
\sousChapitre{Variance, covariance, fonction génératrice}

\contenu{
\texte{

}
\begin{enumerate}
    \item \question{Soit $X$ une variable aléatoire de loi exponentielle $\mathcal{E}(\lambda)$, 
$\lambda>0$. Calculer l'espérance de $X$.}
\reponse{$E(X)=\displaystyle \int_0^{+\infty}x\lambda e^{-\lambda x}dx=\frac{1}{\lambda}$ par intégration par parties.}
    \item \question{Soit $X$ une variable aléatoire de loi de Poisson $\mathcal{P}(\lambda)$, $\lambda>0$. Calculer
la variance de $X$.}
\reponse{$E(X)=\displaystyle\sum_{k\geq 0} e^{-\lambda}\frac{\lambda^k}{k!}k
=\lambda e^{-\lambda}\sum_{k\geq 1}\frac{\lambda^{k-1}}{(k-1)!}=
\lambda e^{-\lambda}e^{\lambda}=\lambda$.

$E(X^2)=\displaystyle\sum_{k\geq 0} e^{-\lambda}\frac{\lambda^k}{k!}k^2
=\lambda e^{-\lambda}\sum_{k\geq 1}\frac{\lambda^{k-1}}{(k-1)!}k$.
On écrit $k=(k-1)+1$ :\\
$\displaystyle E(X^2)=\lambda e^{-\lambda}\left(\lambda\sum_{k\geq 2}
\frac{\lambda^{k-2}}{(k-2)!}+\sum_{k\geq 2}\frac{\lambda^{k-1}}{(k-1)!}+1
\right)=\lambda e^{-\lambda}(\lambda e^{\lambda}+e^{\lambda})=\lambda^2
+\lambda.$

$\text{Var}(X)=E(X^2)-E(X)^2$, donc $\text{Var}(X)=\lambda$.}
\end{enumerate}
}
