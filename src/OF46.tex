\uuid{OF46}
\exo7id{903}
\auteur{cousquer}
\organisation{exo7}
\datecreate{2003-10-01}
\isIndication{false}
\isCorrection{false}
\chapitre{Espace vectoriel}
\sousChapitre{Système de vecteurs}

\contenu{
\texte{
Dans l'espace $\mathbb{R}^4$, on se donne cinq vecteurs~:
$V_1=(1,1,1,1)$, $V_2=(1,2,3,4)$, $V_3=(3,1,4,2)$, $V_4=(10,4,13,7)$,
$V_5=(1,7,8,14)$.
À quelle(s) condition(s) un vecteur $B=(b_1,b_2,b_3,b_4)$
appartient-il au sous-espace engendré par les vecteurs $V_1$, $V_2$, 
$V_3$, $V_4$, $V_5$~? Définir ce sous-espace par une ou des équations.
}
}
