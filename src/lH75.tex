Voici le code LaTeX complet, intégrant la structure de votre document et la méthode analytique (Hessienne) avec des questions progressives.

\uuid{lH75}
\chapitre{Dérivabilité des fonctions réelles}
\niveau{L1}
\module{Analyse}
\sousChapitre{Applications}
\titre{Extremum d'un polynôme}
\theme{calcul différentiel, optimisation}
\auteur{Maxime NGUYEN}
\datecreate{2023-03-20}
\organisation{AMSCC}
\difficulte{1}
\contenu{
	
	\texte{ Soit une fonction $f$ définie sur $\mathbb{R}^2$ par : 
		$$f \colon (x,y) \mapsto x^2+xy+y^2+2x+3y$$ }
	
	\begin{enumerate}
		\item \question{ Justifier que la fonction $f$ est de classe $\mathcal{C}^2$ sur $\mathbb{R}^2$ et calculer ses dérivées partielles d'ordre 1. }
		\reponse{ La fonction $f$ est une fonction polynomiale, elle est donc de classe $\mathcal{C}^\infty$ (et en particulier $\mathcal{C}^2$) sur $\mathbb{R}^2$.
			On calcule le gradient de $f$ :
			$$ \frac{\partial f}{\partial x}(x,y) = 2x+y+2 \quad \text{et} \quad \frac{\partial f}{\partial y}(x,y) = x+2y+3 $$ }
		
		\item \question{ Déterminer l'unique point critique $(x_0, y_0)$ de $f$ en résolvant le système $\nabla f(x,y) = \vec{0}$. }
		\reponse{ On cherche $(x,y)$ tels que :
			$$ \begin{cases} 2x+y+2 = 0 \\ x+2y+3 = 0 \end{cases} \iff \begin{cases} y = -2x-2 \\ x+2(-2x-2)+3 = 0 \end{cases} $$
			La deuxième ligne donne $-3x - 1 = 0 \implies x = -1/3$.
			On remplace $x$ dans la première ligne : $y = -2(-1/3) - 2 = 2/3 - 6/3 = -4/3$.
			Le seul point critique est donc $A = \left(-\frac{1}{3},-\frac{4}{3}\right)$. }
		
		\item \question{ Calculer les dérivées partielles d'ordre 2 de $f$ et écrire la matrice Hessienne $H_f(x,y)$. }
		\reponse{ On calcule les dérivées secondes :
			$$ r = \frac{\partial^2 f}{\partial x^2} = 2, \quad t = \frac{\partial^2 f}{\partial y^2} = 2, \quad s = \frac{\partial^2 f}{\partial x \partial y} = 1 $$
			La matrice Hessienne est constante sur $\mathbb{R}^2$ :
			$$ H_f(x,y) = \begin{pmatrix} 2 & 1 \\ 1 & 2 \end{pmatrix} $$ }
		
		\item \question{ Déterminer la nature de ce point critique (minimum local, maximum local ou point selle). Calculer la valeur de l'extremum. }
		\reponse{ On calcule le déterminant de la matrice Hessienne : $\det(H_f) = rt - s^2 = 2 \times 2 - 1^2 = 3$.
			Comme $\det(H_f) > 0$ et que le terme $r = 2 > 0$, il s'agit d'un \textbf{minimum local}.
			La valeur de ce minimum est $f\left(-\frac{1}{3},-\frac{4}{3}\right) = -\frac{7}{3}$. }
		
		\item \question{ La fonction $f$ admet-elle des extremums globaux sur $\mathbb{R}^2$ ? }
		\indication{
		Pour le minimum, vérifier que pour tout $(x,y) \in \mathbb{R}^2$, on peut écrire $f(x,y)$ sous la forme :
		$$f(x,y) = \left(x + \frac{y}{2} + 1\right)^2 + \frac{3}{4}\left(y + \frac{4}{3}\right)^2 - \frac{7}{3}$$
		
		Pour le maximum, étudier par exemple la limite de $f(x,0)$ quand $x$ tend vers l'infini. 
		}
		\reponse{ 
			\textbf{Minimum global :} On constate en écrivant $f(x,y)$ sous forme canonique que $\left(x + \frac{y}{2} + 1\right)^2 + \frac{3}{4}\left(y + \frac{4}{3}\right)^2 \geq 0$ donc pour tout $(x,y) \in \R^2$, $f(x,y) \geq - \frac{7}{3}$. D'après la question précédente, on en déduit que $- \frac{7}{3}$ est bien le minimum global de la fonction et il est atteint en $\left(-\frac{1}{3},-\frac{4}{3}\right)$. 
			
			\textbf{Maximum global :} On étudie le comportement de la fonction lorsque $x \to +\infty$ avec $y=0$ :
			$$ f(x,0) = x^2 + 2x \xrightarrow[x \to +\infty]{} +\infty $$
			La fonction n'est pas majorée, elle n'admet donc pas de maximum global. }
	\end{enumerate}
}

