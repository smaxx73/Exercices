\uuid{tPeW}
\titre{Modélisation du nombre de connexions simultanées}
\niveau{L2}
\module{Probabilité et statistique}
\chapitre{Probabilité discrète}
\sousChapitre{Variable aléatoire discrète}
\theme{Loi binomiale, inégalité de Bienaymé-Tchebychev, approximation normale}
\auteur{Erwan Hillion}
\datecreate{2025-11-12}
\organisation{AMSCC}
\difficulte{3}

\contenu{
	\texte{Un fournisseur d'accès à Internet met en place un point local d'accès, qui dessert \(5000\) abonnés. Chaque abonné a une probabilité égale à \(20\%\) d'être connecté à 16h. Les comportements des abonnés sont supposés indépendants les uns des autres.}
	
	\begin{enumerate}
		\item \question{On note \( X \) la variable aléatoire égale au nombre d'abonnés connectés à 16h. Quelle est la loi de \( X \) ? Quelle est son espérance, son écart-type ?}
		
		\reponse{
			L'expérience consiste à répéter \( n=5000 \) fois de manière indépendante une épreuve de Bernoulli (l'abonné est connecté ou non), dont la probabilité de succès (être connecté) est \( p=0.2 \).
			La variable aléatoire \( X \) suit donc une loi binomiale de paramètres \( n=5000 \) et \( p=0.2 \). On note \( X \sim \mathcal{B}(5000, 0.2) \).
			
			\begin{itemize}
				\item \textbf{Espérance :} L'espérance de \( X \) est donnée par :
				\[ \mathbb{E}(X) = n \times p = 5000 \times 0.2 = 1000 \]
				En moyenne, 1000 abonnés sont connectés à 16h.
				
				\item \textbf{Variance et écart-type :} La variance de \( X \) est :
				\[ \mathbb{V}(X) = n \times p \times (1-p) = 5000 \times 0.2 \times 0.8 = 1000 \times 0.8 = 800 \]
				L'écart-type \( \sigma_X \) est la racine carrée de la variance :
				\[ \sigma_X = \sqrt{\mathbb{V}(X)} = \sqrt{800} = \sqrt{400 \times 2} = 20\sqrt{2} \]
			\end{itemize}
		}
		
		\item \question{À l'aide de l'inégalité de Bienaymé-Tchebychev, montrer que
			$$\mathbb{P}(X \in [960,1040]) \ge \frac{1}{2}.$$
			\textit{Note : L'énoncé original contenait une coquille probable (\(\le\)), qui est corrigée ici pour correspondre à l'application correcte de l'inégalité.}}
		
		\reponse{
			L'inégalité de Bienaymé-Tchebychev stipule que pour toute variable aléatoire \( X \) d'espérance \( \mu \) et de variance \( \sigma^2 \), et pour tout réel \( a > 0 \), on a :
			\[ \mathbb{P}(|X - \mu| \ge a) \le \frac{\sigma^2}{a^2} \]
			L'événement complémentaire est \( |X - \mu| < a \), donc on a aussi :
			\[ \mathbb{P}(|X - \mu| < a) \ge 1 - \frac{\sigma^2}{a^2} \]
			Ici, \( \mu = 1000 \) et \( \sigma^2 = 800 \). L'intervalle \( [960, 1040] \) correspond à \( |X - 1000| \le 40 \). On choisit donc \( a=40 \).
			
			Appliquons l'inégalité au complémentaire :
			\[ \mathbb{P}(|X - 1000| \ge 40) \le \frac{800}{40^2} = \frac{800}{1600} = \frac{1}{2} \]
			Cette inégalité montre que la probabilité que \(X\) s'écarte de son espérance de plus de 40 unités est au plus de \(1/2\).
			
			Pour la probabilité que \(X\) soit dans l'intervalle, on a :
			\[ \mathbb{P}(X \in [960,1040]) = \mathbb{P}(|X - 1000| \le 40) = 1 - \mathbb{P}(|X - 1000| > 40) \]
			Puisque \( |X - 1000| > 40 \) implique \( |X-1000| \ge 40 \) (car X est un entier), on a \( \mathbb{P}(|X - 1000| > 40) \le \mathbb{P}(|X - 1000| \ge 40) \le \frac{1}{2} \).
			Ainsi :
			\[ \mathbb{P}(X \in [960,1040]) \ge 1 - \frac{1}{2} = \frac{1}{2} \]
		}
		
		\item \question{On pose :
			$$Y= \frac{X-1000}{\sqrt{800}}.$$
			Par quelle loi peut-on approcher la loi de \( Y \) ? Le justifier précisément.}
		
		\reponse{
			La variable aléatoire \( Y \) est la variable centrée-réduite associée à \( X \).
			Comme \( X \) suit une loi binomiale \( \mathcal{B}(n,p) \), on peut utiliser le théorème de Moivre-Laplace (un cas particulier du théorème central limite). Ce théorème stipule que si les conditions d'approximation sont remplies, la loi de la variable centrée-réduite converge vers la loi normale centrée-réduite \( \mathcal{N}(0,1) \).
			
			Les conditions de validité de cette approximation sont :
			\begin{itemize}
				\item \( n \) est grand (en pratique \( n \ge 30 \)). Ici \( n = 5000 \), donc c'est vérifié.
				\item \( np \ge 5 \) et \( n(1-p) \ge 5 \). Ici \( np = 1000 \) et \( n(1-p) = 4000 \). Ces conditions sont largement satisfaites.
			\end{itemize}
			Les conditions étant remplies, on peut approcher la loi de \( Y = \frac{X-1000}{\sqrt{800}} \) par la \textbf{loi normale centrée-réduite} \( \mathcal{N}(0,1) \).
		}
		
		\item \question{Le fournisseur d'accès souhaite savoir combien de connexions simultanées le point d'accès doit pouvoir gérer pour que sa probabilité d'être saturé à un instant donné soit inférieure à \(2.5\%\). En utilisant l'approximation précédente, proposer une valeur approchée de ce nombre de connexions.
			On pourra utiliser l'approximation \(\sqrt{2} \approx 1.41\).}
		
		\reponse{
			On cherche le nombre de connexions \( C \) tel que la probabilité de saturation, \( \mathbb{P}(X > C) \), soit inférieure à \( 0.025 \).
			\[ \mathbb{P}(X > C) \le 0.025 \]
			On utilise l'approximation de \( X \) par une loi normale \( \mathcal{N}(1000, 800) \). Pour utiliser la table de la loi normale centrée-réduite, on standardise l'inégalité :
			\[ \mathbb{P}\left(\frac{X - 1000}{\sqrt{800}} > \frac{C - 1000}{\sqrt{800}}\right) \le 0.025 \]
			En posant \( Y = \frac{X-1000}{\sqrt{800}} \), où \( Y \sim \mathcal{N}(0,1) \), on a :
			\[ \mathbb{P}\left(Y > \frac{C - 1000}{\sqrt{800}}\right) \le 0.025 \]
			Soit \( z_c = \frac{C - 1000}{\sqrt{800}} \). La condition s'écrit \( 1 - \Phi(z_c) \le 0.025 \), où \( \Phi \) est la fonction de répartition de la loi \( \mathcal{N}(0,1) \).
			Cela est équivalent à \( \Phi(z_c) \ge 0.975 \).
			
			D'après la table de la loi normale, la valeur \( z \) pour laquelle \( \Phi(z) = 0.975 \) est \( z \approx 1.96 \).
			On doit donc avoir :
			\[ \frac{C - 1000}{\sqrt{800}} \ge 1.96 \]
			\[ C \ge 1000 + 1.96 \times \sqrt{800} \]
			Calculons la valeur numérique. On sait que \( \sqrt{800} = 20\sqrt{2} \). En utilisant l'approximation \( \sqrt{2} \approx 1.41 \), on obtient :
			\[ \sqrt{800} \approx 20 \times 1.41 = 28.2 \]
			Donc :
			\[ C \ge 1000 + 1.96 \times 28.2 \approx 1000 + 55.272 = 1055.272 \]
			Comme le nombre de connexions \( C \) doit être un entier, on doit choisir la plus petite valeur entière satisfaisant cette condition, soit \( C = 1056 \).
			Le point d'accès doit donc pouvoir gérer au moins \textbf{1056 connexions} simultanées.
		}
	\end{enumerate}
}
