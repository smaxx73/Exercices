\uuid{Y8mj}
\chapitre{Résolution de systèmes linéaires : méthode itérative}
\sousChapitre{Résolution de systèmes linéaires : méthode itérative}
\titre{Méthodes itératives}
\theme{analyse numérique}
\auteur{}
\datecreate{2023-04-21}
\organisation{AMSCC}
\contenu{

\texte{ 	Soit $a \in \mathbb{R}$ et soit la matrice $$A = \begin{pmatrix} 1 & a & a \\ a & 1 & a \\ a & a & 1 \end{pmatrix}$$ 

On admet que le polynôme caractéristique de $A$ est $\chi_A(X) = -(X-1+a)^2(X-1-2a)$.
}

\begin{enumerate}
	\item\question{  En déduire des conditions sur $a \in \mathbb{R}$ telles que la matrice $A$ soit symétrique définie positive. On se restreindra à ce cas par la suite. }
	\reponse{La matrice $A$ est définie positive si ses valeurs propres $1-a>0$ et $1+2a>0$, autrement dit si $-\frac{1}{2} < a < 1 $.}
	\item \question{ Déterminer le rayon spectral de la matrice $I_3 - A$ où $I_3$ est la matrice identité. }
	\reponse{ D'après ce qui précède, la matrice $I_3-A$ a deux valeurs propres : $-a$ et $2a$. Son rayon spectral est donc $2|a|$. }
	\item \question{ En déduire des conditions nécessaires et suffisantes sur  $a \in \mathbb{R}$ pour que la méthode de Jacobi converge vers une solution du système $Ax=b$. }
	\reponse{Pour que la méthode de Jacobi converge, il faut et il suffit que $\rho(D-A)<1$ où $D=I_3$ est la diagonale de $A$.
		La méthode converge si et seulement si $\rho(D-A) = 2|a| <1$ . On en déduit que la condition nécessaire et suffisante de convergence de la méthode est que $-\frac{1}{2} < a < \frac{1}{2}$. 	
	}
	\item \question{ Déterminer des conditions suffisantes sur  $a \in \mathbb{R}$ pour que la méthode de Gauss-Seidel converge vers une solution du système $Ax=b$. }
	\reponse{Pour que la méthode de Gauss converge,  il suffit que la matrice soit symétrique définie positive.
	}
%\item Que permet de calculer le programme suivant ?
%\begin{Piton}
%def fonction_mystere(A, b, x0):
%  n = len(A)
%  x = zeros(n)
%
%  for i in range(n):
%    sum_ax = 0
%    for j in range(n):
%      if i != j:
%        sum_ax += A[i, j] * x0[j]
%    x[i] = (b[i] - sum_ax) / A[i, i]
%  return x
%
%a = -1/3
%A = array([[1, a, a],
%[a, 1, a],
%[a, a, 1]])
%
%b = array([1, 2, 3])
%x0 = zeros(3)  
%x = fonction_mystere(A, b, x0)
%print(x)
%\end{Piton}
\end{enumerate}}
