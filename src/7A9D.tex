\uuid{7A9D}
\titre{Distribution de Gumbel et les fonctions de répartition}
\chapitre{Probabilité continue}
\niveau{L2}
\module{Probabilité et statistique}
\sousChapitre{Densité de probabilité}
\theme{Probabilités, Statistiques, Fonctions de répartition}
\auteur{}
\datecreate{2025-06-08}
\organisation{}
\difficulte{}
\contenu{
	\texte{
		Soient \( g \) et \( G \) deux fonctions définies pour tout \( x \in \mathbb{R} \) par :
		$$ G(x) = e^{-e^{-x}} \text{ et } g(x) = e^{-x}e^{-e^{-x}} = e^{-(x + e^{-x})}. $$
	}
	
\begin{enumerate}
	\item \question{Calculer les limites \( \lim\limits_{x \to -\infty} G(x) \), \( \lim\limits_{x \to +\infty} G(x) \). En déduire que \( g \) est bien une densité de probabilité sur \( \mathbb{R} \). On dira que \( g \), resp. \( G \) sont la densité et la fonction de répartition de la loi de Gumbel.}
	
	\reponse{On a \( \lim_{x \to -\infty} G(x) = e^{-\infty} = 0 \) et \( \lim_{x \to +\infty} G(x) = e^0 = 1 \). La fonction \(G\) est continue et dérivable sur \( \mathbb{R} \) et l'on vérifie que \( G'(x) = g(x) \). De plus, \( g(x) > 0 \) pour tout \(x \in \mathbb{R}\). L'intégrale de \( g \) sur \( \mathbb{R} \) vaut \( \int_{-\infty}^{+\infty} g(x)dx = [G(x)]_{-\infty}^{+\infty} = 1 - 0 = 1 \). Ainsi, \(g\) est bien une densité de probabilité.}
	
	\item \question{Soit \( X \) une variable aléatoire de loi exponentielle \( \mathcal{E}(1) \) de densité \(x \mapsto e^{-x}\mathbf{1}_{x \geq 0} \). Montrer que la variable aléatoire \( Z = -\ln(X) \) suit la loi de Gumbel.}
	
	\reponse{Soit \( F_Z \) la fonction de répartition de \( Z \). Pour tout \( z \in \mathbb{R} \), on a :
		\( F_Z(z) = P(Z \leq z) = P(-\ln(X) \leq z) = P(\ln(X) \geq -z) = P(X \geq e^{-z}) \).
		Comme \( X \sim \mathcal{E}(1) \), sa fonction de survie est \( P(X \geq y) = e^{-y} \) pour \( y \geq 0 \).
		Puisque \( e^{-z} > 0 \), il vient \( F_Z(z) = e^{-e^{-z}} = G(z) \). Donc, \( Z \) suit la loi de Gumbel.}
	
	\item \texte{Soient \( X_1, \ldots, X_n \) des v.a. indépendantes, toutes de loi exponentielle \( \mathcal{E}(1) \). On définit pour tout entier $n \geq 1$ les variables aléatoires :  $$M_n = \max(X_1, \ldots, X_n) \text{ et } Z_n = M_n - \ln(n).$$}
	
	\question{ Montrer que les variables aléatoires \( M_n \) et \( Z_n \) admettent respectivement les fonctions de répartition :
		$$F_{M_n} \colon x \mapsto (1 - e^{-x})^n \mathbf{1}_{x \geq 0}$$
		$$F_{Z_n} \colon x \mapsto \left(1 - \frac{e^{-x}}{n}\right)^n \mathbf{1}_{x \geq -\ln(n)}.$$}
	
	\reponse{Pour \( M_n \), \( F_{M_n}(x) = P(\max X_i \leq x) = P(\forall i, X_i \leq x) = \prod_{i=1}^n P(X_i \leq x) = (F_X(x))^n \). Avec \( F_X(x) = (1-e^{-x})1_{x\geq 0} \), on obtient la formule voulue.
		Pour \( Z_n \), \( F_{Z_n}(x) = P(M_n - \ln n \leq x) = P(M_n \leq x + \ln n) = F_{M_n}(x + \ln n) \). En substituant, on obtient \( F_{Z_n}(x) = (1 - e^{-(x+\ln n)})^n = (1 - \frac{e^{-x}}{n})^n \), pour \( x+\ln n \geq 0 \), soit \( x \geq -\ln n \).}
	
	\item \question{Pour tout entier $n \geq 1$, déterminer des fonctions densité \( f_n \) et \( g_n \) respectivement des variables aléatoires $M_n$ et $Z_n$.}
	
	\reponse{On dérive les fonctions de répartition.
		Pour \( M_n \), \( f_n(x) = F'_{M_n}(x) = n(1-e^{-x})^{n-1}(e^{-x})1_{x \geq 0} = n e^{-x}(1-e^{-x})^{n-1}1_{x \geq 0} \).
		Pour \( Z_n \), \( g_n(x) = F'_{Z_n}(x) = n(1-\frac{e^{-x}}{n})^{n-1}(\frac{e^{-x}}{n})1_{x \geq -\ln n} = e^{-x}(1-\frac{e^{-x}}{n})^{n-1}1_{x \geq -\ln n} \).}
	
	\item \question{Montrer que la suite de fonctions \( (g_n)_{n \geq 1} \) converge simplement vers la densité \( g \) de la loi de Gumbel.}
	
	\reponse{Pour \( x \) fixé, lorsque \( n \to \infty \), la condition \( x \geq -\ln n \) est vérifiée pour \(n\) assez grand. On utilise la limite connue \( \lim_{n\to\infty} (1+\frac{a}{n})^n = e^a \).
		On a \( g_n(x) = e^{-x} (1 - \frac{e^{-x}}{n})^{n-1} = e^{-x} \frac{(1 - \frac{e^{-x}}{n})^n}{(1 - \frac{e^{-x}}{n})} \).
		Lorsque \( n \to \infty \), le numérateur tend vers \( e^{-e^{-x}} \) et le dénominateur vers \( 1 \).
		Ainsi, \( \lim_{n\to\infty} g_n(x) = e^{-x} e^{-e^{-x}} = g(x) \).}
	
	\item \question{ Les suites de variables aléatoire $(M_n)$ et $(Z_n)$ convergent-elles en loi ?}
	
\end{enumerate}
}