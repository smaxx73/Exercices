\uuid{x78l}
\titre{Analyse de données pharmaceutiques avec tableur}
\niveau{L2} 
\module{Probabilités et Statistiques} 
\chapitre{Échantillonnage et estimation}
\sousChapitre{Intervalles de confiance, utilisation d'un tableur}
\theme{Statistiques descriptives, loi normale, intervalle de confiance de Student}
\auteur{AMSCC}
\datecreate{2025-12-05}
\organisation{AMSCC}
\difficulte{2}
\contenu{
	\texte{ 
		Un laboratoire pharmaceutique teste le temps d'efficacité (en heures) d'un nouveau médicament. Les durées mesurées sur \textbf{25 patients} sont fournies dans le fichier tableur ci-joint et reproduites ci-dessous :
		
		\begin{center}
			\begin{tabular}{ccccc}
				4.2 & 5.1 & 4.8 & 5.5 & 4.6 \\
				5.3 & 4.9 & 5.7 & 4.5 & 5.0 \\
				4.7 & 5.2 & 4.4 & 5.6 & 4.3 \\
				5.4 & 4.8 & 5.1 & 4.9 & 5.3 \\
				4.6 & 5.5 & 4.7 & 5.0 & 4.8
			\end{tabular}
		\end{center}
	}
	
	\subsection*{Partie A - Statistiques descriptives}
	
	\begin{enumerate}
		\item   \question{À l'aide du tableur, calculer la moyenne $\bar{x}$, la variance empirique corrigée $s^2$, l'écart-type $s$.}
		\indication{Saisir les données dans une colonne Excel (par exemple A1:A25) et utiliser les fonctions statistiques.}
		\reponse{
			\textbf{Saisie dans Excel :}
			\begin{itemize}
				\item Cellule B1 : \texttt{=MOYENNE(A1:A25)}
				\item Cellule B2 : \texttt{=VAR.S(A1:A25)}
				\item Cellule B3 : \texttt{=ECARTYPE.STANDARD(A1:A25)}
			\end{itemize}
			
			\textbf{Résultats numériques corrigés :}
			\begin{itemize}
				\item Somme des valeurs : $123.9$
				\item Moyenne : $\bar{x} = 4.956$ heures
				\item Variance : $s^2 \approx 0.1701$ heures²
				\item Écart-type : $s \approx 0.4124$ heures
			\end{itemize}
			
			\textbf{Interprétation :} Le temps moyen d'efficacité observé est de 4.956 heures, avec un écart-type d'environ 0.41 heure (soit environ 25 minutes).
		}
		
		\item   \question{Le laboratoire affirme que le temps moyen d'efficacité est de 5 heures.
			\begin{itemize}
				\item Calculer l'écart absolu : $|\bar{x} - 5|$
				\item Calculer l'écart relatif : $\dfrac{|\bar{x} - 5|}{5} \times 100\%$
				\item Commenter : l'écart observé vous semble-t-il significatif ?
		\end{itemize}}
		\indication{Pour l'écart relatif, utiliser \texttt{=ABS(B1-5)/5*100} dans Excel.}
		\reponse{
			\textbf{Calculs :}
			\begin{itemize}
				\item Écart absolu : $|\bar{x} - 5| = |4.956 - 5| = 0.044$ heures $\approx 2.6$ minutes
				\item Écart relatif : $\dfrac{0.044}{5} \times 100\% = 0.88\%$
			\end{itemize}
			
			\textbf{Commentaire :} L'écart relatif est extrêmement faible (moins de 1\%). Cela suggère fortement que la différence observée est due aux fluctuations d'échantillonnage et que la moyenne empirique est très proche de la valeur théorique annoncée.
		}
	\end{enumerate}
	
	\subsection*{Partie B - Modélisation par une loi normale}
	
	\texte{
		On suppose que le temps d'efficacité $X$ suit une loi normale $\mathcal{N}(\mu, \sigma^2)$ où $\mu$ et $\sigma$ sont estimés par $\bar{x} \approx 4.956$ et $s \approx 0.4124$.
	}
	
	\begin{enumerate}
		\item   \question{Avec le tableur, calculer les probabilités suivantes :
			\begin{itemize}
				\item $P(X \leq 4.5)$ ;
				\item $P(4.8 \leq X \leq 5.2)$.
			\end{itemize}
			Interprétation : Quel pourcentage de patients ont une efficacité entre 4.8 et 5.2 heures ?}
		\indication{Pour $P(a \leq X \leq b)$, calculer $P(X \leq b) - P(X \leq a)$.}
		\reponse{
			\textbf{Formules Excel :}
			\begin{itemize}
				\item Cellule B4 : \texttt{=LOI.NORMALE(4.5; B1; B3; VRAI)}
				\item Cellule B5 : \texttt{=LOI.NORMALE(5.2; B1; B3; VRAI) - LOI.NORMALE(4.8; B1; B3; VRAI)}
			\end{itemize}
			
			\textbf{Résultats :}
			\begin{itemize}
				\item $P(X \leq 4.5) \approx 0.1345$, soit environ 13.5\%
				\item $P(4.8 \leq X \leq 5.2) \approx 0.3707$, soit environ 37.1\%
			\end{itemize}
			
			\textbf{Interprétation :} Environ $37\%$ des patients ont un temps d'efficacité compris entre 4.8 et 5.2 heures selon ce modèle.
		}
		
		\item   \question{Déterminer le temps $t_{90}$ tel que $90\%$ des patients ont une efficacité inférieure à $t_{90}$ heures.}
		\indication{Il s'agit du quantile d'ordre 0.9 de la loi normale estimée.}
		\reponse{
			\textbf{Formule Excel :}
			
			Cellule B6 : \texttt{=LOI.NORMALE.INVERSE(0.9; B1; B3)}
			
			\textbf{Résultat :}
			$$t_{90} \approx 5.485 \text{ heures}$$
			
			\textbf{Interprétation :} $90\%$ des patients ont un temps d'efficacité inférieur à 5.485 heures (soit environ 5h 29min).
		}
	\end{enumerate}
	
	\subsection*{Partie C - Intervalle de confiance}
	
	\begin{enumerate}
		\item   \question{Déterminer une estimation du temps moyen $\mu$ à l'aide d'un intervalle de confiance à $95\%$.}
		\indication{On utilise la loi de Student à $n-1$ degrés de liberté.}
		\reponse{
			\textbf{Paramètres :}
			\begin{itemize}
				\item Taille $n=25$, degrés de liberté $ddl = 24$.
				\item Quantile de Student $t_{0.975}^{(24)} \approx 2.0639$ (via \texttt{=LOI.STUDENT.INVERSE.N(0.975; 24)}).
				\item Marge d'erreur : $ME = t \times \frac{s}{\sqrt{n}} = 2.0639 \times \frac{0.4124}{5} \approx 0.1702$.
			\end{itemize}
			
			\textbf{Calcul des bornes :}
			\begin{itemize}
				\item Borne inf : $4.956 - 0.1702 = 4.7858$
				\item Borne sup : $4.956 + 0.1702 = 5.1262$
			\end{itemize}
			
			\textbf{Intervalle de confiance à 95\% :}
			$$IC_{95\%} = [4.786 \; ; \; 5.126] \text{ heures}$$
		}
		
		\item   \question{Comment interpréter ce résultat quant à l'affirmation du laboratoire ?}
		\indication{Vérifier si $5 \in IC_{95\%}$.}

		\reponse{
	\textbf{Oui}, la valeur théorique 5 heures appartient bien à l'intervalle calculé $[4.786; 5.126]$.
	
	\textbf{Conclusion :} Au niveau de confiance de 95\%, les données observées sont \textbf{compatibles} avec l'affirmation du laboratoire selon laquelle le temps moyen d'efficacité est de 5 heures. L'écart observé (4.956 au lieu de 5) n'est pas statistiquement significatif.
}

		\item   \question{Si on voulait diviser par 2 la largeur de l'intervalle de confiance, quelle devrait être la nouvelle taille d'échantillon $n'$ ?}
		\indication{Écrire $\frac{1}{\sqrt{n'}} = \frac{1}{2} \times \frac{1}{\sqrt{n}}$ et résoudre.}
		\reponse{
	La largeur de l'intervalle est définie par $L = 2 \times ME \approx 2 \times t \times \dfrac{s}{\sqrt{n}}$.
	
	Si l'on considère que l'écart-type $s$ et le coefficient $t$ restent relativement stables, la largeur est inversement proportionnelle à $\sqrt{n}$. Pour diviser la largeur par 2, le terme $\sqrt{n}$ doit être multiplié par 2.
	
	$$\sqrt{n'} = 2 \times \sqrt{n} \iff n' = 4 \times n$$
	
	$$n' = 4 \times 25 = \boxed{100 \text{ patients}}$$
	
	\textbf{Conclusion :} Pour obtenir une précision deux fois plus fine (diviser l'intervalle par 2), il est nécessaire de quadrupler la taille de l'échantillon, soit interroger 100 patients.
}
\end{enumerate}
}