\uuid{fRkA}
\titre{Décomposition en éléments simples et intégration}
\niveau{L2}
\module{Analyse}
\chapitre{Intégration}
\sousChapitre{Intégration par parties}
\theme{fonction rationnelle, logarithme}
\auteur{Erwan L'Haridon}
\datecreate{2026-02-10}
\organisation{AMSCC}
\difficulte{3}

\contenu{
    \texte{On considère la fonction \( f(x) = \frac{1}{x(x + 1)} \).}
    \begin{enumerate}
        \item \question{Déterminer deux réels \( a \) et \( b \) tels que pour tout \( x \in \mathbb{R} \setminus \{0; -1\} \) on ait : \( f(x) = \frac{a}{x} + \frac{b}{x + 1} \).}
        \item \question{Déduire de la question précédente la valeur de l'intégrale \( J = \int_{1}^{2} f(x) dx \).}
        \item \question{En intégrant par parties, calculer l'intégrale \( I = \int_{1}^{2} \frac{\ln(1 + t)}{t^2} dt \).}
    \end{enumerate}
}
