\titre{Maximum de vraisemblance pour une loi géométrique}
\theme{statistiques}
\auteur{Maxime NGUYEN}
\organisation{AMSCC}
\contenu{

\texte{ 	On rappelle qu'une variable $X$ suit une loi géométrique de paramètre $p \in ]0;1[$ si $X$ est à valeurs dans $\N^*=\{1;2;3;...\}$ et si pour tout $k \in \N^*$, 
$$\PP(X=k)=p(1-p)^{k-1}$$

On cherche à estimer ce paramètre $p$ à partir d'un échantillon. }

\begin{enumerate}
	\item \texte{ On considère un échantillon $(X_1,X_2,X_3,X_4,X_5)$  ayant pour loi mère une loi géométrique de paramètre $p$ et on suppose que la suite  $(3;4;4;2;3)$ constitue une réalisation de cet échantillon. }
	\begin{enumerate}
		\item \question{ Exprimer la fonction de vraisemblance associée à cet échantillon. }
		\item \question{ En déduire une estimation de $p$ par la méthode du maximum de vraisemblance.  }
	\end{enumerate}
	
	\item \texte{ Afin de déterminer un estimateur de $p$, on considère maintenant un $n$-échantillon $(X_1,...,X_n)$ ayant pour loi mère une loi géométrique de paramètre $p$ et $n$ entiers non nuls $(x_1,...,x_n)$ constituant une réalisation de cet échantillon.  }	 
	\begin{enumerate}
		\item \question{ Exprimer la fonction de vraisemblance associée à cet échantillon. }
		\item \question{ En utilisant la méthode du maximum de vraisemblance, déterminer un estimateur du paramètre $p$. }
	\end{enumerate}
\end{enumerate}}
