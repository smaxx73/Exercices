\uuid{sgck}
\titre{Système linéaire paramétré}
\niveau{L1} 
\module{Algèbre} 
\chapitre{Systèmes linéaires}   
\sousChapitre{Systèmes paramétrés}
\theme{Étude selon un paramètre, déterminant}
\auteur{Maxime Nguyen (Corrigé)}
\datecreate{2026-01-13}
\organisation{AMSCC}
\difficulte{3}
\contenu{
	\texte{ 
		Soit $\lambda \in \mathbb{R}$ un paramètre réel. On considère le système linéaire suivant :
		\begin{equation*}
			(S_\lambda) : 
			\begin{cases}
				(\lambda - 1)x + 2z = 1 \\
				y = 2 \\
				x + \lambda z = 1
			\end{cases}
		\end{equation*}
	}
	\begin{enumerate}
		\item   \question{Écrire la matrice $A_\lambda$ et le vecteur $b$ associés au système $(S_\lambda)$.}
		\reponse{
			$$A_\lambda = \begin{pmatrix} 
				\lambda-1 & 0 & 2 \\ 
				0 & 1 & 0 \\ 
				1 & 0 & \lambda 
			\end{pmatrix}
			\quad \text{et} \quad
			b = \begin{pmatrix} 1 \\ 2 \\ 1 \end{pmatrix}$$
		}
		
		\item   \question{Calculer le déterminant $\det(A_\lambda)$ en fonction de $\lambda$ et déterminer ses racines.}
		\reponse{
			On développe par rapport à la deuxième ligne (qui ne contient qu'un seul coefficient non nul égal à 1) :
			$$\det(A_\lambda) = 1 \cdot \begin{vmatrix} \lambda-1 & 2 \\ 1 & \lambda \end{vmatrix} 
			= (\lambda-1)\lambda - 2 = \lambda^2 - \lambda - 2$$
			
			On cherche les racines du polynôme : $\Delta = (-1)^2 - 4(1)(-2) = 1 + 8 = 9$.
			Les solutions sont $\lambda_1 = \frac{1 - 3}{2} = -1$ et $\lambda_2 = \frac{1 + 3}{2} = 2$.
			
			Ainsi : $\det(A_\lambda) = (\lambda + 1)(\lambda - 2)$.
		}
		
		\item   \question{Pour quelles valeurs de $\lambda$ le système admet-il une solution unique ?}
		\reponse{
			D'après le théorème de Cramer, le système admet une solution unique si et seulement si $\det(A_\lambda) \neq 0$.
			Le système admet donc une solution unique pour tout $\lambda \in \mathbb{R} \setminus \{-1, 2\}$.
		}
		
		\item   \question{Pour $\lambda = 2$, le système a-t-il des solutions ? Si oui, déterminer l'ensemble des solutions.}
		\indication{Le déterminant est nul, il n'y a pas de solution unique. Étudier les équations.}
		\reponse{
			Pour $\lambda = 2$, on a $\det(A_2) = 0$. Le système s'écrit :
			\begin{equation*}
				\begin{cases}
					x + 2z = 1 \quad (L_1) \\
					y = 2 \\
					x + 2z = 1 \quad (L_3)
				\end{cases}
			\end{equation*}
			Les lignes $L_1$ et $L_3$ sont identiques. Le système se réduit à deux équations :
			$y = 2$ et $x = 1 - 2z$.
			
			Le système est **compatible** mais indéterminé (une infinité de solutions). On choisit $z$ comme paramètre libre.
			L'ensemble des solutions est :
			$$\mathcal{S} = \left\{ (1 - 2k, 2, k) \mid k \in \mathbb{R} \right\}$$
		}
		
		\item   \question{Pour $\lambda = -1$, le système a-t-il des solutions ? Si oui, déterminer l'ensemble des solutions.}
		\reponse{
			Pour $\lambda = -1$, on a $\det(A_{-1}) = 0$. Le système s'écrit :
			\begin{equation*}
				\begin{cases}
					-2x + 2z = 1 \\
					y = 2 \\
					x - z = 1
				\end{cases}
			\end{equation*}
			La troisième équation donne $x - z = 1 \iff -2x + 2z = -2$ (en multipliant par -2).
			
			En comparant avec la première équation ($-2x + 2z = 1$), on obtient une contradiction : $1 = -2$.
			
			Le système est **incompatible**. L'ensemble des solutions est vide : $\mathcal{S} = \emptyset$.
		}
		
		\item   \question{Pour $\lambda = 0$, résoudre le système.}
		\reponse{
			Pour $\lambda = 0$, $\det(A_0) = -2 \neq 0$, donc la solution est unique. Le système devient :
			\begin{equation*}
				\begin{cases}
					-x + 2z = 1 \quad (1) \\
					y = 2 \\
					x = 1 \quad (3)
				\end{cases}
			\end{equation*}
			On a directement $x=1$ et $y=2$.
			
			On remplace $x$ dans l'équation (1) :
			$-1 + 2z = 1 \implies 2z = 2 \implies z = 1$.
			
			La solution unique est $(x, y, z) = (1, 2, 1)$.
		}
	\end{enumerate}
}