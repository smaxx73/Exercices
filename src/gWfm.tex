\uuid{gWfm}
\exo7id{4363}
\auteur{quercia}
\organisation{exo7}
\datecreate{2010-03-12}
\isIndication{false}
\isCorrection{true}
\chapitre{Intégration}
\sousChapitre{Intégrale de Riemann dépendant d'un paramètre}

\contenu{
\texte{
Soit $\alpha \in \R$. Trouver la limite de $u_n=\sum_{k=1}^n \frac{\sin (k\alpha)}{n+k}\cdotp$
}
\reponse{
Si $\alpha\in2\pi\,\Z$ alors $u_n=0$ pour tout~$n$.

Sinon, $u_n = \Im\Bigl( \int_{t=0}^1 \sum_{k=1}^nt^{n+k-1}e^{ik\alpha}\,d t\Bigr)
=\Im\Bigl( \int_{t=0}^1\frac{t^ne^{i\alpha}-t^{2n}e^{i(n+1)\alpha}}{1-te^{i\alpha}}\,d t\Bigr)
\to 0$ (lorsque $n\to\infty$) par convergence dominée.
}
}
