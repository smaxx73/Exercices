\titre{Intervalles de confiance}
\theme{statistiques}
\auteur{Maxime NGUYEN}
\organisation{AMSCC}

\texte{Une association de consommateurs a effectué une enquête sur le prix d'un produit dans le supermarchés. Pour ce produit, les prix suivants (en euros) ont été relevés dans 7 supermarchés différents :
 $$52 \qquad 52 \qquad 43 \qquad 51 \qquad 69 \qquad 55 \qquad 49$$
}
 \begin{enumerate}
  \item \question{En supposant que le prix $X$ de ce produit est distribué selon une loi normale $\mathcal{N}(\mu,\sigma^2)$, calculer une estimation $\overline{x}$ de l'espérance $\EX$ et une estimation $s$ de l'écart-type $\sigma(X)$. On précisera les estimateurs choisis et on en donnera les propriétés.}
  \item \question{Déterminer les intervalles de confiances symétriques au seuil de $90\%$ et $99\%$ centrés en $\mu = \EX$.}
 \end{enumerate}