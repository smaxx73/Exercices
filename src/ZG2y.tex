\uuid{ZG2y}
\exo7id{3731}
\auteur{quercia}
\organisation{exo7}
\datecreate{2010-03-11}
\isIndication{false}
\isCorrection{true}
\chapitre{Espace euclidien, espace normé}
\sousChapitre{Forme quadratique}

\contenu{
\texte{
Soit $q$ une forme quadratique non nulle sur $M_2(\C)$ telle que $q(AB)=q(A) q(B)$. Déterminer $q$.
}
\reponse{
Soit $(E_{ij})$ la base canonique de~$\mathcal{M}_2(\C)$~:
$E_{12}^2=0$ donc $q(E_{12}) = 0$ et si $A$ est une matrice quelconque
de rang~$1$, $A$ est équivalente à~$E_{12}$ d'où $q(A) = 0$.
Si $A=0$ on a aussi $q(A)=0$ et si $A$ est inversible alors toute matrice
est multiple de~$A$ donc $q(A)\ne 0$, en particulier $q(I)=1$
car $q^2(I)=q(I)$. On en déduit $q(A) = 0 \Leftrightarrow\det(A)=0$.

Pour $A$ quelconque, les applications~: $z \mapsto\det(A-zI)$ et $z \mapsto q(A-zI)$
sont polynomiales de degré~$2$, avec le même coefficient de~$z^2$
et les mêmes racines, donc sont égales d'où $q=\det$.

\emph{Remarque :} le même raisonnement est applicable sur un corps quelconque en se limitant
aux matrices triangulaires, et toute matrice est produit de triangulaires
(algorithme du pivot de Gauss).
}
}
