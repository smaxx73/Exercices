\uuid{hGvI}
\exo7id{3028}
\auteur{quercia}
\organisation{exo7}
\datecreate{2010-03-08}
\isIndication{false}
\isCorrection{false}
\chapitre{Groupe, anneau, corps}
\sousChapitre{Algèbre, corps}

\contenu{
\texte{
Soit $ K$ un corps fini de cardinal $n$.
Si $a,b\in \N$ sont tels que $ab = n-1$, on consid{\`e}re l'application
${f_a} : { K^*} \to { K^*}, x \mapsto {x^a}$ (remarquer que $f_a$ est
un morphisme de groupe). On note $N_a = \mathrm{Card}\,(\mathrm{Ker} f_a)$.
}
\begin{enumerate}
    \item \question{Expliquer pourquoi $N_a \le a$.}
    \item \question{Montrer que $\Im(f_a) \subset \mathrm{Ker} f_b$. En d{\'e}duire que $N_a = a$ et $N_b=b$.}
    \item \question{Soit $\varphi$ l'indicateur d'Euler. Montrer par r{\'e}currence sur $a$,
    diviseur de $n-1$, que le nombre d'{\'e}l{\'e}ments de $ K^*$ d'ordre~$a$ est
    {\'e}gal {\`a} $\varphi(a)$ (ceci prouve que $ K^*$ est cyclique).}
\end{enumerate}
}
