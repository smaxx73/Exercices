\uuid{6YqE}
\chapitre{Statistique}
\niveau{L2}
\module{Probabilité et statistique}
\sousChapitre{Tests d'hypothèses, intervalle de confiance}
\titre{Contrôle qualité : test de proportion}
\theme{statistiques, tests d'hypothèses, proportion}
\auteur{}
\datecreate{2022-10-17}
\organisation{AMSCC}
\difficulte{3}
\contenu{
	\texte{Une machine de production fabrique des pièces en grande série. 
		Lorsque la machine est bien réglée, la proportion de pièces défectueuses est $p_0=0.05$. 
		Lorsqu'elle est déréglée, cette proportion passe à $p_1=0.10$. 
		
		Pour vérifier l'état de la machine, on prélève un échantillon aléatoire de $n=100$ pièces 
		dans la production, et on observe $k=8$ pièces défectueuses.
		
		On note $\widehat{p} = \frac{k}{n}$ la fréquence observée de pièces défectueuses 
		et on souhaite tester :
		\begin{align*}
			H_0 &: p = p_0 = 0.05 \quad \text{(la machine est bien réglée)} \\
			H_1 &: p = p_1 = 0.10 \quad \text{(la machine est déréglée)}
		\end{align*}
	}
	
	\begin{enumerate}
		\item \question{ Justifier que, sous $H_0$, la statistique de test 
			$$T_n = \frac{\widehat{p} - p_0}{\sqrt{\frac{p_0(1-p_0)}{n}}}$$ 
			suit approximativement une loi $\mathcal{N}(0,1)$ pour $n$ assez grand. }
		
		\reponse{
			Soit $X_i$ la variable indicatrice valant 1 si la $i$-ème pièce est défectueuse, 0 sinon.
			On a $X_i \sim \mathcal{B}(p)$ et $\widehat{p} = \frac{1}{n}\sum_{i=1}^n X_i$.
			
			D'après le théorème central limite :
			$$\frac{\widehat{p} - \mathbb{E}[\widehat{p}]}{\sqrt{\text{Var}(\widehat{p})}} 
			= \frac{\widehat{p} - p}{\sqrt{\frac{p(1-p)}{n}}} 
			\xrightarrow[n \to \infty]{\mathcal{L}} \mathcal{N}(0,1)$$
			
			Sous $H_0$, on a $p = p_0$, donc :
			$$T_n = \frac{\widehat{p} - p_0}{\sqrt{\frac{p_0(1-p_0)}{n}}} \stackrel{H_0}{\sim} \mathcal{N}(0,1) \text{ approximativement}$$
			
			Les conditions d'application sont vérifiées car :
			\begin{itemize}
				\item $np_0 = 100 \times 0.05 = 5 \geq 5$ ✓
				\item $n(1-p_0) = 100 \times 0.95 = 95 \geq 5$ ✓
			\end{itemize}
		}
		
		\item \question{ On note $\pi$ la valeur critique : on rejette $H_0$ si $\widehat{p} > \pi$. 
			En utilisant la définition du risque de première espèce, exprimer $\alpha$ en fonction 
			de $\pi$, $p_0$ et $n$. 
			
			\textit{Rappel : } $\alpha = \mathbb{P}(\text{rejeter } H_0 | H_0 \text{ vraie})$ }
		
		\reponse{
			Le risque de première espèce est la probabilité de rejeter $H_0$ alors qu'elle est vraie :
			\begin{align*}
				\alpha &= \mathbb{P}_{H_0}(\widehat{p} > \pi) \\
				&= \mathbb{P}_{H_0}\left(\frac{\widehat{p} - p_0}{\sqrt{\frac{p_0(1-p_0)}{n}}} > \frac{\pi - p_0}{\sqrt{\frac{p_0(1-p_0)}{n}}}\right) \\
				&= \mathbb{P}\left(Z > \frac{\pi - p_0}{\sqrt{\frac{p_0(1-p_0)}{n}}}\right) 
				\quad \text{où } Z \sim \mathcal{N}(0,1) \\
				&= 1 - \Phi\left(\frac{\pi - p_0}{\sqrt{\frac{p_0(1-p_0)}{n}}}\right)
			\end{align*}
			
			Avec $p_0 = 0.05$ et $n = 100$ :
			$$\alpha = 1 - \Phi\left(\frac{\pi - 0.05}{\sqrt{\frac{0.05 \times 0.95}{100}}}\right) 
			= 1 - \Phi\left(\frac{\pi - 0.05}{0.0218}\right)$$
		}
		
		\indication{On peut calculer $\sqrt{\frac{0.05 \times 0.95}{100}} \approx 0.0218$ avec un tableur.}
		
		\item \question{ Pour un seuil $\alpha = 0.05$, déterminer la valeur critique $\pi$ 
			et en déduire la région d'acceptation de l'hypothèse $H_0$ en termes de fréquence observée.
 }
		
		\reponse{
			On cherche $\pi$ tel que $\alpha = 0.05$.
			
			D'après la question précédente :
			$$1 - \Phi\left(\frac{\pi - 0.05}{0.0218}\right) = 0.05$$
			
			Donc :
			$$\Phi\left(\frac{\pi - 0.05}{0.0218}\right) = 0.95$$
			
			Or $\Phi(1.645) = 0.95$, donc :
			$$\frac{\pi - 0.05}{0.0218} = 1.645$$
			
			D'où :
			$$\pi = 0.05 + 1.645 \times 0.0218 = 0.05 + 0.0359 = 0.0859$$
			
			\textbf{Région d'acceptation de $H_0$ :} $\widehat{p} \leq 0.0859$, soit environ $8.6\%$.
			
			\textbf{Région de rejet de $H_0$ :} $\widehat{p} > 0.0859$.
		}
		\indication{On pourra utiliser :  $\Phi(1.645) \approx 0.95$ où $\Phi$ désigne 
		la fonction de répartition de la loi $\mathcal{N}(0,1)$.}
	
		\indication{Avec un tableur : \texttt{=0.05 + 1.645*RACINE(0.05*0.95/100)}}
		
		\item \question{ Au vu des données ($\widehat{p} = 0.08$), peut-on conclure 
			que la machine est bien réglée au seuil $\alpha = 5\%$ ? 
			Calculer également la $p$-valeur associée à ce test. }
		
		\reponse{
			On observe $\widehat{p} = \frac{8}{100} = 0.08$.
			
			\textbf{Décision :} Puisque $\widehat{p} = 0.08 < 0.0859 = \pi$, 
			on se situe dans la région d'acceptation. 
			\textbf{On ne rejette pas $H_0$ au seuil $\alpha = 5\%$.}
			
			\textbf{Conclusion :} Les données ne permettent pas de conclure que la machine est déréglée. 
			Au niveau de confiance de 95\%, on peut considérer que la machine est bien réglée.
			
			\textbf{Calcul de la $p$-valeur :}
			
			La $p$-valeur est la probabilité d'observer une valeur au moins aussi extrême 
			que celle observée sous $H_0$ :
			\begin{align*}
				p\text{-valeur} &= \mathbb{P}_{H_0}(\widehat{p} \geq 0.08) \\
				&= \mathbb{P}\left(Z \geq \frac{0.08 - 0.05}{0.0218}\right) \\
				&= \mathbb{P}(Z \geq 1.376) \\
				&= 1 - \Phi(1.376) \\
				&\approx 0.084
			\end{align*}
			
			La $p$-valeur est d'environ $8.4\%$, ce qui est supérieur à $\alpha = 5\%$, 
			confirmant la non-rejet de $H_0$.
		}
		
		\indication{Avec un tableur : 
			\begin{itemize}
				\item Statistique de test : \texttt{=(0.08-0.05)/RACINE(0.05*0.95/100)} donne $\approx 1.376$
				\item $p$-valeur : \texttt{=1-LOI.NORMALE.STANDARD(1.376)} donne $\approx 0.084$
			\end{itemize}
		}
		
		\item \question{ Calculer le risque de seconde espèce $\beta$, 
			c'est-à-dire la probabilité d'accepter $H_0$ alors que la machine est effectivement déréglée.}
			
			\indication{ $\beta = \mathbb{P}(\text{accepter } H_0 | H_1 \text{ vraie})$ }
			
		
		\reponse{
			Le risque de seconde espèce est :
			\begin{align*}
				\beta &= \mathbb{P}_{H_1}(\widehat{p} \leq \pi) \\
				&= \mathbb{P}_{H_1}(\widehat{p} \leq 0.0859)
			\end{align*}
			
			Sous $H_1$, on a $p = p_1 = 0.10$, donc :
			$$\frac{\widehat{p} - p_1}{\sqrt{\frac{p_1(1-p_1)}{n}}} \sim \mathcal{N}(0,1)$$
			
			L'écart-type sous $H_1$ est :
			$$\sigma_1 = \sqrt{\frac{0.10 \times 0.90}{100}} = \sqrt{0.0009} = 0.03$$
			
			Donc :
			\begin{align*}
				\beta &= \mathbb{P}_{H_1}\left(\frac{\widehat{p} - 0.10}{0.03} \leq \frac{0.0859 - 0.10}{0.03}\right) \\
				&= \mathbb{P}\left(Z \leq \frac{-0.0141}{0.03}\right) \\
				&= \mathbb{P}(Z \leq -0.47) \\
				&= \Phi(-0.47) \\
				&= 1 - \Phi(0.47) \\
				&\approx 0.319
			\end{align*}
			
			Le risque de seconde espèce est d'environ \textbf{32\%}.
			
			\textbf{Interprétation :} Si la machine est effectivement déréglée (avec $p=0.10$), 
			il y a environ 32\% de chances de ne pas le détecter avec ce plan de contrôle.
			
			La \textbf{puissance du test} est $1 - \beta \approx 0.68$, soit 68\%.
		}
		
		\indication{Avec un tableur : 
			\begin{itemize}
				\item Écart-type sous $H_1$ : \texttt{=RACINE(0.10*0.90/100)} donne $0.03$
				\item Statistique : \texttt{=(0.0859-0.10)/0.03} donne $-0.47$
				\item $\beta$ : \texttt{=LOI.NORMALE.STANDARD(-0.47)} donne $\approx 0.319$
			\end{itemize}
		}
		
		\item \question{ On souhaite désormais dimensionner le plan de contrôle. 
			Quelle taille d'échantillon $n$ permettrait de rejeter $H_0$ avec une fréquence observée 
			de $\widehat{p} = 0.08$, tout en conservant un risque de première espèce $\alpha = 5\%$ ? 
			
			Interpréter ce résultat : pourquoi faut-il augmenter la taille de l'échantillon ? }
		
		\reponse{
			On cherche $n$ tel que $\widehat{p} = 0.08$ soit dans la région critique pour $\alpha = 5\%$.
			
			La condition de rejet est :
			$$\frac{\widehat{p} - p_0}{\sqrt{\frac{p_0(1-p_0)}{n}}} > z_{1-\alpha}$$
			
			avec $z_{0.95} = 1.645$.
			
			On veut que $\widehat{p} = 0.08$ soit juste à la limite, donc :
			$$\frac{0.08 - 0.05}{\sqrt{\frac{0.05 \times 0.95}{n}}} = 1.645$$
			
			D'où :
			$$\frac{0.03}{\sqrt{\frac{0.0475}{n}}} = 1.645$$
			
			$$0.03 = 1.645 \times \sqrt{\frac{0.0475}{n}}$$
			
			$$\frac{0.03}{1.645} = \sqrt{\frac{0.0475}{n}}$$
			
			$$\left(\frac{0.03}{1.645}\right)^2 = \frac{0.0475}{n}$$
			
			$$n = \frac{0.0475}{\left(\frac{0.03}{1.645}\right)^2} 
			= \frac{0.0475 \times (1.645)^2}{(0.03)^2}$$
			
			$$n = \frac{0.0475 \times 2.706}{0.0009} = \frac{0.1285}{0.0009} \approx 143$$
			
			Il faudrait un échantillon d'au moins \textbf{143 pièces}.
			
			\textbf{Interprétation :}
			\begin{itemize}
				\item Avec $n=100$, une fréquence de 8\% n'est pas statistiquement significative 
				car elle pourrait résulter de fluctuations d'échantillonnage.
				\item En augmentant la taille de l'échantillon à $n=143$, on réduit la variabilité 
				d'échantillonnage, ce qui permet de détecter des écarts plus faibles.
				\item Plus l'échantillon est grand, plus on peut détecter de petites différences 
				entre $p_0$ et $p_1$.
				\item Compromis à trouver entre : précision du test vs. coût du contrôle.
			\end{itemize}
		}
		
		\indication{Avec un tableur : \texttt{=0.0475*(1.645/0.03)\^{}2} donne $\approx 142.7$}
		
		\item \textbf{[Bonus]} \question{ Représenter graphiquement sur un même schéma :
			\begin{itemize}
				\item la densité de probabilité de $\widehat{p}$ sous $H_0$
				\item la densité de probabilité de $\widehat{p}$ sous $H_1$
				\item la région critique
				\item les risques $\alpha$ et $\beta$
			\end{itemize}
			Ce graphique permet de visualiser la \textit{puissance} du test. }
		
		\reponse{
			Voici les éléments à représenter :
			
			\textbf{Distributions :}
			\begin{itemize}
				\item Sous $H_0$ : $\widehat{p} \sim \mathcal{N}(0.05, \sqrt{0.0475/100})$ 
				(courbe centrée en 0.05)
				\item Sous $H_1$ : $\widehat{p} \sim \mathcal{N}(0.10, \sqrt{0.09/100})$ 
				(courbe centrée en 0.10)
			\end{itemize}
			
			\textbf{Région critique :} Zone $\widehat{p} > 0.0859$ (à droite de la valeur critique)
			
			\textbf{Visualisation des risques :}
			\begin{itemize}
				\item $\alpha$ (risque de 1ère espèce) : aire sous la courbe de $H_0$ 
				à droite de $\pi = 0.0859$ (zone hachurée en rouge) $\approx 5\%$
				\item $\beta$ (risque de 2ème espèce) : aire sous la courbe de $H_1$ 
				à gauche de $\pi = 0.0859$ (zone hachurée en bleu) $\approx 32\%$
				\item Puissance $= 1-\beta$ : aire sous la courbe de $H_1$ 
				à droite de $\pi$ $\approx 68\%$
			\end{itemize}
			
			\textbf{Code pour un tableur ou Python :}
			
			On peut générer ce graphique avec les valeurs de $\widehat{p}$ 
			allant de 0 à 0.15, et calculer les densités correspondantes.
		}
		
		\indication{
			Suggestions pour le tracé :
			\begin{itemize}
				\item Créer une colonne de valeurs de $\widehat{p}$ de 0 à 0.15 par pas de 0.001
				\item Calculer $f_0(\widehat{p})$ : densité sous $H_0$ avec \texttt{LOI.NORMALE(p; 0.05; 0.0218; FAUX)}
				\item Calculer $f_1(\widehat{p})$ : densité sous $H_1$ avec \texttt{LOI.NORMALE(p; 0.10; 0.03; FAUX)}
				\item Tracer les deux courbes
				\item Ajouter une ligne verticale en $\pi = 0.0859$
				\item Colorier les zones correspondant à $\alpha$ et $\beta$
			\end{itemize}
		}
	\end{enumerate}
}