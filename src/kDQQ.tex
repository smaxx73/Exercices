\uuid{kDQQ}
\exo7id{2444}
\auteur{matexo1}
\organisation{exo7}
\datecreate{2002-02-01}
\isIndication{true}
\isCorrection{true}
\chapitre{Matrice}
\sousChapitre{Propriétés élémentaires, généralités}

\contenu{
\texte{
Soient $A, B$ deux matrices semblables (i.e. il existe $P$
inversible telle que $B = P^{-1} A P$). Montrer que si l'une est
inversible, l'autre aussi\,; que si l'une est idempotente, l'autre
aussi\,; que si l'une est nilpotente, l'autre aussi\,; que si $A =
\lambda I$, alors $A = B$.
}
\indication{$A$ est \emph{idempotente} s'il existe un $n$ tel que $A^n=I$ (la matrice identité).

$A$ est \emph{nilpotente} s'il existe un $n$ tel que $A^n=(0)$ (la matrice nulle).}
\reponse{
Supposons $A$ inversible, alors il existe $A'$ tel que $A\times A'=I$ et $A'\times A=I$.
Notons alors $B'= P^{-1} A'P$. On a
$$B \times B' = \big(P^{-1}A P \big)\times \big(P^{-1} A' P\big)=P^{-1}A \big(P P^{-1}\big)A' P 
= P^{-1}A A' P=P^{-1} I P=I$$
De même $B' \times B=I$. Donc $B$ est inversible d'inverse $B'$.
Supposons que $A^n=I$. Alors 
$$\begin{array}{rcl}
B^n 
&=&\big(P^{-1} A P\big)^n= \big(P^{-1} A P \big)\big(P^{-1} A P \big)\cdots \big(P^{-1} A P \big) \\
&=&P^{-1} A (P P^{-1}) A (P P^{-1}) \cdots  A P \\
&=& P^{-1} A^n P  \\
&=& P^{-1} I P = I \\
\end{array}$$
Donc $B$ est idempotente.
Si $A^n=(0)$ alors le même calcul qu'au-dessus conduit à $B^n=(0)$.
Si $A = \lambda I$ alors $B=P^{-1} (\lambda I) P = \lambda I \times P^{-1}P= \lambda I$
(car la matrice $\lambda I$ commute avec toutes les matrices).
}
}
