\uuid{LBBg}
\exo7id{3841}
\auteur{quercia}
\organisation{exo7}
\datecreate{2010-03-11}
\isIndication{false}
\isCorrection{true}
\chapitre{Espace euclidien, espace normé}
\sousChapitre{Espaces vectoriels hermitiens}

\contenu{
\texte{
Soit $E = \mathcal{C}([a,b],\R)$ et $ u : {[a,b]} \to \R$ une fonction continue
par morceaux. On pose pour $f,g \in E$ :
$(f\mid g) =  \int_{t=a}^b u(t)f(t)g(t)\,d t$.
}
\begin{enumerate}
    \item \question{A quelle condition sur $u$ définit-on ainsi un produit scalaire ?}
\reponse{$u \ge 0$ et $u^{-1}(0)$ est d'intérieur vide.}
    \item \question{Soient $u,v$ deux fonctions convenables. A quelle condition les normes
    associées sont-elles équivalentes~?}
\reponse{Il existe $\alpha,\beta > 0$ tels que $\alpha u \le v \le \beta u$.}
\end{enumerate}
}
