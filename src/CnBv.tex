\uuid{CnBv}
\titre{Etude d'un fonction de deux variables}
\theme{}
\auteur{Grégoire MENET}
\datecreate{2025-04-16}
\organisation{AMSCC}

\contenu{
	
	\texte{
		On considère trois fonctions de deux variables \(f\), \(g\) et \(h\) dont les expressions ne sont pas connues.
		Les domaines de définition de ces fonctions sont respectivement :
		\[
		D_f = \left\{(x,y) \in \mathbb{R}^2 \ \middle|\ x + y \geq 0 \text{ et } x^2 \neq y \right\}
		\]
		\[
		D_g = \left\{(x,y) \in \mathbb{R}^2 \ \middle|\ x \neq y \text{ et } x^2 + y^2 > 1 \right\}
		\]
		\[
		D_h = \left\{(x,y) \in \mathbb{R}^2 \ \middle|\ x + y \neq 1 \right\}
		\]
	}
	
	\begin{enumerate}
		\item \question{Tracer ces domaines de définition. Pour chacun de ces domaines, déterminer s'il est ouvert, fermé ou ni l'un ni l'autre. Justifier brièvement.}
		\indication{Tracer les ensembles en considérant les inégalités et exclusions indiquées dans chaque domaine. Analyser les conditions d'inclusion des frontières pour déterminer si les ensembles sont ouverts ou fermés.}
		\reponse{}
		
		\item \question{On sait que \(f\) est de classe \(\mathcal{C}^1\) sur son domaine de définition et que \(f(x^2,0)=\frac{1}{x^3}\) pour \(x>0\). À l'aide de la dérivée d'une composée, déduire \(\frac{\partial f}{\partial x}(1,0)\).}
		\indication{Appliquer la règle de dérivation en chaîne à l'expression \(f(x^2,0)\) et égaler la dérivée obtenue à celle de \(\frac{1}{x^3}\) pour \(x>0\).}
		\reponse{}
		
		\item \question{On sait que \(\lim\limits_{(x,y)\rightarrow(2,0)} g(x,y)=\ln\left(\frac{2}{3}\right)\) et que les dérivées partielles de \(g\) en \((2,0)\) existent. Peut-on en conclure que \(g(2,0)=\ln\left(\frac{2}{3}\right)\) ? Justifier votre réponse.}
		\indication{Étudier la continuité de \(g\) en \((2,0)\) en considérant l'existence des dérivées partielles et le comportement limite, et déterminer si ces informations sont suffisantes pour conclure à la continuité en ce point.}
		\reponse{}
		
		\item \question{À laquelle des trois fonctions correspond le graphe suivant ? Justifier.
	
		\includegraphics[width=250pt]{exam}	
	}
		
		\indication{Comparer les caractéristiques visibles du graphe avec les propriétés attendues des fonctions \(f\), \(g\) et \(h\), en tenant compte notamment des domaines de définition et du comportement des fonctions.}
		

	\end{enumerate}
	
}
