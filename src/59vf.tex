\uuid{59vf}
\exo7id{1762}
\auteur{maillot}
\organisation{exo7}
\datecreate{2001-09-01}
\isIndication{false}
\isCorrection{false}
\chapitre{Topologie}
\sousChapitre{Ouvert, fermé, intérieur, adhérence}

\contenu{
\texte{
Soit $E$ un espace vectoriel norm\'e. Soit $A$ une partie non vide et
born\'ee de $E$. On d\'efinit $\mathrm{diam}(A)=\sup \{\|y-x\|, x,y\in A\}$.
}
\begin{enumerate}
    \item \question{Montrer que si $A$ est born\'ee, alors $\bar A$ et $\mathrm{Fr}(A)$
sont born\'es.}
    \item \question{Comparer $\mathrm{diam}(A)$, $\mathrm{diam}(\stackrel{\circ}{A})$ et $\mathrm{diam}(\bar A)$
lorsque $\stackrel{\circ}{A}$ est non vide.}
    \item \question{\begin{enumerate}}
    \item \question{Montrer que $\mathrm{diam}(\mathrm{Fr}(A)) \le \mathrm{diam}(A)$.}
    \item \question{Soit $x$ et $u$ des \'el\'ements de $A$ avec $u\neq 0$. On
consid\`ere l'ensemble $X=\{t\ge 0 \mid x+tu\in A\}$. Montrer que
$\sup X$ existe.}
    \item \question{En d\'eduire que toute demi-droite issue d'un point $x$ de
$A$ coupe $\mathrm{Fr}(A)$.}
    \item \question{En d\'eduire que $\mathrm{diam}(\mathrm{Fr}(A)) = \mathrm{diam} (A)$.}
\end{enumerate}
}
