\uuid{66fn}
\exo7id{615}
\auteur{ridde}
\organisation{exo7}
\datecreate{1999-11-01}
\isIndication{false}
\isCorrection{false}
\chapitre{Continuité, limite et étude de fonctions réelles}
\sousChapitre{Limite de fonctions}

\contenu{
\texte{
Soit $f : \Rr^+  \rightarrow \Rr^+ $ croissante telle que $ \lim\limits_{
x \rightarrow  + \infty}f (x + 1)-f (x) = 0 $. Montrer que $\lim\limits_{
x \rightarrow  + \infty}\dfrac{f (x)}x = 0$.
 (on pourra utiliser des $\epsilon$, sommer des in\'egalit\'es et utiliser la
 monotonie de $f$ pour montrer qu'elle est born\'ee sur un segment).\\
 Comment g\'en\'eraliser ce r\'esultat ?
}
}
