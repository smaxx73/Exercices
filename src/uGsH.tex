\chapitre{Probabilité continue}
\sousChapitre{Densité de probabilité}
\uuid{uGsH}
\titre{Calcul avec une densité de probabilité}
\theme{variables aléatoires à densité}
\auteur{}
\datecreate{2024-09-09}
\organisation{AMSCC}	

\contenu{


\texte{ Soit la fonction densité de probabilité $f$ définie sur l'intervalle $[0, 2]$ par :

$$
f(x) =
\begin{cases}
    \frac{3}{16}(4 - x^2), & \text{si } 0 \leq x \leq 2, \\
    0, & \text{sinon}.
\end{cases}
$$ }

\begin{enumerate}
    \item \question{ Vérifier que $f$ est bien une densité de probabilité. On note $X$ une variable aléatoire admettant $f$ pour densité de probabilité. }
    
    \reponse{
    Pour qu'une fonction soit une densité de probabilité, elle doit satisfaire les deux conditions suivantes :
    \begin{enumerate}
        \item $f(x) \geq 0$ pour tout $x$ dans l'intervalle de définition.
        \item L'intégrale de $f(x)$ sur tout l'intervalle $[0, 2]$ doit être égale à 1.
    \end{enumerate}
    
    Vérifions la première condition : \\
    $f(x) = \frac{3}{16}(4 - x^2) \geq 0$ sur $[0, 2]$, car $x^2 \leq 4$ pour 
    $0\leq x \leq 2$.
    
    Vérifions maintenant la seconde condition :
    
    \[
    \int_0^2 \frac{3}{16}(4 - x^2) \, dx = \frac{3}{16} \int_0^2 (4 - x^2) \, dx = \frac{3}{16} \left[ 4x - \frac{x^3}{3} \right]_0^2.
    \]
    
    En évaluant les bornes :
    
    \[
    \left[ 4x - \frac{x^3}{3} \right]_0^2 = \left( 8 - \frac{8}{3} \right) - (0) = \frac{24}{3} - \frac{8}{3} = \frac{16}{3}.
    \]
    
    Ainsi, 
    \[
    \int_0^2 f(x) \, dx = \frac{3}{16} \times \frac{16}{3} = 1.
    \]
    
    Donc, $f$ est bien une densité de probabilité.
    }

    \item \question{ Calculer la probabilité que $X$ prenne une valeur dans l'intervalle $[1, 3]$. }
    \reponse{
    On cherche $\prob(1 \leq X \leq 3)$, ce qui revient à calculer l'intégrale de $f(x)$ sur l'intervalle $[1, 3]$ :
    
    \[
    \prob(1 \leq X \leq 3) = \int_1^3 f(x) \, dx = \int_1^2 \frac{3}{16}(4 - x^2) \, dx = \frac{5}{16}
    \]
    }
\item \question{ Calculer l'espérance et la variance de $X$. }
\reponse{ 
	\begin{align*}
		\mathbb{E}[X] &= \int_{0}^{2} x f(x) \, dx \\
		&= \int_{0}^{2} x \left( \frac{3}{16}(4 - x^2) \right) dx \\
		&= \frac{3}{16} \int_{0}^{2} x (4 - x^2) \, dx \\
		&= \frac{3}{16} \int_{0}^{2} (4x - x^3) \, dx \\
		&= \frac{3}{16} \left( \int_{0}^{2} 4x \, dx - \int_{0}^{2} x^3 \, dx \right) \\
		&= \frac{3}{16} \left( \left[2x^2\right]_{0}^{2} - \left[\frac{x^4}{4}\right]_{0}^{2} \right) \\
		&= \frac{3}{16} \left( \left(2 \times 2^2\right) - \left(\frac{2^4}{4}\right) - \left(0 - 0\right) \right) \\
		&= \frac{3}{16} \left( 8 - 4 \right) \\
		&= \frac{3}{4}.
	\end{align*}

\begin{align*}
	\operatorname{Var}(X) &= \mathbb{E}[X^2] - (\mathbb{E}[X])^2 \\
	\\
	\text{Calcul de } \mathbb{E}[X^2]: \\
	\mathbb{E}[X^2] &= \int_{0}^{2} x^2 f(x) \, dx \\
	&= \int_{0}^{2} x^2 \left( \frac{3}{16}(4 - x^2) \right) dx \\
	&= \frac{3}{16} \int_{0}^{2} x^2 (4 - x^2) \, dx \\
	&= \frac{3}{16} \int_{0}^{2} (4x^2 - x^4) \, dx \\
	&= \frac{3}{16} \left( \int_{0}^{2} 4x^2 \, dx - \int_{0}^{2} x^4 \, dx \right) \\
	&= \frac{3}{16} \left( \left[\frac{4x^3}{3}\right]_{0}^{2} - \left[\frac{x^5}{5}\right]_{0}^{2} \right) \\
	&= \frac{3}{16} \left( \left( \frac{4 \times 8}{3} \right) - \left( \frac{32}{5} \right) - \left( 0 - 0 \right) \right) \\
	&= \frac{3}{16} \left( \frac{32}{3} - \frac{32}{5} \right) \\
	&= \frac{4}{5}.
\end{align*}

\begin{align*}
	\text{Calcul de } \operatorname{Var}(X): \\
	\operatorname{Var}(X) &= \mathbb{E}[X^2] - (\mathbb{E}[X])^2 \\
	&= \frac{4}{5} - \left( \frac{3}{4} \right)^2 \\
	&= \frac{4}{5} - \frac{9}{16} \\
	&= \frac{64}{80} - \frac{45}{80} \\
	&= \frac{19}{80}.
\end{align*}
 }
\end{enumerate}
}
