\titre{Calcul avec une densité de probabilité}
\theme{probabilités}
\auteur{}
\organisation{AMSCC}	

\contenu{


Soit la fonction densité de probabilité $f$ définie sur l'intervalle $[0, 2]$ par :

$$
f(x) =
\begin{cases}
    \frac{3}{16}(4 - x^2), & \text{si } 0 \leq x \leq 2, \\
    0, & \text{sinon}.
\end{cases}
$$

\begin{enumerate}
    \item \question{ Vérifier que $f$ est bien une densité de probabilité sur $[0, 2]$. }
    
    \reponse{
    Pour qu'une fonction soit une densité de probabilité, elle doit satisfaire les deux conditions suivantes :
    \begin{enumerate}
        \item $f(x) \geq 0$ pour tout $x$ dans l'intervalle de définition.
        \item L'intégrale de $f(x)$ sur tout l'intervalle $[0, 2]$ doit être égale à 1.
    \end{enumerate}
    
    Vérifions la première condition : \\
    $f(x) = \frac{3}{16}(4 - x^2) \geq 0$ sur $[0, 2]$, car $x^2 \leq 4$ pour 
    $0\leq x \leq 2$.
    
    Vérifions maintenant la seconde condition :
    
    \[
    \int_0^2 \frac{3}{16}(4 - x^2) \, dx = \frac{3}{16} \int_0^2 (4 - x^2) \, dx = \frac{3}{16} \left[ 4x - \frac{x^3}{3} \right]_0^2.
    \]
    
    En évaluant les bornes :
    
    \[
    \left[ 4x - \frac{x^3}{3} \right]_0^2 = \left( 8 - \frac{8}{3} \right) - (0) = \frac{24}{3} - \frac{8}{3} = \frac{16}{3}.
    \]
    
    Ainsi, 
    \[
    \int_0^2 f(x) \, dx = \frac{3}{16} \times \frac{16}{3} = 1.
    \]
    
    Donc, $f$ est bien une densité de probabilité.
    }

    \item \question{ Calculer la probabilité que $X$, une variable aléatoire de densité $f$, prenne une valeur dans l'intervalle $[1, 3]$ à l'aide d'une intégrale. }
    
    \reponse{
    On cherche $\prob(1 \leq X \leq 3)$, ce qui revient à calculer l'intégrale de $f(x)$ sur l'intervalle $[1, 3]$ :
    
    \[
    \prob(1 \leq X \leq 3) = \int_1^3 f(x) \, dx = \int_1^2 \frac{3}{16}(4 - x^2) \, dx = \frac{5}{16}
    \]
    }
\end{enumerate}
}
