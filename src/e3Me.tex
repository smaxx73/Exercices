\chapitre{Matrice}
\sousChapitre{Propriétés élémentaires, généralités}
\uuid{e3Me}
\titre{Résolution d'équation matricielle}
\theme{calcul matriciel}
\auteur{}
\datecreate{2023-01-03}
\organisation{AMSCC}
\contenu{

\texte{ Soient $A$ et $B$ deux matrices définies par $A = \begin{pmatrix} 2 & -1 \\ -1 & 4 \end{pmatrix}$ et $AB = \begin{pmatrix} 12 & 5 & -6 \\ 1 & 1 & 10 \end{pmatrix}$. }
	
\question{ Déterminer $B$. }
\indication{ Vérifier que $A$ est inversible et donner son inverse. }
\reponse{ On détermine d'abord $A^{-1}$ par pivot de Gauss ou une autre formule : $$A^{-1} = \frac{1}{7}\begin{pmatrix}
	4 & 1 \\ 1 & 2 \end{pmatrix}$$
Puis on déduit que $B = IB =  A^{-1}AB = \begin{pmatrix} 7 & 3 & -2 \\ 2 & 1 & 2 \end{pmatrix}$. 
 }}
