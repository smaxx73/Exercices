\uuid{Qc6o}
\titre{Primitives de fonctions rationnelles}
\niveau{L2}
\module{Analyse}
\chapitre{Intégration}
\sousChapitre{Intégration par parties}
\theme{fonction rationnelle, décomposition en éléments simples}
\auteur{Erwan L'Haridon}
\datecreate{2026-02-10}
\organisation{AMSCC}
\difficulte{4}

\contenu{
    \texte{L'objectif de cet exercice est de trouver l'expression explicite des primitives de \( x \to \frac{1}{(x^2 + 1)^2} \). On pose \( I_1 = \int \frac{1}{x^2 + 1} dx \) et \( I_2 = \int \frac{1}{(x^2 + 1)^2} dx \).}
    \begin{enumerate}
        \item \question{En intégrant \( I_1 \) par parties, donner une relation entre \( I_2 \) et \( I_1 \). On pourra écrire \( \frac{x^2}{(1 + x^2)^2} = \frac{a}{1 + x^2} + \frac{b}{(1 + x^2)^2} \), où \( a \) et \( b \) sont deux réels à préciser.}
        \item \question{En déduire l'expression de \( I_2 \).}
    \end{enumerate}
}
