\titre{ Fiabilité d'un système électronique }
\theme {probabilités}
\auteur{ }
\organisation{ AMSCC }

\contenu{
    \texte{ Un système électronique est constitué de $n$ composants disposés en série. Cela implique que la panne d'un composant entraîne la panne de tout le système. Chacun des composants a une durée de vie $T_k$ qui suit une loi exponentielle de paramètre $\lambda = 1$, pour tout $k \in \{1, \ldots, n\}$. On admet que les variables aléatoires $T_1, \ldots, T_n$ sont indépendantes. On note $S$ la durée de vie du système et on note $t \geq 0$ la variable de temps. }

    \begin{enumerate}
        \item \question{ 
            Soit $t \geq 0$. Pour tout $k \in \{1, \ldots, n \}$, on note $R_k(t) = \prob(T_k > t)$ la fiabilité du composant $k$ à l'instant $t$. C'est la probabilité que le composant $k$ fonctionne encore après un temps d'utilisation $t$. Déterminer $R_k(t)$.
        }
        \reponse{
            $R_k(t) = \int_{t}^{+\infty} e^{-x} dx = e^{-t}$ pour tout $t \geq 0$.
        }
    \end{enumerate}
}