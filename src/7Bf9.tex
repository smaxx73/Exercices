\uuid{7Bf9}
\titre{ Détection de menaces par radar}
\niveau{L1}
\module{Probabilités}
\chapitre{Probabilités conditionnelles}
\sousChapitre{Théorème de Bayes}
\theme{Détection, probabilité conditionnelle, fausse alarme, sensibilité}
\auteur{AMSCC}
\datecreate{2025-09-16}
\organisation{}
\difficulte{3}
\contenu{
	\texte{
		Un système de radar militaire est utilisé pour détecter des missiles ennemis. Le radar détecte un missile avec une probabilité de 95\%. Le radar déclenche une fausse alarme (détection alors qu'il n'y a pas de missile) avec une probabilité de 5\%. Selon les renseignements, la probabilité qu'un missile soit effectivement présent dans la zone surveillée est de 1\%. On note les événements $M$ : "un missile est présent" et $D$ : "le radar signale une détection".
	}
	\begin{enumerate}
		\item   \question{Que valent les probabilités $\PP(M)$, $\PP(D | M)$ et $\PP(D | \overline{M})$ ?}
		\reponse{D'après l'énoncé, on a : $\PP(M) = 0{,}01$, $\PP(D | M) = 0{,}95$ et $\PP(D | \overline{M}) = 0{,}05$.}
		
		\item   \question{Calculer la probabilité de détection $\PP(D)$.}
		\indication{Utiliser la formule des probabilités totales : $\PP(D) = \PP(D | M)\PP(M) + \PP(D | \overline{M})\PP(\overline{M})$.}
		\reponse{
			On utilise la formule des probabilités totales. On a d'abord besoin de la probabilité qu'il n'y ait pas de missile :
			$\PP(\overline{M}) = 1 - \PP(M) = 1 - 0{,}01 = 0{,}99$.
			\begin{align*}
				\PP(D) &= \PP(D | M)\PP(M) + \PP(D | \overline{M})\PP(\overline{M}) \\
				&= (0{,}95 \times 0{,}01) + (0{,}05 \times 0{,}99) \\
				&= 0{,}0095 + 0{,}0495 \\
				&= 0{,}059
			\end{align*}
			La probabilité totale de détection par le radar est de 5,9\%.
		}
		
		\item   \question{Calculer la probabilité qu'un missile soit effectivement présent si le radar a détecté une menace.}
		\indication{Appliquer le théorème de Bayes : $\PP(M | D) = \frac{\PP(D | M)\PP(M)}{\PP(D)}$.}
		\reponse{
			On applique le théorème de Bayes pour trouver la probabilité $\PP(M | D)$ :
			\begin{align*}
				\PP(M | D) &= \frac{\PP(D | M)\PP(M)}{\PP(D)} \\
				&= \frac{0{,}95 \times 0{,}01}{0{,}059} \\
				&= \frac{0{,}0095}{0{,}059} \\
				&\approx 0{,}161
			\end{align*}
			La probabilité qu'un missile soit réellement présent sachant que le radar a émis une alerte est d'environ 16,1\%.
		}
		
		\item   \question{Que se passe-t-il si la probabilité a priori d'un missile passe à 10\% ?}
		\indication{Recalculer avec $\PP(M) = 0{,}1$ et réévaluer $\PP(D)$ et $\PP(M | D)$ avec cette nouvelle valeur.}
		\reponse{
			On refait les calculs avec la nouvelle probabilité a priori $\PP(M) = 0{,}1$.
			D'abord, la nouvelle probabilité qu'il n'y ait pas de missile : $\PP(\overline{M}) = 1 - 0{,}1 = 0{,}9$.
			On recalcule la probabilité totale de détection $\PP(D)$ :
			\begin{align*}
				\PP(D) &= (0{,}95 \times 0{,}1) + (0{,}05 \times 0{,}9) \\
				&= 0{,}095 + 0{,}045 \\
				&= 0{,}14
			\end{align*}
			Ensuite, on recalcule la probabilité qu'un missile soit présent sachant la détection :
			\begin{align*}
				\PP(M | D) &= \frac{\PP(D | M)\PP(M)}{\PP(D)} \\
				&= \frac{0{,}95 \times 0{,}1}{0{,}14} \\
				&= \frac{0{,}095}{0{,}14} \\
				&\approx 0{,}679
			\end{align*}
			Si la probabilité a priori de la présence d'un missile augmente à 10\%, la probabilité qu'une alerte soit correcte monte significativement à environ 67,9\%.
		}
	\end{enumerate}
}