\titre{ Convergence vers une loi exponentielle }
\theme{probabilités}
\auteur{}
\organisation{AMSCC}

\texte{Soit une suite de variables indépendantes $(U_i)_{i \in \N^*}$ suivant chacune une loi uniforme $\mathcal{U}([0;1])$. Pour tout $n \in \N^*$, on pose $M_n = \max(U_1,...,U_n)$. }

\begin{enumerate}
	\item \question{ Démontrer que $M_n \xrightarrow[]{\mathcal{P}} 1$. }
	\item \question{ En déduire que $M_n \xrightarrow[]{\text{p.s.}} 1$ et $M_n \xrightarrow[]{\text{en loi}} 1$. }
	\item \question{ Pour tout $n \in \N^*$, on pose $Y_n = n(1-M_n)$. Démontrer que la suite $(Y_n)$ converge en loi vers une loi exponentielle dont on précisera le paramètre.  }
\end{enumerate}