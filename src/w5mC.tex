\uuid{w5mC}
\exo7id{4701}
\auteur{quercia}
\organisation{exo7}
\datecreate{2010-03-16}
\isIndication{false}
\isCorrection{true}
\chapitre{Suite}
\sousChapitre{Convergence}

\contenu{
\texte{
Soit une fonction continue~$f$ de~$\R$ dans~$\R$ et~$x_0\in\R$.
On d{\'e}finit $(x_n)_{n\in\N}$ par la relation de r{\'e}currence~: ${x_{n+1} = f(x_n)}$.
Montrer que si la suite $(x_n)$ admet une unique valeur d'adh{\'e}rence alors elle est convergente.
}
\reponse{
Soit $\ell$ une valeur d'adh{\'e}rence de~$(x_n)$.
Si l'on suppose que~$(x_n)$ ne converge pas vers~$\ell$ alors il existe un voisinage~$[a,b]$
de~$\ell$ tel qu'il y a une infinit{\'e} de termes dans~$[a,b]$ et une infinit{\'e}
hors de~$[a,b]$. Ceci implique que $[c,d] = f([a,b])$ n'est pas inclus dans~$[a,b]$
et que $[c,d]\setminus[a,b]$ contient une infinit{\'e} de termes, donc $(x_n)$
a une deuxi{\`e}me valeur d'adh{\'e}rence dans $[c,d]\setminus]a,b[$.
}
}
