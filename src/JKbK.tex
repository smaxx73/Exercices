\uuid{JKbK}
\exo7id{4167}
\auteur{quercia}
\organisation{exo7}
\datecreate{2010-03-11}
\isIndication{false}
\isCorrection{false}
\chapitre{Fonction de plusieurs variables}
\sousChapitre{Dérivée partielle}

\contenu{
\texte{
Soit $f(x,y) = x\ln x - y\ln y$ $(x,y > 0)$.

Pour $k \in \R$, on considère la courbe $\mathcal{C}_k$ d'équation $f(x,y) = k$.
}
\begin{enumerate}
    \item \question{Suivant la position de $(a,b) \in \mathcal{C}_k$, préciser l'orientation
     de la tangente à $\mathcal{C}_k$ en $(a,b)$.}
    \item \question{Dresser le tableau de variations de $\phi(t) = t\ln t$.}
    \item \question{Dessiner $\mathcal{C}_0$. (\'Etudier en particulier les points
     $(0,1), (1,0)$ et $\Bigl(\frac 1e,\frac 1e\Bigr)$ à l'aide de DL)}
    \item \question{Indiquer l'allure générale des courbes $\mathcal{C}_k$ suivant le signe de $k$.}
\end{enumerate}
}
