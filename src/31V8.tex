\uuid{31V8}
\exo7id{1212}
\auteur{legall}
\organisation{exo7}
\datecreate{1998-09-01}
\isIndication{false}
\isCorrection{false}
\chapitre{Continuité, limite et étude de fonctions réelles}
\sousChapitre{Continuité : pratique}

\contenu{
\texte{
Soit $  f   $ une fonction continue de $  [0,1]  $ dans lui-m\^eme telle que $  f(0)=0  $
et pour tout couple $  (x,y)  $ de $  [0,1]\times[0,1]  $ on ait $  \vert f(x)-f(y)\vert \geq
\vert x-y\vert   .$
}
\begin{enumerate}
    \item \question{Soit $  x   $ un \' el\' ement de $  [0,1]  .$ On pose $  x_0=x  $ et $  x_{n+1}=f(x_n)  .$ Montrer
que la suite $  (x_n)_{n \in { \Nn}}  $ est convergente.}
    \item \question{En d\' eduire que $  f(x)=x  $ pour tout $  x\in [0,1]  .$}
    \item \question{Le r\' esultat reste-t-il vrai sans l'hypoth\`ese  $  f(0)=0  ?$}
\end{enumerate}
}
