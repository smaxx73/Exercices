\titre{Calcul de probabilités}
\theme{statistiques, tableur}
\auteur{}
\organisation{AMSCC}
\contenu{


\texte{ 	A l'aide des formules du tableur, calculer : }
\begin{enumerate}
	\item 	\question{ $\PP(X \leq 1.2)$ pour $X$ suivant une $\mathcal{N}(0,1)$ }
	\item 	\question{ $\PP(X \leq 102)$ pour $X$ suivant une $\mathcal{N}(120,4)$ }
	\item \question{ la valeur $t$ telle que $\PP(X \leq t) = 90\%$ pour $X$ suivant une $\mathcal{N}(120,4)$ }
	\item \question{ 	$\PP(X \geq 40)$ pour $X$ suivant une $\mathcal{N}(50,3)$ }
	\item 	\question{ $\PP( 90 \leq X \leq 120)$ pour $X$ suivant une $\mathcal{N}(100,15)$ }
	\item 	\question{ $\PP(X \geq 5)$ pour $X$ suivant une $\mathcal{E}(1/2)$ }
	\item 	\question{ $\PP(X = 3)$ et $\PP(X \geq 2)$ pour $X$ suivant une $\mathcal{P}(1.5)$ }
	\item \question{ la valeur $t$ telle que $\PP(100-t \leq X \leq 100+t) = 95\%$ où $X$ suit une $\mathcal{N}(100,15)$ }
\end{enumerate}}
