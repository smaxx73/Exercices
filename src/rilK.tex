\uuid{rilK}
\exo7id{3341}
\auteur{quercia}
\organisation{exo7}
\datecreate{2010-03-09}
\isIndication{false}
\isCorrection{false}
\chapitre{Application linéaire}
\sousChapitre{Morphismes particuliers}

\contenu{
\texte{
Soit $E$ un $\R$-ev et $f \in \mathcal{L}(E)$ tel que $f\circ f = -\mathrm{id}_E$.
Pour $z = x + iy \in \C$ et $\vec u \in E$, on pose :
$z\vec u = x\vec u + yf(\vec u)$.
}
\begin{enumerate}
    \item \question{Montrer qu'on définit ainsi une structure de $\C$-ev sur $E$.}
    \item \question{En déduire que $\dim_{\R}(E)$ est paire.}
\end{enumerate}
}
