\titre{Calcul de dérivées partielles}
\theme{calcul différentiel}
\auteur{}
\organisation{AMSCC}

Pour chacune des fonctions suivantes, calculer  $\dpa{f}{x}$, $\dpa{f}{y}$, $\dpsp{f}{x}$, $\dpsp{f}{y}$ et $\dpsm{f}{x}{y}$ sur leur ensemble de définition. 
\begin{enumerate}
\item \question{ $f(x,y) = \frac{x^3 + 3xy}{x+y}$ }
\reponse{La fonction $f$ est polynomiale en $x$ et $y$, elle définie sur $\R^2$. Les dérivées partielles existent également pour tout $(x,y) \in \R^2$ :
	\begin{align*}
\frac{\partial f}{\partial x}(x,y) &=  \\
\frac{\partial f}{\partial y}(x,y) &=  \\
\frac{\partial^2 f}{\partial x^2}(x,y) &=  \\
\frac{\partial^2 f}{\partial y^2}(x,y) &=  \\
\frac{\partial^2 f}{\partial x \partial y}(x,y) &= \frac{\partial^2 f}{\partial y \partial x}(x,y)= -6 
\end{align*}}
\item \question{ $g(x, y) = \ln(1 + x + xy + e^{y})$ }
\reponse{La fonction $f$ est  définie sur $\{ (x,y) \in \R^2 \mid y\neq 0 \}$. Les dérivées partielles existent également pour tout $(x,y) \in \R^2$ telles que $y \neq 0$ :
	\begin{align*}
		\frac{\partial f}{\partial x}(x,y) &=  \\
		\frac{\partial f}{\partial y}(x,y) &=  \\
		\frac{\partial^2 f}{\partial x^2}(x,y) &=  \\
		\frac{\partial^2 f}{\partial y^2}(x,y) &=  \\
		\frac{\partial^2 f}{\partial x \partial y}(x,y) &= \frac{\partial^2 f}{\partial y \partial x}(x,y)=
\end{align*}}
\end{enumerate}